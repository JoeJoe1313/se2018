% arara: pdflatex: { shell: true }
% arara: biber
% arara: pdflatex: { shell: true }

\documentclass[numbers=endperiod, DIV=15]{scrartcl}

% Base packages
\usepackage[T2A]{fontenc}
\usepackage[utf8]{inputenc}
\usepackage[bulgarian]{babel}
\usepackage[pdfencoding=unicode]{hyperref}
\usepackage{biblatex}
\usepackage[style=german]{csquotes}

% Base math packages
\usepackage{amsmath}
\usepackage{amssymb}
\usepackage{amsthm}
\usepackage{mathtools}

% Custom packages
\usepackage{../../common/macros}
\usepackage{../../common/theorems}

% Bibliography
\addbibresource{./references.bib}

% Document
\title{Тема 3}
\subtitle{Полиноми на една променлива. Теорема за деление с остатък. Най-голям общ делител на полиноми - тъждество на Безу и алгоритъм на Евклид. Зависимост между корени и коефициенти на полиноми (формули на Виет).}
\author{Янис Василев, \Email{ianis@ivasilev.net}}
\date{27 март 2019}

\begin{document}

\maketitle

\section{Анотация (от \cite{Syllabus})}

\subsection{Теория}

\begin{enumerate}
  \item Полином с коефициенти над поле.
  \item Степен на полином.
  \item Корени на полиноми.
  \item Теорема за деление с остатък.
  \item Схема на Хорнер.
  \item Всеки идеал в $F[x]$ е главен.
  \item Принцип за сравняване на коефициенти.
  \item Определение на най-голям общ делител на два полинома.
  \item Теорема за съществуване на най-голям общ делител на два полинома с коефициенти над поле.
  \item Изразяване на НОД чрез полиномите (тъждество на Безу).
  \item Алгоритъм на Евклид.
  \item Корени на полиноми.
  \item Формули на Виет.
\end{enumerate}

\subsection{Задачи}

\begin{enumerate}
  \item Намиране на НОД на два полинома - алгоритъм на Евклид, тъждество на Безу.
  \item Прилагане на формулите на Виет за полином с числови коефициенти.
\end{enumerate}

\section{Тема (от \cite{Notes} и \cite{Knapp})}

Нека $F$ е фиксирано поле. За удобство ще означаваме с $0$ и $1$ съответно нулевият и единичният елементи на полето.

\begin{definition}
  \underline{Полином} $p$ на една променлива над $F$ наричаме редица с краен брой ненулеви коефициенти $p = (a_0, a_1, \ldots) \subseteq F$.

  \underline{Степента $\deg p$ на полинома $p$} наричаме най-малкият индекс, след който всички елементи на редицата са $0$. Ако всички елементи на редицата са нули, наричаме полиномът нулев, бележим го с $0$ полагаме $\deg 0 = -\infty$.

  Множеството от всички полиноми на една променлива над $F$ бележим с $F[x]$. Полиномите от степен $\leq 0$ наричаме константи и чрез каноничният изоморфизъм $(a_0, 0, \ldots) \mapsto a_0$ ги отъждествяваме с единственият им ненулев коефициент. Това ни позволява да разглеждаме $F$ е подпръстен на $F[x]$.
\end{definition}

Нека $p = (a_0, a_1, \ldots)$ и $q = (a_0, a_1, \ldots)$ са два полинома. Сума на $p$ и $q$ дефинираме покоординатно, т.е.
\begin{displaymath}
  (p + q) = (a_0 + b_0, a_1 + b_1, \ldots),
\end{displaymath}

а произведението им дефинираме като полинома $pq = (c_0, c_1, \ldots)$, където:
\begin{displaymath}
  c_i = \sum_{k=0}^i p_i q_{k-i}.
\end{displaymath}

Очевидно $\deg(p + q) \leq \max(\deg p, \deg q)$ и $\deg(pq) = \deg p + \deg q$.

За удобство ще означаваме полиномът $p = (a_0, a_1, \ldots)$ чрез
\begin{displaymath}
  p(x) = \sum_{k=0}^n a_k x^k.
\end{displaymath}

Този израз не бива да се отъждествява със съответстващата му по очевиден начин полиномиална функция $p(x): F[x] \to F$, тъй като ако $F$ има характеристика, различна от $0$ (или ако $F$ е произволен пръстен), е възможно различни полиноми да имат една и съща полиномиална функция (например $x^2$ и $x$ във $F_2$).

\begin{definition}
  \underline{Корен на полинома $p(x)$} наричаме всяка стойност $x \in F$, за която $p(x) = 0$.
\end{definition}

\begin{theorem}[за делене с остатък]
  Нека са зададени полиномите $p(x) = \sum_{k=0}^n a_k x^k$ и $q(x) = \sum_{k=0}^m b_k x^k$, където $q(x) \neq 0$. Тогава съществуват единствени полиноми $s$ и $r$ с $\deg r < m$, такива че
  \begin{displaymath}
    p = sq + r.
  \end{displaymath}
\end{theorem}

\begin{proof}
  Първо ще докажем единствеността. Нека
  \begin{displaymath}
    p = sq + r = \hat sq + \hat r.
  \end{displaymath}

  Тогава $0 = p - p = (s - \hat s) q + (r - \hat r)$.

  \begin{displaymath}
    (s - \hat s) q = \hat r - r.
  \end{displaymath}

  Тъй като $q \neq 0$, то $s - \hat s = 0 \iff \hat r - r = 0$. Ако сега допуснем, че $\hat r \neq r$ (и следователно $\hat s \neq s$), получаваме, че $\deg[(s - \hat s) q] = \deg(s - \hat s) + m > m$. Но по условие $\deg(\hat r - r) \leq \max(\deg \hat r, \deg r) < m$. Тъй като степента на полинома в двете страни на равенството трябва да бъде равна, получаваме противоречие от допускането, че $\hat r \neq r$. Следователно $r = \hat r$ и $s = \hat s$.

  Сега ще докажем съществуване. Ако $n < m$, полагаме $s(x) \coloneqq 0$ и $r(x) \coloneqq p(x)$. Нека че $n \geq m$. Ще докажем теоремата с индукция по $n$. Случаят $n = 0$ е тривиален, тъй като тогава полагаме $s(x) \coloneqq \frac {b_0} {a_0}$ и $r(x) \coloneqq 0$. Да предположим, че теоремата е вярна за всички полиноми с $\deg < n$ и да означим $p'(x) = \sum_{k=0}^{n-1} a_k x^k$.

  Индукционното предположение ни дава полиноми $s'(x)$ и $r'(x)$, такива че $p'(x) = s'(x) q(x) + r'(x)$ с $\deg r' < m$. Но ние имаме
  \begin{multline*}
    p(x)
    =
    \sum_{k=0}^n a_k x^k
    =
    \sum_{k=0}^{n-1} a_k x^k + a_n x^n
    =
    p'(x) + a_n x^n
    =
    s'(x) q(x) + r'(x) + a_n x^n
    = \\ =
    \left( s'(x) + \frac {a_n} {b_m} x^{n-m} \right) q(x) + r'(x).
  \end{multline*}

  Полагаме $s(x) \coloneqq s'(x) + \frac {a_n} {b_m} x^{n-m}$ и $r(x) \coloneqq r'(x)$. Очевидно $\deg r = \deg r' < m$. С това съществуването е доказано.
\end{proof}

За да пресметнем стойността на полиномиалната функция $p$ в точката $x$, можем да използваме наивният алгоритъм от дефиницията
\begin{displaymath}
  p(x) = \sum_{k=0}^n a_k x^k
\end{displaymath}

Схемата на Хорнер за пресмятане на стойността на полинома $p$ в точката $x$ се дължи на следното представяна на $p(x)$:
\begin{displaymath}
  p(x) = \sum_{k=0}^n a_k x^k = a_0 + x \sum_{k=1}^n a_k x^{k-1} = \cdots = a_0 + x (a_1 + \cdots + x(a_{n-1} + x a_n) + \cdots),
\end{displaymath}
или, записано чрез спомагателната функция $\hat p$,
\begin{align*}
  \hat p(x, k) &\coloneqq
  \begin{cases}
    a_n, &k = 0, \\
    \hat p(x, k-1) x + a_{n-k}, &k > 0.
  \end{cases} \\
  p(x) = &\hat p(x, n).
\end{align*}

\begin{proposition}
  Пресмятане на $p(x)$ по наивния алгоритъм и по правилото на Хорнер дават един и същ резултат.
\end{proposition}

\begin{proof}
  Ще използваме индукция по $n = \deg p$. За $n = 0$ съвпадението е очевидно. Допускаме, че твърдението е вярно за полиноми от степени $< n$. Нека $p'(x) = a_1, a_2, \ldots$. Тогава
  \begin{displaymath}
    p(x) = \hat p(x, n) = x \hat p(x, n-1) + a_0 = x p'(x) + a_0.
  \end{displaymath}

  Но $p'$ е полином от степен $n-1$ и по индукционно предположение, за него важи твърдението на теоремата, т.е.
  \begin{displaymath}
    p(x) = x \sum_{k=1}^n a_k x^{k-1} + a_0 = \sum_{k=0}^n a_k x^{k-1}.
  \end{displaymath}
\end{proof}

\section{Примерни задачи (от \cite{Knapp})}

\printbibliography

\end{document}
