% arara: pdflatex: { shell: true, interaction: nonstopmode }
% arara: biber
% arara: pdflatex: { shell: true }

\documentclass[numbers=endperiod, DIV=15, bibliography=totocnumbered]{scrartcl}

% Base packages
\usepackage[T2A]{fontenc}
\usepackage[utf8]{inputenc}
\usepackage[bulgarian]{babel}
\usepackage[pdfencoding=unicode]{hyperref}
\usepackage{biblatex}
\usepackage[style=german]{csquotes}

% Base math packages
\usepackage{amsmath}
\usepackage{amssymb}
\usepackage{amsthm}
\usepackage{mathtools}

% Misc packages
\usepackage{polynom} % Polynomial long division
\usepackage{ulem} % Line-breaking underlines

% Custom packages
\usepackage{../../common/macros}
\usepackage{../../common/theorems}

% Bibliography
\addbibresource{./references.bib}

% Document
\title{Тема 3}
\subtitle{Полиноми на една променлива. Теорема за деление с остатък. Най-голям общ делител на полиноми - тъждество на Безу и алгоритъм на Евклид. Зависимост между корени и коефициенти на полиноми (формули на Виет).}
\author{Янис Василев, \Email{ianis@ivasilev.net}}
\date{15 юни 2019}

\begin{document}

\maketitle

\section{Теория}

Някои твърдения и доказателства са взаимствани от~\cite{Knapp} и~\cite{RoyachkiNotes}.

\subsection{Анотация}

Изложената анотацията е взета от конспекта~\cite{Syllabus} за 2018г.

\begin{enumerate}
  \item Полином с коефициенти над поле
  \item Степен на полином
  \item Корени на полиноми
  \item Теорема за деление с остатък
  \item Схема на Хорнер
  \item Всеки идеал в $F[x]$ е главен
  \item Принцип за сравняване на коефициенти
  \item Определение на най-голям общ делител на два полинома
  \item Теорема за съществуване на най-голям общ делител на два полинома с коефициенти над поле
  \item Изразяване на НОД чрез полиномите (тъждество на Безу)
  \item Алгоритъм на Евклид
  \item Формули на Виет
\end{enumerate}

\subsection{Основни понятия}

Нека $F$ е фиксирано поле. За удобство ще означаваме с $0$ и $1$ съответно нулевият и единичният елемент на полето.

\begin{definition}
  \uline{Полином} $p$ на една променлива над $F$ наричаме редица $p = (a_0, a_1, \ldots)$ от елементи на $F$, наречени коефициенти, само краен брой от които са различни от $0$. Ако всички елементи на редицата са нули, наричаме полинома нулев и също както нулевия елемент на полето го бележим с $0$.

  \uline{Степен $\deg(p)$ на полинома $p$} наричаме най-малкия индекс, след който всички елементи на редицата са $0$. Формално,
  \begin{displaymath}
    \deg(p) \coloneqq \min_{k \in \ZNNeg} (\forall m \in \ZPos : a_{k + m} = 0).
  \end{displaymath}

  По конвенция оставяме степента $\deg(0)$ на нулевия полином да бъде неопределена, макар и горната дефиниция да ни дава $\deg(0) = 0$.

  \uline{Старшият коефициент $\LC(p)$ на полинома $p$} наричаме последната ненулева стойност в редицата от коефициенти и полагаме $\LC(p) \coloneqq 0$. Формално,
  \begin{displaymath}
    \LC(p) \coloneqq
    \begin{cases}
      0, &p = 0, \\
      a_{\deg(p)}, &p \neq 0.
    \end{cases}
  \end{displaymath}

  Полинома $p$ наричаме \uline{унитарен}, ако $\LC(p) = 1$.
\end{definition}

Фиксираме променлива $x$, която ще наричаме \uline{свободна}.
Ще пишем $p(x)$ вместо $p$, когато.искаме да поясним, че $x$ е свободната променлива. Множеството от всички полиноми на една променлива над $F$ бележим с $F[x]$.

Нека $p = (a_0, a_1, \ldots)$ и $q = (b_0, b_1, \ldots)$ са два полинома. Сума на $p$ и $q$ дефинираме покоординатно, т.е.
\begin{displaymath}
  (p + q) = (a_0 + b_0, a_1 + b_1, \ldots),
\end{displaymath}

а произведението им дефинираме като полинома $pq = (c_0, c_1, \ldots)$, където:
\begin{displaymath}
  c_k = \sum_{i+j=k} a_i b_j.
\end{displaymath}

Сумата на ненулеви полиноми $p + q$ е полином, при това $p + q$ или е нулевият полином, или $\deg(p + q) \leq \max(\deg(p), \deg(q))$. Произведението на ненулеви полиноми е ненулев полином, при това $\deg(pq) = \deg p + \deg q$.

Относно въведените операции $F[x]$ е комутативен пръстен с единица $1$, тъй като
\begin{enumerate}
  \item $F[x]$ наследява нулата си $0$ и единицата си $1$ от полето $F$.
  \item Събирането на произволни полиноми наследява асоциативността и комутативността си директно от събирането в полето $F$.
  \item Ако $p(x) = \sum_{k=0}^n a_k x^k$, то $-p(x) = \sum_{k=0}^n (-a_k) x^k$ е обратен на $p(x)$ относно събиране.
  \item Произведението на ненулеви полиноми $p = (a_0, a_1, \ldots)$, $q = (b_0, b_1, \ldots)$ и $r = (c_0, c_1, \ldots)$ е асоциативно, тъй като
  \begin{displaymath}
    \sum_{k+m=n} \left(\sum_{i+j=k} a_i b_j \right) c_m
    =
    \sum_{i+j+m=n} a_i b_j c_m
    =
    \sum_{i+l=n} a_i \left( \sum_{j+m=l} b_j c_m \right),
  \end{displaymath}
  където всички индекси са неотрицателни цели числа.
  \item Произведението на ненулеви полиноми наследява комутативността си и дистрибутивността си относно събирането директно от полето $F$.
\end{enumerate}

Това ни позволява да разглеждаме $F$ като подпръстен на $F[x]$ и да разглеждаме $F[x]$ като $F$-алгебра.

Нулевият полином и полиномите от степен $0$ наричаме константи и чрез каноничната проекция $\pi: (a_0, 0, \ldots) \mapsto a_0$ ги отъждествяваме с първия им коефициент. Аналогично, каноничната инекция $\iota: a_0 \mapsto (a_0, 0, \ldots)$ влага $F$ във $F[x]$.

За удобство обикновено записваме ненулевите полиноми $p = (a_0, a_1, \ldots)$ от степен $\deg(p) = n$ като
\begin{displaymath}
  p(x) = \sum_{k=0}^n a_k x^k.
\end{displaymath}

Нека $(F \mapsto F)$ е пръстенът от функции над $F$. Дефинираме хомоморфизма
\begin{align*}
  &\Phi: F[x] \mapsto (F \mapsto F) \\
  &\Phi\left( \sum_{k=0}^n a_k x^k \right) = \left( u \mapsto a_k u^k \right),
\end{align*}
който на всеки полином съпоставя \uline{полиномиална функция}.

Когато полето $F$ е крайно, изображението $\Phi$ не е инективно (например $x$ и $x^2$ съвпадат като функции в $GF_2$) и по тази причина се налага да правим разлика между един полином $p$ и съответната функция $\Phi(p)$. Можем да избегнем този проблем, ако винаги предполагаме, че степените на полиномите са строго по-малки от броя елементи в полето (в.ж. теорема~\ref{thm:coeff-principle}).

\subsection{Делимост на полиноми}

Отсега нататък ще използваме означението $p(x)$ едновременно за полинома и за съответната полиномиална функция, тъй като това няма да предизвиква двусмислици.

\begin{theorem}[Делене с остатък]
  Нека са дадени ненулевите полиноми $p(x) = \sum_{k=0}^n a_k x^k$ и $q(x) = \sum_{k=0}^m b_k x^k$, където $q(x) \neq 0$. Тогава съществуват единствени полиноми $s$ и $r$, където $r = 0$ или $\deg(r) < m$, такива че
  \begin{displaymath}
    p = sq + r.
  \end{displaymath}
\end{theorem}
\begin{proof}
  Първо ще докажем единствеността. Нека
  \begin{displaymath}
    p = sq + r = \hat sq + \hat r.
  \end{displaymath}

  Тогава $0 = p - p = (s - \hat s) q + (r - \hat r)$ и $(s - \hat s) q = \hat r - r$.

  Тъй като $q \neq 0$, то $s - \hat s = 0 \iff \hat r - r = 0$. Ако сега допуснем, че $\hat r \neq r$ (и следователно $\hat s \neq s$), получаваме, че $\deg[(s - \hat s) q] = \deg(s - \hat s) + m > m$. Но по условие $\deg(\hat r - r) \leq \max(\deg \hat r, \deg r) < m$. Тъй като степента на полинома в двете страни на равенството трябва да бъде равна, получаваме противоречие от допускането, че $\hat r \neq r$. Следователно $r = \hat r$ и $s = \hat s$.

  Сега ще докажем съществуване. Ако $n < m$, полагаме $s(x) \coloneqq 0$ и $r(x) \coloneqq p(x)$. Нека $n \geq m$. Ще докажем теоремата с индукция по $n$. Случаят $n = 0$ е тривиален, тъй като тогава полагаме $s(x) \coloneqq \frac {b_0} {a_0}$ и $r(x) \coloneqq 0$. Да предположим, че теоремата е вярна за всички полиноми с $\deg < n$ и да означим $g(x) \coloneqq \frac {a_n} {b_m} x^{n-m} q(x)$.

  Тъй като $\deg(p) = \deg(g)$ и $\LC(p) = \LC(g)$, то $\deg(p - g) < \deg(p) = n$ и индукционното предположение ни дава полиноми $\hat s$ и $\hat r$, такива че $p - g = \hat s q + \hat r$ и $\hat r = 0$ или $\deg(\hat r) < m$. Но ние имаме
  \begin{displaymath}
    p(x)
    =
    g(x) + \hat s(x) q(x) + \hat r(x)
    =
    \left( \frac {a_n} {b_m} x^{n-m} + \hat s(x) \right) q(x) + \hat r(x).
  \end{displaymath}

  Полагаме $s(x) \coloneqq \hat s(x) + \frac {a_n} {b_m} x^{n-m}$ и $r(x) \coloneqq \hat r(x)$. Очевидно $\deg(r) = \deg(\hat r) < m$. С това и съществуването е доказано.
\end{proof}

\begin{definition}
  Казваме, че полиномът $q \in F[x]$ \uline{дели} $p \in F[x]$, ако съществува ненулев $s \in F[x]$, така че $p = sq$, т.е. ако алгоритъмът за делене с остатък дава нулев остатък. Това очевидно е еквивалентно на условието $p$ да принадлежи на идеала $\Gen q \lhd F[x]$.
\end{definition}

\begin{theorem}\label{thm:pid}
  Всеки идеал в $F[x]$ е главен.
\end{theorem}
\begin{proof}
  Нулевият идеал $\Gen 0 \lhd F[x]$ очевидно е главен. Нека $I \lhd F[x]$ е ненулев идеал и нека $q \in I$ е полином от минимална за $I$ степен. Ще докажем, че идеалът $\Gen q \lhd F[x]$, породен от $q$, съвпада с $I$.

  Нека първо $p \in \Gen q$. Тъй като идеалът $\Gen q$ издържа умножение, то съществува полином $s \in F[x]$, за който $p = sq$. Но тъй като $q \in I$, то $p = sq \in I$.

  Нека сега $p \in I$. Теоремата за делене с остатък ни дава полиноми $s$ и $r$ с $r = 0$ или $\deg r < \deg q$, такива че $p = qs + r$. Но понеже $I$ е затворен относно събиране, имаме $r = p - qs \in I$. Ако $r$ е ненулев, то $\deg r < \deg q$, което противоречи на минималността на $q$. Значи $r = 0$ и $p = qs \in \Gen q$.
\end{proof}

\begin{definition}
  \uline{Корен на полинома $p(x)$} наричаме всяка стойност $a \in F$, за която съответната функция се анулира, т.е. $p(a) = 0$.
\end{definition}

\begin{proposition}\label{thm:div-iff-root}
  Полиномът $(x - u)$ дели ненулевия полином $p(x) \in F[x]$ тогава и само тогава, когато $u \in F$ е корен на $p \in F[x]$.
\end{proposition}
\begin{proof}
  ($\implies$) Ако $(x - u)$ дели $p(x)$, то $p(x) \in \Gen{(x - u)}$. Тъй като $u$ е корен на полинома $(x - u)$, той е корен и на всички полиноми от идеала $\Gen{(x - u)}$ и значи $u$ е корен на $p(x)$.

  ($\impliedby$) Нека $u$ е корен на $p(x)$.

  Разглеждаме идеала $I \lhd F[x]$ на всички полиноми, на които $u$ е корен. Това наистина е идеал, тъй като ако $u$ е корен на $q(x)$ и $s(x)$, то
  \begin{enumerate}
    \item $q(u) + s(u) = 0 + 0 = 0$.
    \item $q(u) r(u) = 0 \cdot r(u) = 0~\forall r \in F[x]$.
  \end{enumerate}

  По теорема~\ref{thm:pid} всеки идеал на $F[x]$ е главен, следователно $I = \Gen{(x - u)}$, тъй като $(x - u)$ е ненулев полином от минимална степен, принадлежащ на $I$.

  Тъй като $p(x) \in I = \Gen{(x - u)}$, то $(x - u)$ дели $p(x)$.
\end{proof}

\begin{proposition}\label{thm:gen-div-iff-root}
  За всеки $p(x) \in F[x]$ и $u \in F$ съществува полином $q(x)$ със степен $\deg q < \deg p$, за който $p(x) = (x - u) q(x) + p(u)$.
\end{proposition}
\begin{proof}
  Тъй като $u$ непременно е корен на $p(x) - p(u)$, по твърдение~\ref{thm:div-iff-root} полиномът $(x - u)$ дели $p(x) - p(u)$. Следователно съществува полином $q(x)$ със степен $\deg q < \deg p$, такъв че $p(x) - p(u) = (x - u) q(x)$.
\end{proof}

Схемата (или правилото) на Хорнер за пресмятане на стойността на ненулевия полином $p$ в дадена точка се дължи на следното представяне на $p(x)$:
\begin{displaymath}
  p(x) = \sum_{k=0}^n a_k x^k = a_0 + x \sum_{k=1}^n a_k x^{k-1} = \cdots = a_0 + x (a_1 + \cdots + x(a_{n-1} + x a_n) + \cdots).
\end{displaymath}

Формално правилото се основава на следното наблюдение:

Нека $u \in F$. Искаме да пресметнем $p(u)$. От твърдение~\ref{thm:gen-div-iff-root} знаем, че съществува $q(x)$, така че

\begin{displaymath}
  p(x) = (x - u) q(x) + p(u).
\end{displaymath}

Ако $q(x)$ има представяне $\sum_{k=0}^{n-1} b_k x^k$, то
\begin{align*}
  p(x)
  &=
  (x - u) \sum_{k=0}^{n-1} b_k x^k + p(u),
  \\
  \sum_{k=0}^n a_k x^k
  &=
  \sum_{k=0}^{n-1} b_k x^{k+1} - u \sum_{k=0}^{n-1} b_k x^k + p(u),
  \\
  0
  &=
  (p(u) - u b_0 - a_0) + \sum_{k=1}^{n-1} (b_{k-1} - u b_k - a_k) x^k + (b_{n-1} - a_n) x^n
\end{align*}

Като приравним коефициентите пред съответните едночлени, получаваме следната рекурентна зависимост за коефициентите $b_k, k = 0, \ldots, n - 1$:
\begin{displaymath}
  \begin{cases}
    p(u) &= u b_0 + a_0 \\
    b_{k-1} &= a_k + u b_k, k = 1, \ldots, n - 1 \\
    b_{n-1} &= a_n.
  \end{cases}
\end{displaymath}

Правилото на Хорнер изисква само $n$ умножения и $n$ събирания, докато директното пресмятане на $p(u)$ изисква $\frac {n(n+1)} 2$ умножения и $n$ събирания.

\begin{theorem}[Принцип за сравняване на коефициентите]\label{thm:coeff-principle}
  Нека $p$ и $q$ са полиноми от степен $n$ и нека $u_1, \ldots, u_{n+1} \in F$ са различни скалари (това изисква в полето има поне $n+1$ елемента). Ако е изпълнено $p(u_i) = q(u_i)~\forall i = 1, \ldots, n + 1$, то полиномите $p(x)$ и $q(x)$ съвпадат.
\end{theorem}
\begin{proof}
  Тъй като $a_1$ е корен и на $p(x)$, и на $q(x)$, от твърдение~\ref{thm:div-iff-root} имаме, че $(x - u_1)$ дели и $p(x)$, и $q(x)$. Нека $p_1(x)$ и $q_1(x)$ са частните им с $(x - u_1)$. Имаме $\deg p_1 = \deg q_1 = n - 1$.

  Аналогично получаваме, че $(x - a_2)$ дели $p(x) = (x - u_1) p_1(x)$ и $p(x) = (x - u_1) p_1(x)$. Но тъй като $a_1 \neq a_2$, то $(x - u_2)$ не дели $(x - u_1)$, защото иначе $a_2$ би бил корен на $(x - u_1)$ според твърдение~\ref{thm:div-iff-root}. Следователно $(x - a_2)$ дели $p_1(x)$ и $q_1(x)$ с частни $p_2(x)$ и $q_2(x)$, при това $\deg p_2 = \deg q_2 = n - 2$.

  С индукция по $k < n + 1$ установяваме, че съществуват полиноми $p_k$ и $q_k$ от степен $n - k$, такива че
  \begin{align*}
    p(x) &= (x - u_1) \ldots (x - u_{k-1}) p_k(x) \\
    q(x) &= (x - u_1) \ldots (x - u_{k-1}) q_k(x).
  \end{align*}

  Но $p_n$ и $q_n$ са константни полиноми и следователно ако функциите им са равни за някаква стойност (например $u_{n+1}$), то полиномите съвпадат. Полагаме $v \coloneqq p_n = q_n$ и получаваме
  \begin{displaymath}
    p(x) = v (x - u_1) \ldots (x - u_n) = q(x).
  \end{displaymath}
\end{proof}

\subsection{Най-голям общ делител на полиноми}

\begin{definition}
  Казваме, че един полином $d \in F[x]$ е \uline{най-голям общ делител} (НОД) на $p \in F[x]$ и $q \in F[x]$ и пишем $d = \GCD(p, q)$, ако $d$ дели $p$ и $q$ и ако всеки общ делител на $p$ и $q$ дели $d$. Тъй като всички НОД на $p$ и $q$ се различават по умножение с ненулева константа, ако не е казано иначе, за определеност взимаме $\GCD(p, q)$ да бъде унитарен.

  Казваме, че полиномите $p$ и $q$ са \uline{взаимно прости}, ако $\GCD(p, q) = 1$.

  Оставяме НОД на два нулеви полинома да бъде неопределен.
\end{definition}

\begin{theorem}
  За всеки два полинома $p, q \in F[x]$ съществува единствен с точност до умножение с ненулева константа $\GCD(p, q)$.
\end{theorem}
\begin{proof}
  От теорема~\ref{thm:pid} следва, че идеалът $I = \Gen p + \Gen q \lhd F[x]$ е главен, т.е. съществува унитарен полином $d \in I$, който го поражда.
  Тогава $d$ е общ делител на $p$ и $q$.

  Но $d \in I$, следователно съществуват полиноми $u, v \in F[x]$, такива че $u p + v q = d$.

  Тогава за всеки общ делител $g$ на $p$ и $q$ имаме $p, q \in \Gen g$ и следователно $d = u p + v q \in \Gen g$, т.е. $g$ дели $d$. Ако $\deg g = \deg d$, то те се различават с ненулева константа.

  Тогава $d$ е най-голям общ делител на $p$ и $q$.
\end{proof}

Като част от горното доказателство ние доказахме и следната
\begin{theorem}[Тъждество на Безу]
  За всеки два полинома $p, q \in F[x]$ съществуват полиноми $u$ и $v$, такива че $u p + v q = \GCD(p, q)$.
\end{theorem}

Ако $p$ е нулев и $q$ е ненулев, имаме $\GCD(p, q) = p$ (и обратно). За ненулеви полиноми имаме явен алгоритъм за намиране на НОД.
\begin{theorem}[Алгоритъм на Евклид]
  Нека $p$ и $q$ са произволни ненулеви полиноми.

  Полагаме
  \begin{align*}
     f_{-1} \coloneqq p &&
     f_0 \coloneqq q.
  \end{align*}

  Алгоритъм на Евклид ($k$-та стъпка): Деленето с остатък ни дава полиноми $s$ и $r$ такива, че $f_{k-2} = p_{k-1} s + r$.
    \begin{enumerate}
      \item Ако $r = 0$, алгоритъмът приключва.
      \item Ако $r \neq 0$ и $\deg r < \deg f_{k-1}$, полагаме $f_k \coloneqq r$ и алгоритъмът преминава към стъпка $k + 1$.
    \end{enumerate}

  Твърдим, че така построената редица е крайна с дължина $m$ и освен това $\GCD(p, q) = f_m$.
\end{theorem}
\begin{proof}
  Тъй като построяваме редица със строго намаляващи степени (с евентуално изключение $\deg f_{-1} < \deg f_0$), тази редица непременно е крайна. Нека $m$ е дължината ѝ.

  С индукция по $i = 2, \ldots, m + 1$ ще докажем, че $f_m$ дели $f_{m-i}$. Разглеждаме базовия случай $i = 2$:
  \begin{enumerate}
    \item $f_{m-1} = f_m s$ за някой полином $s$ и значи $f_m$ дели $f_{m-1}$.
    \item $f_{m-2} = f_{m-1} t + f_m = f_m (s t + 1)$ за някой полином $t$ и значи $f_m$ дели и $f_{m-2}$.
  \end{enumerate}

  Сега допускаме, че $f_m$ дели $f_{m-j}$ за $j < i$. Но $f_{m-i} = f_{m-(i-1)} s + f_{m-(i-2)}$ за някой полином $s$ и по индукционно предположение $f_m$ дели $f_{m-(i-1)}$ и $f_{m-(i-2)}$, следователно $f_m$ дели и $f_{m-i}$.

  В частност, доказахме, че, $f_m$ дели $p$ и $q$.

  Нека сега $g$ е произволен общ делител на $p$ и $q$, т.е. съществуват полиноми $h_1$ и $h_2$, така че $p = g h_1$ и $q = g h_2$. Тогава за някой полином $s$ е изпълнено
  \begin{displaymath}
    f_1 = p - qs = g h_1 - g h_2 s = g (h_1 - h_2 s),
  \end{displaymath}
  следователно $g$ дели $f_1$. Със същото разсъждение и с индукция по $i = 1, \ldots, m$ стигаме до извода, че $g$ дели $f_i$, в частност $g$ дели $f_m$. Следователно $\GCD(p, q) = f_m$.
\end{proof}

\subsection{Формули на Виет}

\begin{theorem}[Формули на Виет]
  Нека е даден унитарен неконстантен полином $p(x) = \sum_{k=0}^n a_k x^k \in F[x]$ и нека всичките му корени $u_1, \ldots, u_n$ (с евентуални повторения) са от $F$.

  Тогава $a_n = 1$ и за $k = 0, \ldots, n-1$ имаме следната връзка между коефициентите и корените на полинома $p$:
  \begin{displaymath}
    a_{n-k} = {(-1)}^k \sum_{1 \leq i_1 < \cdots < i_k \leq n} u_{i_1} \ldots u_{i_k}.
  \end{displaymath}
\end{theorem}
\begin{proof}
  След като всички корени на $p$ са във $F$, то $p$ се разлага на линейни множители над $F[x]$, т.е.
  \begin{displaymath}
    p(x) = (x - u_1) \cdots (x - u_n).
  \end{displaymath}

  Ще докажем теоремата с индукция по $n = \deg p$. Базовият случай $n = 1$ е тривиален, тъй като тогава $p(x) = (x - u_1)$ и $a_0 = {(-1)}^1 u_1 = -u_1$.

  Нека теоремата е вярна за всички полиноми от степен $n$ и $p = (x - u_1) \cdots (x - u_{n+1})$. Полагаме
  \begin{displaymath}
    q(x) \coloneqq (x - u_1) \cdots (x - u_n).
  \end{displaymath}
  Нека коефициентите на $q$ са $q = (b_0, \ldots, b_n)$.

  Индукционното предположение е изпълнено за $q(x)$ и освен това имаме връзката
  \begin{multline*}
    (x - u_{n+1}) q(x)
    =
    (x - u_{n+1}) \sum_{k=0}^n b_k x^k
    =
    \sum_{k=1}^{n+1} b_{k-1} x^k + \sum_{k=0}^n (-u_{n+1}) b_k x^k
    = \\ =
    (-u_{n+1}) b_0 + \sum_{k=1}^n (b_{k-1} - u_{n+1} b_k) x^k + b_n x^{n+1}
    =
    \sum_{k=0}^{n+1} a_k x^k
    =
    p(x).
  \end{multline*}

  Като приравним коефициентите пред съответните едночлени, получаваме
  \begin{align*}
    a_0
    &=
    (-u_{n+1}) b_0
    =
    (-u_{n+1}) {(-1)}^n \sum_{1 \leq i_1 < \cdots < i_n \leq n} u_{i_1} \ldots u_n
    =
    {(-1)}^{n+1} u_1 \ldots u_{n+1}
    = \\ &=
    {(-1)}^n \sum_{1 \leq i_1 < \cdots < i_n < i_{n+1} \leq n + 1} u_{i_1} \ldots u_{n+1},
    \\ \\
    a_{n+1-k}
    &=
    b_{n-k} - u_{n+1} b_{n+1-k}
    = \\ &=
    {(-1)}^k \sum_{1 \leq i_1 < \cdots < i_k \leq n} u_{i_1} \ldots u_{i_k} - u_{n+1} {(-1)}^{k-1} \sum_{1 \leq i_1 < \cdots < i_{k-1} \leq n} u_{i_1} \ldots u_{i_{k-1}}
    = \\ &=
    {(-1)}^k \left( \sum_{1 \leq i_1 < \cdots < i_k \leq n} u_{i_1} \ldots u_{i_k} + u_{n+1} \sum_{1 \leq i_1 < \cdots < i_{k-1} \leq n} u_{i_1} \ldots u_{i_{k-1}} \right)
    = \\ &=
    {(-1)}^k \sum_{1 \leq i_1 < \cdots < i_{k-1} \leq n} u_{i_1} \ldots u_{i_{k-1}} \left(\sum_{i_k=i_{k-1}+1}^n u_{i_k} + u_{n+1} \right)
    = \\ &=
    {(-1)}^k \sum_{1 \leq i_1 < \cdots < i_k \leq n + 1} u_{i_1} \ldots u_{i_k},
    \\ \\
    a_{n+1} &= b_n = 1.
  \end{align*}
\end{proof}

\section{Примерни задачи}

Условията на представените задачи са взети от~\cite{PolynomialExercises}.

\subsection{Анотация}

\begin{enumerate}
  \item Намиране на НОД на два полинома - алгоритъм на Евклид, тъждество на Безу
  \item Прилагане на формулите на Виет за полином с числови коефициенти
\end{enumerate}

\subsection{Най-голям общ делител на полиноми}

\begin{exercise}
  \mbox{}
  \begin{enumerate}
    \item Да се намери най-големият общ делител $d(x)$ на полиномите
    \begin{align*}
      f(x) &\coloneqq x^3 + x^2 + x + 1, \\
      g(x) &\coloneqq x^2 - x + 2.
    \end{align*}

    \item Да се намерят полиноми $u(x)$ и $v(x)$, за които е изпълнено тъждеството на Безу
    \begin{displaymath}
      d(x) = f(x) u(x) + g(x) v(x).
    \end{displaymath}
  \end{enumerate}
\end{exercise}
\begin{solution}
  \mbox{}
  \begin{enumerate}
    \item Делим $f(x)$ на $g(x)$:
    \begin{displaymath}
      \polylongdiv {x^3 + x^2 + x + 1} {x^2 - x + 2},
    \end{displaymath}

    Делим $g(x)$ на $f_1(x) \coloneqq x - 3$:
    \begin{displaymath}
      \polylongdiv {x^2 - x + 2} {x - 3},
    \end{displaymath}

    Полиномът $f_2(x) \coloneqq 8$ дели $f_1(x)$, следователно $d(x) = f_2(x) = \GCD(f, g) = 8$ и $f(x)$ и $g(x)$ са взаимно прости.

    \item Изразяваме остатъците от деленето при алгоритъма на Евклид:
    \begin{align*}
      f_1(x)
      &=
      f(x) - (x + 2) g(x),
      \\
      d(x)
      &=
      g(x) - (x + 2) f_1(x)
      =
      g(x) - (x + 2) [f(x) - (x + 2) g(x)]
      = \\ &=
      (x + 2) f(x) + [{(x + 2)}^2 + 1] g(x)
      =
      \boxed{(x + 2) f(x) + (x^2 + 4x + 5) g(x)}.
    \end{align*}
  \end{enumerate}
\end{solution}

\subsection{Формули на Виет}

\begin{exercise}
  За кои стойности на параметъра $p \in \R$ корените $x_1, \ldots, x_4$ на полинома
  \begin{displaymath}
    f(x) = x^4 - 8x^3 + 22x^2 + px + 16
  \end{displaymath}
  изпълняват равенството $x_1 + x_2 + x_3 = x_4$?
\end{exercise}
\begin{solution}
  Заместваме $x_4 = x_1 + x_2 + x_3$ във формулите на Виет:
  \begin{align*}
    8 &= (x_1 + x_2 + x_3) + x_4 = 2x_4 = 8
    \\&\implies
    x_4 = 4,
    \\ \\
    22 &= x_1 x_2 + x_1 x_3 + x_1 x_4 + x_2 x_3 + x_2 x_4 + x_3 x_4 = x_1 x_2 + x_1 x_3 + x_2 x_3 + (x_1 + x_2 + x_3) x_4
    \\&\implies
    x_1 x_2 + x_1 x_3 + x_2 x_3 = 6,
    \\ \\
    -p &= x_1 x_2 x_3 + x_1 x_2 x_4 + x_1 x_3 x_4 + x_2 x_3 x_4 = x_1 x_2 x_3 + (x_1 x_2 + x_1 x_3 + x_2 x_3) x_4 \\&\implies
    x_1 x_2 x_3 = -p - 24,
    \\ \\
    16 &= (x_1 x_2 x_3) x_4
    \\&\implies
    (-p - 24) 4 = 16 \implies p = -28.
  \end{align*}
\end{solution}

\printbibliography

\end{document}
