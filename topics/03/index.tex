% arara: pdflatex: { shell: true, interaction: nonstopmode }
% arara: biber
% arara: pdflatex: { shell: true }

\documentclass[numbers=endperiod, DIV=15]{scrartcl}

% Base packages
\usepackage[T2A]{fontenc}
\usepackage[utf8]{inputenc}
\usepackage[bulgarian]{babel}
\usepackage[pdfencoding=unicode]{hyperref}
\usepackage{biblatex}
\usepackage[style=german]{csquotes}

% Base math packages
\usepackage{amsmath}
\usepackage{amssymb}
\usepackage{amsthm}
\usepackage{mathtools}

% Custom packages
\usepackage{../../common/macros}
\usepackage{../../common/theorems}

% Bibliography
\addbibresource{./references.bib}

% Document
\title{Тема 3}
\subtitle{Полиноми на една променлива. Теорема за деление с остатък. Най-голям общ делител на полиноми - тъждество на Безу и алгоритъм на Евклид. Зависимост между корени и коефициенти на полиноми (формули на Виет).}
\author{Янис Василев, \Email{ianis@ivasilev.net}}
\date{6 юни 2019}

\begin{document}

\maketitle

\section{Анотация (от \cite{Syllabus})}

\subsection{Теория}

\begin{enumerate}
  \item Полином с коефициенти над поле.
  \item Степен на полином.
  \item Корени на полиноми.
  \item Теорема за деление с остатък.
  \item Схема на Хорнер.
  \item Всеки идеал в $F[x]$ е главен.
  \item Принцип за сравняване на коефициенти.
  \item Определение на най-голям общ делител на два полинома.
  \item Теорема за съществуване на най-голям общ делител на два полинома с коефициенти над поле.
  \item Изразяване на НОД чрез полиномите (тъждество на Безу).
  \item Алгоритъм на Евклид.
  \item Корени на полиноми.
  \item Формули на Виет.
\end{enumerate}

\subsection{Задачи}

\begin{enumerate}
  \item Намиране на НОД на два полинома - алгоритъм на Евклид, тъждество на Безу.
  \item Прилагане на формулите на Виет за полином с числови коефициенти.
\end{enumerate}

\section{Тема (от~\cite{Notes} и~\cite{Knapp})}

Нека $F$ е фиксирано поле. За удобство ще означаваме с $0$ и $1$ съответно нулевият и единичният елемент на полето.

\begin{definition}
  \underline{Полином} $p$ на една променлива над $F$ наричаме редица с краен брой ненулеви коефициенти $p = (a_0, a_1, \ldots) \subseteq F$. Ако всички елементи на редицата са нули, наричаме полинома нулев и също както нулевия елемент на полето го бележим с $0$.

  \underline{Степен $\deg(p)$ на полинома $p$} наричаме най-малкия индекс, след който всички елементи на редицата са $0$. Формално,
  \begin{displaymath}
    \deg(p) \coloneqq \min_{k \in \ZNNeg} (\forall m \in \ZPos : a_{k + m} = 0).
  \end{displaymath}

  По конвенция оставяме степента $\deg(0)$ на нулевия полином да бъде неопределена, макар и по горната дефиниция да имаме $\deg(0) = 0$.

  \underline{Старшият коефициент $\LC(p)$ на полинома $p$} наричаме последната ненулева стойност в редицата от коефициенти и полагаме $\LC(p) \coloneqq 0$. Формално,
  \begin{displaymath}
    \LC(p) \coloneqq
    \begin{cases}
      0, &p = 0, \\
      a_{\deg(p)}, &p \neq 0.
    \end{cases}
  \end{displaymath}

  Полинома $p$ ще наричаме \underline{унитарен}, ако $\LC(p) = 1$.
\end{definition}

Множеството от всички полиноми на една променлива над $F$ бележим с $F[x]$. Нулевият полином и полиномите от степен $0$ наричаме константи и чрез каноничната проекция $\pi: (a_0, 0, \ldots) \mapsto a_0$ ги отъждествяваме с първия им коефициент.

За удобство ще означаваме ненулевите полиноми $p = (a_0, a_1, \ldots)$ чрез
\begin{displaymath}
  p(x) = \sum_{k=0}^{\deg(p)} a_k x^k.
\end{displaymath}

На всеки елемент $u \in F$ съпоставяме стойността $p(u) = \sum_{k=0}^n a_k u^k$, т.е. разглеждаме полинома $p \in F[x]$ като функция $p(x): F \to F$. В общия случай, когато полето $F$ има ненулева характеристика (или когато $F$ не е поле), е възможно различни полиноми да имат една и съща полиномиална функция, например $x^2$ и $x$ във $F_2$. Въпреки това, за нас ще бъде удобно да отъждествяваме полиномите със съответстващите им функции.

Нека $p = (a_0, a_1, \ldots)$ и $q = (a_0, a_1, \ldots)$ са два полинома. Сума на $p$ и $q$ дефинираме покоординатно, т.е.
\begin{displaymath}
  (p + q) = (a_0 + b_0, a_1 + b_1, \ldots),
\end{displaymath}

а произведението им дефинираме като полинома $pq = (c_0, c_1, \ldots)$, където:
\begin{displaymath}
  c_k = \sum_{i+j=k} a_i b_j.
\end{displaymath}

Сумата на ненулеви полиноми $p + q$ е полином, при това $p + q$ или е нулевият полином, или $\deg(p + q) \leq \max(\deg(p), \deg(q))$. Произведението на ненулеви полиноми е ненулев полином, при това $\deg(pq) = \deg p + \deg q$.

Относно въведените операции $F[x]$ е комутативен пръстен с единица $1$, тъй като
\begin{enumerate}
  \item $F[x]$ наследява нулата си $0$ и единицата си $1$ от полето $F$.
  \item Събирането на произволни полиноми наследява асоциативността и комутативността си директно от събирането в полето $F$.
  \item Ако $p(x) = \sum_{k=0}^n a_k x^k$, то $-p(x) = \sum_{k=0}^n (-a_k) x^k$ е обратен на $p(x)$ относно събиране.
  \item Произведението на ненулеви полиноми $p = (a_0, a_1, \ldots)$, $q = (b_0, b_1, \ldots)$ и $r = (c_0, c_1, \ldots)$ е асоциативно, тъй като
  \begin{displaymath}
    \sum_{k+m=n} \left(\sum_{i+j=k} a_i b_j \right) c_m
    =
    \sum_{i+j+m=n} a_i b_j c_m
    =
    \sum_{i+l=n} a_i \left( \sum_{j+m=l} b_j c_m \right),
  \end{displaymath}
  където всички индекси са неотрицателни цели числа.
  \item Произведението на ненулеви полиноми наследява дистрибутивността си относно събирането директно от полето $F$.
\end{enumerate}

В частност, понеже $\deg(pq) = \deg p + \deg q$, то $F[x]$ е област. Това ни позволява да разглеждаме $F$ като подпръстен на $F[x]$ и да разглеждаме $F[x]$ като $F$-алгебра.

\begin{definition}
  \underline{Корен на полинома $p(x)$} наричаме всяка стойност $x \in F$, за която съответната функция се анулира, т.е. $p(x) = 0$.
\end{definition}

\begin{theorem}[за делене с остатък]
  Нека са зададени ненулевите полиноми $p(x) = \sum_{k=0}^n a_k x^k$ и $q(x) = \sum_{k=0}^m b_k x^k$, където $q(x) \neq 0$. Тогава съществуват единствени полиноми $s$ и $r$, където $r = 0$ или $\deg(r) < m$, такива че
  \begin{displaymath}
    p = sq + r.
  \end{displaymath}
\end{theorem}
\begin{proof}
  Първо ще докажем единствеността. Нека
  \begin{displaymath}
    p = sq + r = \hat sq + \hat r.
  \end{displaymath}

  Тогава $0 = p - p = (s - \hat s) q + (r - \hat r)$ и $(s - \hat s) q = \hat r - r$.

  Тъй като $q \neq 0$, то $s - \hat s = 0 \iff \hat r - r = 0$. Ако сега допуснем, че $\hat r \neq r$ (и следователно $\hat s \neq s$), получаваме, че $\deg[(s - \hat s) q] = \deg(s - \hat s) + m > m$. Но по условие $\deg(\hat r - r) \leq \max(\deg \hat r, \deg r) < m$. Тъй като степента на полинома в двете страни на равенството трябва да бъде равна, получаваме противоречие от допускането, че $\hat r \neq r$. Следователно $r = \hat r$ и $s = \hat s$.

  Сега ще докажем съществуване. Ако $n < m$, полагаме $s(x) \coloneqq 0$ и $r(x) \coloneqq p(x)$. Нека че $n \geq m$. Ще докажем теоремата с индукция по $n$. Случаят $n = 0$ е тривиален, тъй като тогава полагаме $s(x) \coloneqq \frac {b_0} {a_0}$ и $r(x) \coloneqq 0$. Да предположим, че теоремата е вярна за всички полиноми с $\deg < n$ и да означим $g(x) \coloneqq \frac {a_n} {b_n} x^{n-m} q(x)$.

  Тъй като $\deg(p) = \deg(g)$ и $\LC(p) = \LC(g)$, то $\deg(p - g) < \deg(p) = n$ и индукционното предположение ни дава полиноми $s'$ и $r'$, такива че $p - g = s' q + r'$ и $r' = 0$ или $\deg(r') < m$. Но ние имаме
  \begin{displaymath}
    p(x)
    =
    g(x) + s'(x) q(x) + r'(x)
    =
    \left( \frac {a_n} {b_n} x^{n-m} + s'(x) \right) q(x) + r'(x).
  \end{displaymath}

  Полагаме $s(x) \coloneqq s'(x) + \frac {a_n} {b_m} x^{n-m}$ и $r(x) \coloneqq r'(x)$. Очевидно $\deg(r) = \deg(r') < m$. С това и съществуването е доказано.
\end{proof}

Схемата на Хорнер за пресмятане на стойността на полинома $p$ в точката $x$ се дължи на следното представяне на $p(x)$:
\begin{displaymath}
  p(x) = \sum_{k=0}^n a_k x^k = a_0 + x \sum_{k=1}^n a_k x^{k-1} = \cdots = a_0 + x (a_1 + \cdots + x(a_{n-1} + x a_n) + \cdots),
\end{displaymath}
или, записано чрез спомагателната функция $\hat p$,
\begin{align*}
  \hat p(x, k) &\coloneqq
  \begin{cases}
    a_n, &k = 0, \\
    \hat p(x, k-1) x + a_{n-k}, &k > 0.
  \end{cases} \\
  p(x) = &\hat p(x, n).
\end{align*}

\begin{proposition}
  Директното пресмятане на $p(a)$ и пресмятането правилото на Хорнер дават един и същ резултат.
\end{proposition}
\begin{proof}
  Ще използваме индукция по $n = \deg p$. За $n = 0$ съвпадението е очевидно. Допускаме, че твърдението е вярно за полиноми от степени $< n$. Нека $p'(x) = \sum_{k=0}^n a_k x^k$. Тогава
  \begin{displaymath}
    p(x) = \hat p(x, n) = x \hat p(x, n-1) + a_0 = x p'(x) + a_0.
  \end{displaymath}

  Но $p'$ е полином от степен $n-1$ и, по индукционно предположение, за него важи твърдението на теоремата, т.е.
  \begin{displaymath}
    p(x) = x \sum_{k=1}^n a_k x^{k-1} + a_0 = \sum_{k=0}^n a_k x^{k-1}.
  \end{displaymath}
\end{proof}

Правилото на Хорнер изисква само $n$ умножения и $n$ събирания, докато директното пресмятане изисква $\frac {n(n+1)} 2$ умножения и $n$ събирания.

\begin{theorem}
  Всеки идеал в $F[x]$ е главен.
\end{theorem}
\begin{proof}
  Нулевият идеал $\Gen 0 \lhd F[x]$ очевидно е главен. Нека $I \lhd F[x]$ е ненулев идеал и нека $q \in I$ е минимален полином за $I$, т.е. унитарен полином от най-ниската възможна степен. Ще докажем, че идеалът $\Gen q \lhd F[x]$, породен от $q$, съвпада с $I$.

  Нека първо $p \in \Gen q$. Тъй като идеалът $\Gen q$ издържа умножение, то съществува полином $s \in F[x]$, за който $p = sq$. Но тъй като $q \in I$, то $p = sq \in I$.

  Нека сега $p \in I$. Теоремата за делене с остатък ни дава полиноми $s$ и $r$ с $r = 0$ или $\deg r < \deg q$, такива че $p = qs + r$. Но понеже $I$ е затворен относно събиране, имаме $r = p - qs \in I$. Ако $r$ е ненулев, то $\deg r < \deg q$, което противоречи на минималността на $q$. Значи $r = 0$ и $p = qs \in \Gen q$.
\end{proof}

% \section{Примерни задачи (от~\cite{Knapp})}

\printbibliography

\end{document}
