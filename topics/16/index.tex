% arara: pdflatex: { shell: true, interaction: nonstopmode }
% arara: biber
% arara: pdflatex: { shell: true }

\documentclass[numbers=endperiod, DIV=15, bibliography=totocnumbered]{scrartcl}

% Base packages
\usepackage[T2A]{fontenc}
\usepackage[utf8]{inputenc}
\usepackage[bulgarian]{babel}
\usepackage[pdfencoding=unicode]{hyperref}
\usepackage{biblatex}
\usepackage[style=german]{csquotes}

% Base math packages
\usepackage{amsmath}
\usepackage{amssymb}
\usepackage{amsthm}
\usepackage{mathtools}

% Misc packages
\usepackage{enumitem} % Customization of enum counters
\usepackage{ulem} % Line-breaking underlines

% Custom packages
\usepackage{../../common/macros}
\usepackage{../../common/theorems}

% Bibliography
\addbibresource{./references.bib}

% Document
\title{Тема 16}
\subtitle{Случайни величини с непрекъснати разпределения. Нормално разпределение. Равномерно разпределение, експоненциално разпределение или гама разпределение. Задачи, в които възникват.}
\author{Янис Василев, \Email{ianis@ivasilev.net}}
\date{21 юни 2019}

\begin{document}

\maketitle

\section{Анотация}

Изложената анотацията е взета от конспекта~\cite{Syllabus} за 2018г.

\subsection{Теория}

\begin{enumerate}
  \item Дефиниция на непрекъснато разпределение на случайна величина
  \item Вероятностна плътност и свойствата ѝ - неотрицателност и нормираност
  \item Дефиниция на моментите на непрекъсната случайна величина
  \item Дефиниция и свойства (без доказателства) на пораждаща моментите/характеристична функция (по избор)
  \item Дефиниция, коректност, мотивиращ пример, пораждаща моментите/характеристична функция, очакване и дисперсия на нормално разпределение и още едно избрано от комисията непрекъснато разпределение
\end{enumerate}

\subsection{Задачи}

Не е даден списък с възможни задачи, затова съм включил разни задачи, давани на държавен изпит.

\section{Теория}

Теорията е представена с минимални препратки към теорията на мярката и е базирана частично на изложението в~\cite{Borovkov} и~\cite{DimitrovYanev}. За пълнота съм включил доказателства на основните свойства на пораждащи моментите и характеристична функции.

\subsection{Основни дефиниции и теореми}

\begin{definition}
  \uline{(Реална) случайна величина} над вероятностното пространство $(\Omega, \F, \Prob)$ наричаме всяка измерима функция $\xi : \Omega \to \R$.

  Условието за измеримост на $\xi$ може да се запише така: за всяко борелово множество $B \in \BorelAlgebra(\R)$ имаме
  \begin{displaymath}
    \xi^{-1} (B) = \{ \omega \in \Omega \mid \xi(\omega) \in B \} \in \F.
  \end{displaymath}

  \uline{Разпределение на $\xi$} наричаме мярката
  \begin{displaymath}
    \Prob_\xi(A) \coloneqq \Prob(\xi \in A).
  \end{displaymath}

  Две случайни величини $\xi$ и $\eta$ наричаме \uline{независими}, ако за всички $A, B \in \F$ е изпълнено
  \begin{displaymath}
    \Prob(\xi \in A, \eta \in B) = \Prob_\xi(A) \Prob_\eta(B).
  \end{displaymath}

  \uline{Функция на разпределение} на случайната величина $\xi$ наричаме
  \begin{displaymath}
    F_\xi(x) \coloneqq \Prob(\xi \leq x).
  \end{displaymath}

  Случайната величина $\xi$ наричаме \uline{(абсолютно) непрекъсната} и казваме, че $\xi$ има \uline{(абсолютно) непрекъснато разпределение}, ако функцията ѝ на разпределение е локално абсолютно непрекъсната в $\R$, т.е. абсолютно непрекъсната във всеки затворен интервал. Известно е, че абсолютно непрекъснатите в затворен интервал $[a, b]$ функции са точно тези, които са диференцируеми почти навсякъде в интервала, производните им в $[a, b]$ са интегруеми по Риман и за $x \in [a, b]$ е изпълнено
  \begin{displaymath}
    F_\xi(x) = \int_a^x F'_\xi(x) + F_\xi(a).
  \end{displaymath}

  Функцията $f_\xi: \R \mapsto \R$ наричаме \underline{вероятностна плътност} на случайната величина $\xi$, ако $F'_\xi(x) = f_\xi(x)$ във всяка точка, в която $F_\xi$ е диференцируема. Ако за едно разпределение съществуват множество плътности, те се различават само по изброимо много точки и тъй като $f_\xi(x)$ се използва основно за интегриране, на практика няма значение с коя от плътностите ще работим.
\end{definition}

\begin{proposition}[Основни свойства на функцията на разпределение]\label{thm:cdf-props}
  Функцията $F_\xi$ е функция на разпределение на някаква (не непременно абсолютно непрекъсната) случайна величина $\xi$ тогава и само тогава, когато са изпълнени
  \begin{enumerate}
    \item $F_\xi(x) \leq F_\xi(y), x < y$ (монотонност)
    \item $F_\xi(x)$ е непрекъсната отдясно
    \item $\lim_{x \downarrow -\infty} F_\xi(x) = 0$
    \item $\lim_{x \uparrow \infty} F_\xi(x) = 1$
  \end{enumerate}
\end{proposition}

Твърдение~\ref{thm:cdf-props} ни дава обосновка да задаваме случайни величини изцяло чрез функцията им на разпределение, т.е. без изрично да задаваме вероятностни пространства.

\begin{proof}[Доказателство на твърдение~\ref{thm:cdf-props}]
  ($\implies$)
  \begin{enumerate}
    \item За всички $x < y$
    \begin{multline*}
      F_\xi(x)
      =
      \Prob(\xi \leq x)
      =
      \Prob(\{ \omega \in \Omega \mid \xi(\omega) \leq x \})
      =
      \Prob(\{ \omega \in \Omega \mid \xi(\omega) \leq x \} \cap \{ \omega \in \Omega \mid \xi(\omega) \leq y \})
      \leq \\ \leq
      \Prob(\{ \omega \in \Omega \mid \xi(\omega) \leq y \})
      =
      \Prob(\xi \leq y)
      =
      F_\xi(y).
    \end{multline*}

    \item От монотонността на вероятностната мярка имаме
    \begin{multline*}
      \lim_{h \downarrow 0} F_\xi(x + h)
      =
      \lim_{h \downarrow 0} \Prob(\xi \leq x + h)
      =
      \lim_{h \downarrow 0} \Prob(\{ \omega \in \Omega \mid \xi(\omega) \leq x + h \})
      =
      \Prob(\cup_{h \geq 0} \{ \omega \in \Omega \mid \xi(\omega) \leq x + h \})
      = \\ =
      \Prob(\{ \omega \in \Omega \mid \xi(\omega) \leq x \})
      =
      \Prob(\xi \leq x)
      =
      F_\xi(x).
    \end{multline*}

    \item От монотонността на вероятностната мярка имаме
    \begin{multline*}
      \lim_{x \uparrow \infty} F_\xi(x)
      =
      \lim_{x \uparrow \infty} \Prob(\xi \leq x)
      =
      \lim_{x \uparrow \infty} \Prob(\{ \omega \in \Omega \mid \xi(\omega) \leq x \})
      =
      \Prob(\cup_{x \uparrow \infty} \{ \omega \in \Omega \mid \xi(\omega) \leq x \})
      = \\ =
      \Prob(\cup_{x \uparrow \infty} \{ \omega \in \Omega \mid \xi(\omega) \leq 1 \})
      =
      \Prob(\{ \omega \in \Omega \mid \xi(\omega) \leq \infty \})
      =
      \Prob(\Omega)
      =
      1.
    \end{multline*}

    \item $\lim_{x \uparrow \infty} F_\xi(x) = 0$ се доказва напълно аналогично на $\lim_{x \uparrow \infty} F_\xi(x) = 1$.
  \end{enumerate}

  ($\impliedby$) Нека функцията $F_\xi$ удовлетворява условията на теоремата. Дефинираме
  \begin{align*}
    \xi: \R \to \R,     &&& \Prob: \BorelAlgebra(\R) \to [0, 1], \\
    \xi(x) \coloneqq x, &&& \Prob((a, b]) \coloneqq F_\xi \left(\lim_{h \downarrow 0} b + h \right) - F_\xi(a), a < b \in \R.
  \end{align*}

  Интервалите от вида $(a, b]$ пораждат бореловата $\sigma$-алгебра $\BorelAlgebra(\R)$. Ще пропуснем доказателството на това, че $\Prob$ е вероятностна мярка над $(\R, \BorelAlgebra(\R))$.

  Тогава $\xi$ е измерима функция над $(\R, \BorelAlgebra(\R), \Prob)$ и освен това
  \begin{displaymath}
    \Prob(\xi \leq x)
    =
    \Prob((-\infty, x])
    =
    F_\xi(x)~\forall x \in \R.
  \end{displaymath}

  С други думи, построихме вероятностно пространство и случайна величина $\xi$, чиято функцията на разпределение е $F_\xi$.
\end{proof}

\begin{theorem}\label{thm:density-props}
  Интегруемата по Риман функция $f_\xi: \R \to \R$ е плътност на някаква абсолютно непрекъсната случайна величина $\xi$ тогава и само тогава, когато са изпълнени
  \begin{enumerate}
    \item $f_\xi(x) \geq 0$ почти навсякъде (неотрицателност)
    \item $\int_\R f_\xi(x) dx = 1$ (нормираност)
  \end{enumerate}
\end{theorem}

Тъй като от теорема~\ref{thm:density-props} плътността на една непрекъсната случайна величина определя функцията ѝ на разпределение, понякога разпределение на непрекъсната случайна величина се нарича плътността ѝ.

\begin{proof}[Доказателство на теорема~\ref{thm:density-props}]
  ($\implies$) Нека $f_\xi$ е плътност на $\xi$.
  \begin{enumerate}
    \item За всяка точка $x \in \R$, в която $F_\xi$ е диференцируема, имаме $f_\xi(x) = F'_\xi(x) = \lim_{h \downarrow 0} \frac {F_\xi(x + h) - F_\xi(x)} h \geq 0$ поради монотонността на $F_\xi$
    \item За произволно $c > 0$ функцията $F_\xi$ е абсолютно непрекъсната в $[-c, c]$. Следователно
    \begin{displaymath}
      \int_\R f_\xi(x) dx
      =
      \lim_{c \uparrow \infty} \int_{-c}^c f_\xi(x) dx
      =
      \lim_{c \uparrow \infty} F_\xi(c) - \lim_{c \to \infty} F_\xi(-c)
      =
      1 - 0.
    \end{displaymath}
  \end{enumerate}

  ($\impliedby$) Нека $f_\xi: \R \to \R$ е неотрицателна, интегруема по Риман и нормирана. Дефинираме функцията $F_\xi(x) \coloneqq \int_{-\infty}^x f_\xi(t) dt$. Ще покажем, че за $F_\xi$ са изпълнени свойствата на функция на разпределение:
  \begin{enumerate}
    \item Ако $x < y$, от адитивността на римановия интеграл и неотрицателността на $f_\xi$ следва
    \begin{displaymath}
      F_\xi(x)
      =
      \int_{-\infty}^x f_\xi(t) dt
      \leq
      \int_{-\infty}^x f_\xi(t) dt + \int_x^y f_\xi(t) dt
      =
      \int_{-\infty}^y f_\xi(t) dt
      =
      F_\xi(y).
    \end{displaymath}

    \item $F_\xi$ е непрекъсната отдясно, тъй като
    \begin{displaymath}
      \lim_{h \downarrow 0} F_\xi(x + h)
      =
      \lim_{h \downarrow 0} \int_{-\infty}^{x + h} f_\xi(t) dt
      =
      \int_{-\infty}^x f_\xi(t) dt + \lim_{h \downarrow 0} \int_x^{x + h} f_\xi(t) dt
      =
      \int_{-\infty}^x f_\xi(t) dt
      =
      F_\xi(x).
    \end{displaymath}

    \item От предположението за нормираност имаме
    \begin{displaymath}
      \lim_{x \uparrow \infty} F_\xi(x)
      =
      \lim_{x \uparrow \infty} \int_{-\infty}^x f_\xi(x) dx
      =
      \int_\R f_\xi(x) dx = 1.
    \end{displaymath}

    \item Директно пресмятаме
    \begin{displaymath}
      \lim_{x \downarrow -\infty} \int_{-\infty}^x f_\xi(x) dx
      =
      \int_{-\infty}^{-\infty} f_\xi(x) dx
      =
      0.
    \end{displaymath}
  \end{enumerate}

  Видяхме, че $F_\xi$ удовлетворява свойствата на функция на разпределение и по~\ref{thm:cdf-props} съществува случайна величина $\xi$, чиято плътност е $f_\xi$.

  Освен това $F_\xi$ е абсолютно непрекъсната във всеки затворен интервал $[a, b]$, тъй като тя има интегруема производна почти навсякъде и за $x \in [a, b]$ е изпълнено
  \begin{displaymath}
    F_\xi(x)
    =
    \int_{-\infty}^x f_\xi(x) dx
    =
    \int_{-\infty}^a f_\xi(x) dx + \int_a^x f_\xi(x) dx
    =
    F_\xi(a) + \int_a^x f_\xi(x) dx.
  \end{displaymath}
\end{proof}

\printbibliography

\end{document}
