% arara: pdflatex: { shell: true, interaction: nonstopmode }
% arara: biber
% arara: pdflatex: { shell: true }

\documentclass[numbers=endperiod, DIV=15, bibliography=totocnumbered]{scrartcl}

% Base packages
\usepackage[T2A]{fontenc}
\usepackage[utf8]{inputenc}
\usepackage[bulgarian]{babel}
\usepackage[pdfencoding=unicode]{hyperref}
\usepackage{biblatex}
\usepackage[style=german]{csquotes}

% Base math packages
\usepackage{amsmath}
\usepackage{amssymb}
\usepackage{amsthm}
\usepackage{mathtools}

% Custom packages
\usepackage{../../common/macros}
\usepackage{../../common/theorems}

% Misc packages
\usepackage{ulem} % Line-breaking underlines

% Bibliography
\addbibresource{./references.bib}

% Document
\title{[WIP] Тема 20}
\subtitle{Точкови и интервални оценки за параметрите на нормалното разпределение.}
\author{Янис Василев, \Email{ianis@ivasilev.net}}
\date{29 юни 2019}

\begin{document}

\maketitle

\section{Теория}

Теорията е базирана на~\cite{DimitrovYanev}.

\subsection{Анотация}

Изложената анотацията е взета от конспекта~\cite{Syllabus} за 2018г.

\begin{enumerate}
  \item Определения за точкови и интервални оценки
  \item Свойства на точковите оценки
  \item Неравенство на Рао-Крамер
  \item Доверителни интервали за параметрите на нормалното разпределение
\end{enumerate}

\subsection{Основни понятия}

Считаме, че е зададено вероятностно пространство $(\Omega, \F, \Prob)$.

\begin{definition}[Извадки]
  Нека $\xi$ е случайна величина над $(\Omega, \F, \Prob)$. Множеството от елементарни събития $\Omega$ в статистиката често се нарича \uline{генерална съвкупност}

  \begin{itemize}
    \item Ако случайните величини $\xi_1, \ldots, \xi_n$ са независими две по две и имат същото разпределение като $\xi$, казваме, че $\xi_1, \ldots, \xi_n$ са \uline{наблюдения над $\xi$} и че те са \uline{проста извадка с обем $n$} над генералната съвкупност $\Omega$.
    \item \uline{Функция на правдоподобие $l(x_1, \ldots, x_n)$ на извадката $\xi_1, \ldots, \xi_n$} наричаме, в непрекъснатия случай, съвместната плътност или, в дискретния случай, съвместната функция на вероятностите. При извадки от независими случайни величини, функцията на правдоподобие е просто произведение на съответните плътности или функции на вероятностите.
    \item \uline{Извадково} пространство, съответстващо на извадката $\xi_1, \ldots, \xi_n$, наричаме множеството $\SampleSpace \subseteq \R^n$ от стойности на случайния вектор $(\xi_1, \ldots, \xi_n)$.
    \item \uline{Реализации} на извадката наричаме вектори от $\SampleSpace$. Те моделират истинските данни в математическата статистика, съпоставяйки ги на~\enquote{теоретичната} извадка $(\xi_1, \ldots, \xi_n)$.
  \end{itemize}
\end{definition}

\begin{definition}[Оценки]
  Нека $\xi_1, \ldots \xi_n$ е проста извадка над случайната величина $\xi$, чието разпределение не ни е известно. Целта ни е на базата на тази извадка да оценим някакви числени характеристики $\theta = (\theta_1, \ldots, \theta_n)$ на $\xi$, които напълно да определят разпределението на $\xi$. Обикновено $\theta$ е вектор от моменти на $\xi$ или, в параметричната статистика, $\theta$ е някой от параметрите на класът от разпределения, на който се предполага, че принадлежи $\xi$. Условна вероятност при условие, че $\theta$ е \uline{истинската} стойност на вектора, бележим с
  \begin{displaymath}
    \Prob(\cdot \mid \theta).
  \end{displaymath}

  \begin{itemize}
    \item \uline{Статистика} наричаме всяка измерима функция на извадката. По определение статистиките са случайни величини.

    \item Ако $\theta$ е някаква числена характеристика на $\xi$ и статистиката $t_n = t_n(x_1, \ldots, x_n)$ не зависи от $\theta$, казваме, че $t_n$ е \uline{точкова оценка за $\theta$}.


    \item Стойността $t_n - \Expect(t_n \mid \theta)$ наричаме \uline{случайна грешка}, а стойността $\Expect(t_n \mid \theta) - \theta$ наричаме \uline{систематична грешка} или \uline{изместване} на оценката. Разликата
    \begin{displaymath}
      t_n - \theta = t_n - \Expect(t_n \mid \theta) + \Expect(t_n \mid \theta) - \theta
    \end{displaymath}
    се разпада на систематична и случайна грешка.

    \item Точковата оценка $t_n$ за $\theta$ наричаме \uline{неизместена}, ако $\Expect(t_n \mid \theta) = \theta$, и асимптотично неизместена, ако $\Expect(t_n \mid \theta) \underset {n \mapsto \infty} \longrightarrow \theta$.

    \item Точковата оценка $t_n$ за $\theta$ наричаме \uline{състоятелна}, ако $t_n \underset {n \mapsto \infty} \longrightarrow \theta$ по вероятност, т.е.
    \begin{displaymath}
      \Prob(\Abs{t_n - \theta} > \varepsilon \mid \theta) \underset {n \mapsto \infty} \longrightarrow 0~\forall \varepsilon > 0.
    \end{displaymath}

    Оценката наричаме силно състоятелна, ако сходимостта е почти сигурна, т.е.
    \begin{displaymath}
      \Prob(\sup_{k \geq n} \Abs{t_k - \theta} > \varepsilon \mid \theta) \underset {n \mapsto \infty} \longrightarrow 0~\forall \varepsilon > 0.
    \end{displaymath}

    \item Ако векторът $\theta$ е едномерен (т.е. $\theta$ е число), \uline{интервална оценка} с ниво на доверие $\gamma$ за $\theta$ наричаме двойка точкови оценки $a_n$ и $b_n$ за $\theta$, за които
    \begin{displaymath}
      \Prob(a_n \leq \theta \leq b_n \mid \theta) = \gamma
    \end{displaymath}
    независимо от стойността на $\theta$.
  \end{itemize}
\end{definition}

\printbibliography

\end{document}
