% arara: pdflatex: { shell: true, interaction: nonstopmode }
% arara: biber
% arara: pdflatex: { shell: true }

\documentclass[numbers=endperiod, DIV=15, bibliography=totocnumbered]{scrartcl}

% Base packages
\usepackage[T2A]{fontenc}
\usepackage[utf8]{inputenc}
\usepackage[bulgarian]{babel}
\usepackage[pdfencoding=unicode]{hyperref}
\usepackage{biblatex}
\usepackage[style=german]{csquotes}

% Base math packages
\usepackage{amsmath}
\usepackage{amssymb}
\usepackage{amsthm}
\usepackage{mathtools}

% Misc packages
\usepackage{polynom} % Polynomial long division
\usepackage{ulem} % Line-breaking underlines

% Custom packages
\usepackage{../../common/macros}
\usepackage{../../common/theorems}

% Bibliography
\addbibresource{./references.bib}

% Document
\title{Тема 2 [WIP]}
\subtitle{Симетрични оператори в крайномерни евклидови пространства. Основни свойства. Теорема за диагонализация.}
\author{Янис Василев, \Email{ianis@ivasilev.net}}
\date{4 юли 2019}

\begin{document}

\maketitle

\section{Теория}

Изложената анотацията е взета от конспекта~\cite{Syllabus} за 2018г.

\subsection{Анотация}

Някои твърдения и доказателства са взаимствани от~\cite{Knapp} и~\cite{RoyachkiNotes}.

\begin{enumerate}
  \item Определение за симетричен оператор и матрица на симетричен оператор
  \item Всички характеристични корени на симетричен оператор са реални числа
  \item Всеки два вектора, съответстващи на различни собствени стойности, са ортогонални
  \item Съществува ортонормиран базис на пространството, в който матрицата на симетричен оператор е диагонална
\end{enumerate}

\subsection{Основни понятия}

Нека $V_n$ е фиксирано $n$-мерно евклидово пространство над полето $\R$, т.е. $\R$-линейно пространство снабдено със скаларно произведение $\Prod \cdot \cdot$.

\begin{definition}
  \mbox{}
  \begin{enumerate}
    \item Линейният оператор $L: V_n \to V_n$ наричаме \uline{симетричен} или \uline{самоспрегнат}, ако за произволен вектор $x \in V_n$ е изпълнено равенството
    \begin{displaymath}
      \Prod x {L(x)} = \Prod {L(x)} x.
    \end{displaymath}

    \item Квадратната матрица $A \in \R^{n \times n}$ наричаме \uline{симетрична}, ако $A = A^T$.
  \end{enumerate}
\end{definition}

\begin{theorem}
  Симетричните оператори имат симетрични матрици.
\end{theorem}
\begin{proof}
  Нека $e_1, \ldots, e_n$ е (нареден) ортонормиран базис на $V_n$, нека $L: V_n \to V_n$ е линеен оператор и нека $A = {(a)}_{i,j}$ е матрицата на $L$ спрямо $e_1, \ldots, e_n$.

  Да отбележим първо, че действието на оператора върху произволен вектор се определя напълно от действието му върху базисните вектори.

  За произволни базисни вектори $e_i$ и $e_j$ имаме
  \begin{align*}
    \Prod {e_i} {L(e_j)}
    =
    \Prod {e_i} {\sum_{k=1}^n a_{k,j} e_k}
    =
    \sum_{k=1}^n a_{k,j} \Prod {e_i} {e_k}
    =
    \sum_{k=1}^n a_{k,j} \delta_{i,k}
    =
    a_{i,j}.
  \end{align*}

  ($\implies$) Ако операторът $L$ е симетричен, за произволни базисни вектори получаваме
  \begin{displaymath}
    a_{i,j}
    =
    \Prod {e_i} {L(e_j)}
    =
    \Prod {L(e_i)} {e_j}
    =
    \Prod {e_j} {L(e_i)}
    =
    a_{j,i}.
  \end{displaymath}
  Следователно матрицата $A$ е симетрична.

  ($\impliedby$) Обратно, ако матрицата $A$ е симетрична, за произволни базисни вектори получаваме
  \begin{displaymath}
    \Prod {e_i} {L(e_j)}
    =
    a_{i,j}
    =
    a_{j,i}
    =
    \Prod {e_j} {L(e_i)}
    =
    \Prod {L(e_i)} {e_j}.
  \end{displaymath}
  Следователно операторът $L$ е симетричен.
\end{proof}

\subsection{Собствени стойности на симетрични оператори}

Тъй като в общия случай корените на характеристичния полином на линеен оператор могат да не бъдат реални числа, ще работим с по-базови понятия от собствени стойности и собствени вектори, тъй като не разполагаме с необходимия апарат за комплексни линейни пространства. В частност,~\enquote{имитираме} комплексно скаларно произведение.

\begin{theorem}
  Корените на характеристичния полином $\det(L - \lambda I)$ на симетричен оператор $L:V_n \to V_n$ са реални числа.
\end{theorem}

\begin{proof}
  Нека $\lambda \in \Complex$ е корен на характеристичния полином $\det(L - \lambda I)$. Ще покажем, че $\lambda \in \R$.

  Нека $A = {(a)}_{i,j}$ е матрица на оператора $L$ спрямо базиса $e_1, \ldots, e_n$ на $V_n$. Тогава съществува ненулев комплексен вектор $x = (x_1, \ldots, x_n) \in \Complex^n$, такъв че
  \begin{displaymath}
    (A - \lambda I) x = x,
  \end{displaymath}
  което е еквивалентно на равенството
  \begin{displaymath}
    A x = \lambda x
  \end{displaymath}
  и на системата
  \begin{displaymath}
    \sum_{j=1}^n a_{i,j} x_j = \lambda x_i, i = 1, \ldots, n.
  \end{displaymath}

  За фиксирано $i$ умножаваме двете страни на равенството с комплексно спрегнатото $\CConj{x_i}$ на $x_i$ и сумираме по $i$, за да получим
  \begin{align*}
    \sum_{j=1}^n a_{i,j} x_j
    &=
    \lambda x_i~\mid~\cdot \; \CConj{x_i}
    \\
    \sum_{j=1}^n a_{i,j} \CConj{x_i} x_j
    &=
    \lambda x_i \CConj{x_i} = \lambda \Abs{x_i}^2~~\mid \sum_{i=1}^n \cdot
    \\
    \sum_{i=1}^n \sum_{j=1}^n a_{i,j} \CConj{x_i} x_j
    &=
    \lambda \sum_{i=1}^n \Abs{x_i}^2.
  \end{align*}

  Поради симетричността на матрицата $A$ е изпълнено
  \begin{displaymath}
    \CConj{a_{i,j} \CConj{x_i} x_j}
    =
    a_{j,i} x_i \CConj{x_j},~i, j = 1, \ldots, n,
  \end{displaymath}
  следователно
  \begin{displaymath}
    \CConj{\lambda \sum_{i=1}^n \Abs{x_i}^2}
    =
    \CConj{\sum_{i=1}^n \sum_{j=1}^n a_{i,j} \CConj{x_i} x_j}
    =
    \sum_{i=1}^n \sum_{j=1}^n \CConj{a_{i,j} \CConj{x_i} x_j}
    =
    \sum_{i=1}^n \sum_{j=1}^n a_{j,i} \CConj{x_j} x_i
    =
    \sum_{i=1}^n \sum_{j=1}^n a_{i,j} \CConj{x_i} x_j
    =
    \lambda \sum_{i=1}^n \Abs{x_i}^2.
  \end{displaymath}

  Тъй като $\sum_{i=1}^n \Abs{x_i}^2 > 0$, получаваме, че $\lambda = \CConj{\lambda}$ и $\lambda \in \R$.
\end{proof}

\subsection{Спектрална теорема}

\begin{theorem}[Спектрална теорема]
  Собствените вектори на симетричен оператор образуват ортонормиран базис във $V_n$.
\end{theorem}

\printbibliography

\end{document}
