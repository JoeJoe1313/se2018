% arara: pdflatex: { shell: true, interaction: nonstopmode }
% arara: biber
% arara: pdflatex: { shell: true }

\documentclass[numbers=endperiod, DIV=15, bibliography=totocnumbered]{scrartcl}

% Base packages
\usepackage[T2A]{fontenc}
\usepackage[utf8]{inputenc}
\usepackage[bulgarian]{babel}
\usepackage[pdfencoding=unicode]{hyperref}
\usepackage{biblatex}
\usepackage[style=german]{csquotes}

% Base math packages
\usepackage{amsmath}
\usepackage{amssymb}
\usepackage{amsthm}
\usepackage{mathtools}

% Misc packages
\usepackage{enumitem} % Customization of enum counters
\usepackage{mathabx}
\usepackage{tikz}
\usepackage{float} % Allowing figures inside minipages
\usepackage{ulem} % Line-breaking underlines

% Custom packages
\usepackage{../../common/macros}
\usepackage{../../common/theorems}

% Bibliography
\addbibresource{./references.bib}

% Setup custom packages
\usetikzlibrary{arrows.meta}

% Document
\title{Тема 1}
\subtitle{Уравнение на права и равнина. Формули за разстояния и ъгли. Криви от втора степен.}
\author{Янис Василев, \Email{ianis@ivasilev.net}}
\date{11 юни 2019}

\begin{document}

\maketitle

\section{Анотация}

Изложената анотацията е взета от конспекта~\cite{Syllabus} за 2018г.

\subsection{Теория}

\begin{enumerate}
  \item Прави в равнината и пространството
  \begin{enumerate}
    \item Векторни и параметрични (скаларни) уравнения на права и равнина
    \item Общо уравнение на права в равнината
    \item Декартово уравнение
    \item Взаимно положение на две прави
    \item Нормално уравнение на права
    \item Разстояние от точка до права
    \item Ъгъл между прави
  \end{enumerate}

  \item Равнини в пространството
  \begin{enumerate}
    \item Общо уравнение на равнина
    \item Взаимно положение на две равнини
    \item Нормално уравнение на равнина
    \item Разстояние от точка до равнина
  \end{enumerate}

  \item Криви от втора степен
  \begin{enumerate}
    \item Уравнение на окръжност
    \item Канонични уравнения на елипса, хипербола и парабола
    \item Фокални свойства на елипса, хипербола и парабола
  \end{enumerate}
\end{enumerate}

\subsection{Задачи}

Не е даден списък с възможни задачи, затова съм включил всички видове задачи, свързани с материала, които са давани на упражнения.

\section{Теория}

Теорията е до голяма степен базирана на~\cite{Notes}.

\subsection{Прави в равнината}

Нека $K = Oxy$ е афинна координатна система в равнината $A_2$. Считаме, че е зададена единична отсечка за измерване на дължини.

\begin{definition}
  Нека $l$ е права в $A_2$, нека са дадени $\Point P_0 \in l$ и направляващият за $l$ вектор $v \parallel l$. Очевидно векторът $v$ е ненулев. Нека $r_0 = \V{OP_0}$ и $r = \V{OP}$ са радиус-векторите на точките $\Point P_0$ и $\Point P$ спрямо $K$.

  От тъждествата $r - r_0 = \V{P_0 P} \parallel v \iff P \in l$ и $\forall \lambda \in \R : v \parallel \lambda v$ следва, че правата $l$ удовлетворява уравнението
  \begin{displaymath}
    l: r = r_0 + \lambda v, \lambda \in \R,
  \end{displaymath}
  което наричаме \uline{векторно параметрично уравнение} на $l$ спрямо $K$.

  Нека $\Point P_0$, $\Point P$ и $v$ имат координати $P_0(x_0, y_0)$, $P(x, y)$ и $v(a, b)$ спрямо $K$. Уравненията
  \begin{displaymath}
    l: \begin{cases}
      x = x_0 + \lambda a \\
      y = y_0 + \lambda b
    \end{cases},
    \lambda \in \R
  \end{displaymath}
  наричаме \uline{система скаларни параметрични уравнения} на $l$ спрямо $K$.
\end{definition}

Изхождайки от горната система, имаме $\lambda a b = b(x - x_0) = a(y - y_0)$, откъдето получаваме уравнението $l: bx + (-a)y + (-x_0 + y_0) = 0$.

\begin{definition}
  Нека правата $l$ се задава спрямо $K$ с уравнението $l: Ax + By + C = 0$, където $A, B, C \in \R$, т.е. уравнението е изпълнено за координатите на $\Point P(x, y)$ точно когато $\Point P \in l$.

  Уравнение от този вид наричаме \uline{общо уравнение на правата $l$} спрямо $K$.
\end{definition}

\begin{note}
  От определението е ясно, че е изпълнено $A \neq 0$ или $B \neq 0$, тъй като в противен случай уравнението е еквивалентно на $l: C = 0$ и, в зависимост от стойността на $C$, уравнението задава или празното множество, или цялата равнина $A_2$.
\end{note}

Вече видяхме, че от всяка система скаларни параметрични уравнение за права $l$ можем да намерим поне едно общо уравнение за $l$. Ще докажем съществуване на общо уравнение за равнина по друг начин.

\begin{proposition}
  Всяко уравнение от вида $Ax + By + C = 0$, където $A$ и $B$ не са едновременно равни на $0$, е уравнение на точно една права спрямо $K$.
\end{proposition}
\begin{proof}
  Без ограничение на общността допускаме, че $A \neq 0$.

  Да забележим, че уравнението има поне едно решение, например $x = -\frac C A$ и $y = 0$.

  Нека е дадена точката $P_0(x_0, y_0)$, чиито координати удовлетворяват уравнението. Нека $v$ е ненулевият вектор с координати $v(B, -A)$. Тогава системата
  \begin{align*}
    l: \begin{cases}
      x = x_0 + \lambda B \\
      y = y_0 - \lambda A,
    \end{cases}
    \lambda \in \R
  \end{align*}
  е система скаларни параметрични уравнения за някаква права $l$. Ще покажем, че тази система е еквивалентна на даденото уравнение.

  От една страна, всяко решение на общото уравнение удовлетворява системата, тъй като $C = -Ax_0 - By_0$. Получаваме
  \begin{align*}
    Ax + By + C &= 0, \\
    x &= -\frac C A -\frac {By} A, \\
    x &= \frac {Ax_0 + By_0} A -\frac {By} A, \\
    x &= x_0 + \frac {y_0 - y} A B.
  \end{align*}

  Полагаме $\lambda \coloneqq \frac{y_0 - y} A$, откъдето получаваме
  \begin{displaymath}
    \begin{cases}
      x = x_0 + \lambda B \\
      y = y_0 - \lambda A
    \end{cases}.
  \end{displaymath}

  От друга страна, всяко решение на системата удовлетворява общото уравнение, тъй като
  \begin{displaymath}
    A (x_0 + \lambda B) + B (y_0 - \lambda A) + C
    =
    (A x_0 + B y_0 + C) + (\lambda AB - \lambda AB)
    =
    0 + 0 = 0.
  \end{displaymath}
\end{proof}

\begin{definition}
  Нека правата $l$ има спрямо $K$ общо уравнение $l: Ax + By + C = 0$. Нека още $l$ не е успоредна на оста $Oy$. Тогава $B \neq 0$ и можем да запишем общото уравнение във вида

  \begin{align*}
    l: \frac A B x + y + \frac C B = 0
    &&\sim&&
    l: y = \left(-\frac A B \right) x + \left(-\frac C B \right).
  \end{align*}

  Полагаме $k = -\frac A B$ и $m = -\frac C B$. Уравнението
  \begin{displaymath}
    l: y = kx + m
  \end{displaymath}
  наричаме \uline{декартово уравнение} на $l$ спрямо $K$, а коефициента $k$ наричаме \uline{ъглов коефициент} на $l$ спрямо $K$.
\end{definition}

\begin{definition}
  \mbox{}
  \begin{enumerate}
    \item \uline{Ъгъл между две пресекателни прави} наричаме по-малкия от двата ъгъла, които те сключват.

    Ако е зададена ориентация, \uline{ориентиран ъгъл между пресекателни прави} наричаме ъгъла, на който трябва да бъде завъртяна спрямо зададената ориентация първата права около пресечната си точка с втората права, за да съвпадне с нея.

    \item \uline{Ъгъл между два лъча с общо начало} наричаме по-малкия от двата ъгъла, които те сключват.

    \uline{Ориентиран ъгъл между два лъча с общо начало} наричаме ъгъла, на който трябва да бъде завъртян спрямо зададената ориентация първият лъч относно началната си точка, за да съвпадне с втория.

    \item Нека са дадени ненулеви свободните вектори $u$ и $v$. \uline{Ъгъл между $u$ и $v$} наричаме по-малкия от двата ъгъла между два произволни техни представителя с общо начало и бележим с $\angle(u, v)$.

    \uline{Ориентиран ъгъл между $u$ и $v$} наричаме ъгъла, на който трябва да бъде завъртян някой представител на $u$ спрямо зададената ориентация, за да съвпадне с някой представител на $v$.

    И двете дефиниции са коректни, понеже всички представители на $u$ са колинеарни помежду си и сключват еднакви ъгли с представители на $v$, които също са колинеарни помежду си.
  \end{enumerate}
\end{definition}

\begin{proposition}
  Нека координатната система $K = Oxy$ е ортонормирана и правата $l$ има спрямо $K$ декартово уравнение $l: y = kx + m$. Дефинираме лъчът $r\Ray : y = kx + m \geq 0$, лежащ върху $l$. Тогава $\tan \angle(r\Ray, Ox\Ray) = k$.
\end{proposition}
\begin{proof}
  Нека $u$ е векторът с координати $u(1, k)$ спрямо $K$.

  Ако $k \geq 0$, то $u \upuparrows r\Ray$ и
  \begin{displaymath}
     \tan \angle(r\Ray, Ox\Ray) = \tan \angle(u, Ox\Ray) = k.
  \end{displaymath}

  Ако $k < 0$, то $u \updownarrows r\Ray$ и
  \begin{displaymath}
    \tan \angle(r\Ray, Ox\Ray) = \tan(\pi - \angle(-u, Ox\Ray)) = -\tan \angle(-u, Ox\Ray) = -(-k) = k.
  \end{displaymath}
\end{proof}

\begin{theorem}
  Нека правите $l_1$ и $l_2$ имат спрямо $K$ общи уравнения
  \begin{equation}\label{thm:plane-line-position:scalar-equations}
    l_i : A_i x + B_i y + C_i, i = 1, 2.
  \end{equation}

  Означаваме

  \begin{align*}
    L \coloneqq \begin{pmatrix}
      A_1 & B_1 \\
      A_2 & B_2
    \end{pmatrix}
    &&
    \tilde L \coloneqq \begin{pmatrix}
      A_1 & B_1 & C_1 \\
      A_2 & B_2 & C_2
    \end{pmatrix}.
  \end{align*}

  Тогава
  \begin{enumerate}
    \item $l_1 \equiv l_2 \iff \Rank L = \Rank \tilde L = 1$.
    \item $l_1 \parallel l_2$, но $l_1 \nequiv l_2 \iff 1 = \Rank L < \Rank \tilde L = 2$.
    \item $l_1$ и $l_2$ са пресекателни $\iff \Rank L = \Rank \tilde L = 2$.
  \end{enumerate}
\end{theorem}
\begin{proof}
  Да забележим, че
  \begin{enumerate}
    \item $L$ е подматрица на $\tilde L \implies \Rank L \leq \Rank \tilde L$.
    \item $A_1$ и $B_1$ са коефициенти в общо уравнения на права, съответно поне един от тях е различен от нула и $\Rank L \geq 1$.
    \item Матрицата $\tilde L$ има само два реда, следователно $\Rank \tilde L \leq 2$.
  \end{enumerate}

  Така получаваме $1 \leq \Rank L \leq \Rank \tilde L \leq 2$.

  Разглеждаме матричното уравнение
  \begin{equation}\label{thm:plane-line-position:matrix-equation}
    L
    \begin{pmatrix}
      x \\ y \\ 1
    \end{pmatrix}
    =
    \begin{pmatrix}
      -C_1 \\ -C_2
    \end{pmatrix},
  \end{equation}
  което ни дава едновременните решения на уравненията (\ref{thm:plane-line-position:scalar-equations}).

  Ако $\Rank L \neq \Rank \tilde L$, то по теоремата на Руше матричното уравнение (\ref{thm:plane-line-position:matrix-equation}) няма решение, правите $l_1$ и $l_2$ нямат общи точки и $l_1 \parallel l_2$. Обратно, ако правите са успоредни, те нямат общи точки, матричното уравнение (\ref{thm:plane-line-position:matrix-equation}) няма решения и $\Rank L \neq \Rank \tilde L$.

  Ако $\Rank L = \Rank \tilde L$, то по теоремата на Руше матричното уравнение (\ref{thm:plane-line-position:matrix-equation}) има поне едно решение. Разглеждаме два случая:
  \begin{enumerate}
    \item Ако $\Rank L = \Rank \tilde L = 1$, то редовете на $L$ са линейно зависими, следователно уравненията (\ref{thm:plane-line-position:scalar-equations}) са еквивалентни и задават едно и също множество. Тъй като и двете са уравнения на прави, те задават една и съща права.

    Обратно, нека уравненията (\ref{thm:plane-line-position:scalar-equations}) задават една и съща права и нека за определеност $A_1 \neq 0$. Тогава за произволна точка $P(x, y) \in l_1 \equiv l_2$ имаме
    \begin{displaymath}
      A_1x + B_1y + C_1 = 0
      \implies
      x = - \frac {B_1} {A_1} y - \frac {C_1} {A_1},
    \end{displaymath}
    \begin{align*}
      A_2x + B_2y + C_2 &= 0,
      \\
      - A_2 \left(\frac {B_1} {A_1} y + \frac {C_1} {A_1} \right) + B_2y + C_2 &= 0,
      \\
      \left(B_2 - \frac {A_2} {A_1} B_1 \right) y + \left(C_2 - \frac {A_2} {A_1} C_1 \right) &= 0.
    \end{align*}

    Последното уравнение е еквивалентно на системата
    \begin{align*}
      B_2 = \frac {A_2} {A_1} B_1
      &&
      C_2 = \frac {A_2} {A_1} C_1.
    \end{align*}

    Тогава второто уравнение от (\ref{thm:plane-line-position:scalar-equations}) има вида
    \begin{displaymath}
      l_2:
      A_2 x + B_2 y + C_2 =
      \frac {A_2} {A_1} A_1 x + \frac {A_2} {A_1} B_1 y + \frac {A_2} {A_1} C_1
      = 0,
    \end{displaymath}
    откъдето виждаме, че двете уравнения са пропорционални и следователно $\Rank \tilde L = 1$.

    \item Ако $\Rank L = \Rank \tilde L = 2$, то системата (\ref{thm:plane-line-position:matrix-equation}) има максимален ранг и решението е единствено. Това е пресечната точка на $l_1$ и $l_2$.

    Обратно, ако $l_1$ и $l_2$ имат само една обща точка $P$, то координатите ѝ удовлетворяват (\ref{thm:plane-line-position:matrix-equation}). По теоремата на Руше това или е единственото решение, или има още безброй решения. Но вече видяхме, че системата има безброй решения точно при $\Rank \tilde L = 1$. Следователно координатите на $P$ са единственото решение на (\ref{thm:plane-line-position:matrix-equation}) и матрицата $L$ има максимален ранг 2.
  \end{enumerate}
\end{proof}

\begin{corollary}
  Ако правата $l$ има общо уравнение $l: Ax + By + C = 0$ спрямо $K$, то всяко общо уравнение спрямо $K$ има вида $l: \lambda(Ax + By + C) = 0$ за някое $\lambda \in \R \setminus \{ 0 \}$.
\end{corollary}

\begin{definition}
  Всеки вектор, перпендикулярен на някоя права $l$, се нарича \uline{нормален} за $l$.
\end{definition}

\begin{proposition}
  Ако правата $l$ има общо уравнение $l: Ax + By + C = 0$ спрямо $K$ и $K$ е ортонормирана, то
  \begin{enumerate}
    \item векторът $v(-B, A)$ е ненулев и колинеарен с $l$.
    \item векторът $n(A, B)$ е ненулев и нормален за $l$.
  \end{enumerate}
\end{proposition}
\begin{proof}
  По условие имаме, че $A$ и $B$ не са едновременно равни на $0$, следователно $v$ и $n$ са ненулеви.
  \begin{enumerate}
    \item Разглеждаме точката $P_0(x_0, y_0) \in l$ и точката $P_1(x_0 - B, y_0 + A)$, която също принадлежи на $l$, тъй като
    \begin{displaymath}
      A(x_0 - B) + B(y_0 + A) + C = (Ax_0 + By_0 + C) + (-AB + AB) = 0 + 0 = 0.
    \end{displaymath}
     Тогава $\V{P_0 P_1} \parallel l$. Но векторът $\V{P_0 P_1}(-B, A)$ е равен на $v$, следователно $v \parallel l$.

    \item Тъй като координатната система $K$ е ортонормирана, имаме
    \begin{displaymath}
      \Prod v n = -AB + AB = 0,
    \end{displaymath}
    следователно $v \perp n$ и тъй като $v$ е направляващ за $l$, то $n$ е нормален за $l$.
  \end{enumerate}
\end{proof}

\begin{definition}
  Общото уравнение $l: \alpha x + \beta y + \gamma = 0$ на права $l$ спрямо ортонормирана $K$ се нарича \uline{нормално}, ако е изпълнено условието $\Norm{n(\alpha, \beta)}^2 = \alpha^2 + \beta^2 = 1$.
\end{definition}

\begin{proposition}
  Всяка права има спрямо $K$ точно две нормални уравнения.
\end{proposition}
\begin{proof}
  Нека $l$ има спрямо $K$ общо уравнение $l: Ax + By + C = 0$. Тогава всевъзможните общи уравнения на $l$ спрямо $K$ имат вида
  \begin{displaymath}
    l: \lambda(Ax + By + C) = 0, \lambda \in \R \setminus \{ 0 \}.
  \end{displaymath}

  Но ${(\lambda A)}^2 + {(\lambda B)}^2 = 1 \iff \lambda = \pm \frac 1 {\sqrt{A^2 + B^2}}$.

  Двете нормални уравнения са
  \begin{displaymath}
    l: \pm \frac {Ax + By + C} {\sqrt{A^2 + B^2}} = 0.
  \end{displaymath}
\end{proof}

\begin{theorem}\label{thm:plane-dist}
  Нека координатната система $K$ е ортонормирана и са дадени точката $P_0(x_0, y_0)$ и правата $l$ с нормално уравнение
  \begin{displaymath}
    l: \alpha x + \beta y + \gamma = 0.
  \end{displaymath}

  Означаваме $F(x, y) \coloneqq \alpha x + \beta y + \gamma$. Тогава разстоянието между $P_0$ и $l$ е
  \begin{displaymath}
    \Dist(P_0, l) = \Abs{F(x_0, y_0)},
  \end{displaymath}
  а величината $\ODist(P, l) = F(x_0, y_0)$ се нарича \uline{ориентирано разстояние} между $P_0$ и $l$.
\end{theorem}
\begin{proof}
  Нека $P_1(x_1, y_1)$ е ортогоналната проекция на $P_0$ върху $l$. Понеже $\V{P_0 P_1} \perp l$, то $\V{P_0 P_1} \parallel n(\alpha, \beta)$, следователно $\exists \lambda \in \R: \V{P_0 P_1} = \lambda n$. Тогава
  \begin{displaymath}
    \Dist(P_0, l)
    =
    \Norm{\V{P_0 P_1}}
    =
    \Abs{\lambda} \Norm{n}
    =
    \Abs{\lambda}.
  \end{displaymath}

  Намираме $\lambda$ от $\V{P_0 P_1}(x_1 - x_0, y_1 - y_0) = \lambda n(\alpha, \beta)$:
  \begin{displaymath}
    \begin{cases}
      x_1 - x_0 = \lambda \alpha \\
      y_1 - y_0 = \lambda \beta
    \end{cases}
    \sim
    \begin{cases}
      x_1 = x_0 + \lambda \alpha \\
      y_1 = y_0 + \lambda \beta.
    \end{cases}
  \end{displaymath}

  Заместваме в нормалното уравнение на $l$:
  \begin{align*}
    \alpha x_1 + \beta y_1 + \gamma &= 0,
    \\
    \alpha (x_0 + \lambda \alpha) + \beta (y_0 + \lambda \beta) + \gamma &= 0,
    \\
    (\alpha x_0 + \beta y_0 + \gamma) + \lambda(\alpha^2 + \beta^2) &= 0,
    \\
    F(x_0, y_0) + \lambda &= 0,
  \end{align*}
  откъдето следва $\lambda = -F(x_0, y_0)$ и $\Dist(P_0, l) = \Abs{\lambda} = \Abs{F(x_0, y_0)}$.
\end{proof}

\begin{theorem}
  Нека са дадени правите $l_1$ и $l_2$ с нормални уравнения спрямо ортонормирана $K$
  \begin{displaymath}
    l_i: \alpha_i x + \beta_i y + \gamma_i = 0, i = 1, 2.
  \end{displaymath}

  Тогава за ъгълът между $l_1$ и $l_2$ е $\angle(l_1, l_2) = \arccos \Abs{\alpha_1 \alpha_2 + \beta_1 \beta_2}$.
\end{theorem}
\begin{proof}
  Нека векторите $n_i(\alpha_i, \beta_i)$ са нормални за $n_i, i = 1, 2$.

  Разглеждаме два случая:
  \begin{enumerate}
    \item Ако $0 < \angle(n_1, n_2) \leq \frac {\pi} 2$, то $\angle(l_1, l_2) = \angle(n_1, n_2)$ и
    \begin{displaymath}
      \cos \angle (l_1, l_2)
      =
      \cos \angle(n_1, n_2)
      =
      \Abs{\cos \angle (n_1, n_2)}.
    \end{displaymath}

    \item Ако $\frac {\pi} 2 < \angle(n_1, n_2) \leq \pi$, то $\angle(l_1, l_2) = \pi - \angle(n_1, n_2)$
    \begin{displaymath}
      \cos \angle (l_1, l_2)
      =
      \cos (\pi - \angle(n_1, n_2))
      =
      -\cos \angle(n_1, n_2)
      =
      \Abs{\cos \angle (n_1, n_2)}.
    \end{displaymath}
  \end{enumerate}

  И в двата случая имаме
  \begin{displaymath}
    \cos \angle(l_1, l_2)
    =
    \Abs{\cos \angle(n_1, n_2)}
    =
    \frac {\Abs {\Prod {n_1} {n_2}}} {\Norm{n_1} \Norm{n_2}}
    =
    \Abs{\alpha_1 \alpha_2 + \beta_1 \beta_2}
  \end{displaymath}
  и
  \begin{displaymath}
    \angle(l_1, l_2)
    =
    \arccos \Abs{\alpha_1 \alpha_2 + \beta_1 \beta_2}.
  \end{displaymath}
\end{proof}

\subsection{Равнини в пространството}

\begin{note}
  Някои от теоремите са напълно аналогични на тези за прави в равнината и ще пропуснем техните доказателства.
\end{note}

Нека $K = Oxyz$ е афинна координатна система в пространството $A_3$. Считаме, че е зададена единична отсечка за измерване на дължини.

\begin{definition}
  Нека равнината $\pi$ се задава спрямо $K$ с уравнението $\pi: Ax + By + Cz + D = 0$, където $A, B, C, D \in \R$, т.е. уравнението е изпълнено за координатите на $\Point P(x, y, z)$ точно когато $\Point P \in \pi$.

  Уравнение от този вид наричаме \uline{общо уравнение на равнината $\pi$} спрямо $K$.
\end{definition}

\begin{note}
  От определението е ясно, че е изпълнено $A \neq 0$ или $B \neq 0$ или $C \neq 0$, тъй като в противен случай уравнението е еквивалентно на $l: D = 0$ и, в зависимост от стойността на $D$, уравнението задава или празното множество, или цялото пространство $A_3$.
\end{note}

\begin{proposition}
  \mbox{}
  \begin{enumerate}
    \item Всяка равнина има поне едно общо уравнение спрямо $K$.
    \item Всяко уравнение от вида $Ax + By + Cz + D = 0$, където $A$, $B$ и $C$ не са едновременно равни на $0$, е уравнение точно една равнина спрямо $K$.
  \end{enumerate}
\end{proposition}
\begin{proof}
  \mbox{}
  \begin{enumerate}
    \item Нека са дадени равнината $\pi$, точката $P_0(x_0, y_0, z_0) \in \pi$ и компланарните с $\pi$ неколинеарни вектори $u_i(a_i, b_i, c_i), i = 1, 2$.

    Разглеждаме произволна точка $P(x, y, z)$. Имаме $\Point P \in l$ точно когато $\V{P_0 P} \parallel \pi$, а последното условие е аналогично на това векторите $\V{P_0 P}$, $u_1$ и $u_2$ да бъдат компланарни. Това условие може да се изрази в координатен вид,
    \begin{displaymath}
      \det \begin{pmatrix}
        x - x_0 & a_1 & a_2 \\
        y - y_0 & b_1 & b_2 \\
        z - z_0 & c_1 & c_2 \\
      \end{pmatrix}
      =
      0.
    \end{displaymath}
    Полагаме
    \begin{align*}
      A \coloneqq \det \begin{pmatrix}
        b_1 & b_2 \\
        c_1 & c_2
      \end{pmatrix}
      &&
      B \coloneqq -\det \begin{pmatrix}
        a_1 & a_2 \\
        c_1 & c_2
      \end{pmatrix}
      &&
      C \coloneqq \det \begin{pmatrix}
        a_1 & a_2 \\
        b_1 & b_2
      \end{pmatrix}.
    \end{align*}
    Тъй като векторите $u_1$ и $u_2$ не са колинеарни, имаме
    \begin{displaymath}
      \Rank \begin{pmatrix}
        a_1 & a_2 \\
        b_1 & b_2 \\
        c_1 & c_2 \\
      \end{pmatrix} = 2,
    \end{displaymath}
    следователно поне едно от адюнгираните количества $A$, $B$ или $C$ на голямата матрица ще бъде различно от $0$.

    След като разложим пресмятането на детерминантата по първия стълб, можем да запишем полученото уравнение във вида
    \begin{displaymath}
      \det \begin{pmatrix}
        x - x_0 & a_1 & a_2 \\
        y - y_0 & b_1 & b_2 \\
        z - z_0 & c_1 & c_2 \\
      \end{pmatrix}
      =
      Ax + By + Cz - (Ax_0 + By_0 + Cz_0) = 0,
    \end{displaymath}
    откъдето след полагането $D \coloneqq - (Ax_0 + By_0 + Cz_0)$ получаваме общо уравнение за $\pi$.

    \item Нека е дадено уравнението $Ax + By + Cz + D = 0$, където $A$, $B$ и $C$ не са едновременно равни на $0$. Без ограничение на общността допускаме, че $A \neq 0$.

    Да забележим, че уравнението има поне едно решение, например $x = -\frac D A$ и $y = z = 0$.

    Нека са дадени векторите $u_1(-B, A, 0)$ и $u_2 \left(-\frac C A, 0, 1 \right)$ и точките $P_0(x_0, y_0, z_0)$ и $P(x, y, z)$. Нека освен това координатите на $P_0$ удовлетворяват уравнението. Тогава $D = -(Ax_0 + By_0 + Cz_0)$.

    Нека $\pi$ е равнината, образувана от векторите $u_1$ и $u_2$. Тогава $\V{P_0 P} \in \pi \iff$ трите вектора са компланарни. Ако изразим това условие в координатен вид, ще получим оригиналното уравнение. Наистина,
    \begin{displaymath}
      \det \begin{pmatrix}
        x - x_0 & -B & -\frac C A \\
        y - y_0 & A & 0 \\
        z - z_0 & 0 & 1 \\
      \end{pmatrix}
      =
      Ax + By + Cz - (Ax_0 + By_0 + Cz_0) = Ax + By + Cz + D = 0.
    \end{displaymath}

    Следователно $Ax + By + Cz + D$ е уравнение равнината $\pi$ спрямо $K$.
  \end{enumerate}
\end{proof}

\begin{theorem}
  Нека правите $\alpha_1$ и $\alpha_2$ имат спрямо $K$ общи уравнения
  \begin{equation}
    \pi_i : A_i x + B_i y + C_i z + D_i, i = 1, 2.
  \end{equation}

  Означаваме

  \begin{align*}
    L \coloneqq \begin{pmatrix}
      A_1 & B_1 & C_1 \\
      A_2 & B_2 & C_2
    \end{pmatrix}
    &&
    \tilde L \coloneqq \begin{pmatrix}
      A_1 & B_1 & C_1 & D_1 \\
      A_2 & B_2 & C_2 & D_2
    \end{pmatrix}.
  \end{align*}

  Тогава
  \begin{enumerate}
    \item $\pi_1 \equiv \pi_2 \iff \Rank L = \Rank \tilde L = 1$.
    \item $\pi_1 \parallel \pi_2$, но $\pi_1 \nequiv \pi_2 \iff 1 = \Rank L < \Rank \tilde L = 2$.
    \item $\pi_1$ и $\pi_2$ са пресекателни $\iff \Rank L = \Rank \tilde L = 2$.
  \end{enumerate}
\end{theorem}

\begin{definition}
  Всеки вектор, перпендикулярен на някоя равнина $\pi$, се нарича \uline{нормален} за $\pi$.
\end{definition}

\begin{proposition}
  Ако равнината $\pi$ има общо уравнение $\pi: Ax + By + Cz + D = 0$ спрямо $K$ и $K$ е ортонормирана, то
  \begin{enumerate}
    \item векторите $u(-B, A, 0)$ и $v(-C, 0, A)$ са неколинеарни, ненулеви и компланарни с $\pi$.
    \item векторът $n(A, B, C)$ е ненулев и нормален за $l$.
  \end{enumerate}
\end{proposition}
\begin{proof}
  \item Разглеждаме точката $P_0(x_0, y_0, z_0) \in l$ и точките $P_1(x_0 - B, y_0 + A, z_0)$ и $P_2(x_0 - C, y_0, z_0 + A)$, които също принадлежат на $l$, тъй като
  \begin{align*}
    A(x_0 - B) + B(y_0 + A) + Cz_0 + D = (Ax_0 + By_0 + Cz_0 + D) + (-AB + AB) = 0 + 0 = 0, \\
    A(x_0 - C) + By_0 + C(z_0 + A) + D = (Ax_0 + By_0 + Cz_0 + D) + (-AC + AC) = 0 + 0 = 0.
  \end{align*}

  Тогава векторите $\V{P_0 P_1}$ и $\V{P_0 P_2}$ са компланарни с $\pi$. Но $\V{P_0 P_1} = u$ и $\V{P_0 P_2} = v$, следователно $u$ и $v$ са компланарни с $\pi$.

  \item Тъй като координатната система $K$ е ортонормирана, имаме
  \begin{displaymath}
    \Prod u n = -AB + AB = 0 = -AC + AC = \Prod v n,
  \end{displaymath}
  следователно $u \perp n$ и $v \perp n$. Тогава $n$ е перпендикулярен и на всички линейни комбинации на $u$ и $v$ и понеже $u$ и $v$ образуват равнината $\pi$, то $n$ е нормален за $\pi$.
\end{proof}

\begin{definition}
  Общото уравнение $\pi: \alpha x + \beta y + \gamma z + \delta = 0$ на равнина $\pi$ спрямо ортонормирана $K$ се нарича \uline{нормално}, ако е изпълнено условието $\Norm{n(\alpha, \beta, \gamma)}^2 = \alpha^2 + \beta^2 + \gamma^2 = 1$.
\end{definition}

\begin{proposition}
  Всяка равнина има спрямо ортонормирана $K$ точно две нормални уравнения.
\end{proposition}

\begin{theorem}
  Нека координатната система $K$ е ортонормирана и са дадени точката $P_0(x_0, y_0, z_0)$ и правата $l$ с нормално уравнение
  \begin{displaymath}
    l: \alpha x + \beta y + \gamma z + \delta = 0.
  \end{displaymath}

  Означаваме $F(x, y, z) \coloneqq \alpha x + \beta y + \gamma z + \delta$. Тогава разстоянието между $P_0$ и $\pi$ е
  \begin{displaymath}
    \Dist(P_0, \pi) = \Abs{F(x_0, y_0, z_0)},
  \end{displaymath}
  а величината $\ODist(P, \pi) = F(x_0, y_0, z_0)$ се нарича \uline{ориентирано разстояние} между $P_0$ и $\pi$.
\end{theorem}

\subsection{Криви от втора степен}

\begin{note}
  Ще пропуснем доказателствата на фокалните свойства на кривите.
\end{note}

Нека $K = Oxy$ е ортонормирана координатна система в равнината $A_2$. Считаме, че са зададени единична отсечка за измерване на дължини и ориентация.

\begin{definition}
  Една крива наричаме \uline{алгебрична крива от степен $d$}, ако координатите на произволна нейна точка имат алгебрично уравнение от степен $d$.
\end{definition}

\begin{definition}
  \uline{Окръжност} $k$ с \uline{център} $P_0(x_0, y_0)$ и \uline{радиус} $r > 0$ наричаме множеството от всички точки $P$ на разстояние $r$ от $P_0$.

  Точката $P(x, y)$ е от $k$ точно когато
  \begin{displaymath}
    \Dist(P_0, P)
    =
    \Norm{P_0 P}
    =
    \sqrt{{(x - x_0)}^2 + {(y - y_0)}^2} = r
    \iff
    {(x - x_0)}^2 + {(y - y_0)}^2 = r^2.
  \end{displaymath}

  Уравнението ${(x - x_0)}^2 + {(y - y_0)}^2 = r$ наричаме \uline{централно уравнение на окръжността $k$}.

  В частност, оттук следва, че окръжностите са алгебрични криви от втора степен.

  \bigskip
  \begin{minipage}{0.5\textwidth}
    Нека сега $P_0 x\Ray$ е лъчът с начало $P_0$, успореден на $Ox\Ray$. Ако означим с $\varphi$ ориентирания ъгъл между лъчите $P_0x\Ray$ и $P_0 P\Ray$, от определенията за $\sin$ и $\cos$ за координатите на вектора $\V{P_0 P}$ получаваме
    \begin{displaymath}
      \begin{cases}
        x - x_0 = r \cos \varphi \\
        y - y_0 = r \sin \varphi
      \end{cases}.
    \end{displaymath}

    \uline{Скаларни параметрични уравнения на окръжност} наричаме системата
    \begin{displaymath}
      \begin{cases}
        x = x_0 + r \cos \varphi \\
        y = y_0 + r \sin \varphi
      \end{cases},
      \varphi \in [0, 2\pi).
    \end{displaymath}
  \end{minipage}
  \begin{minipage}{0.5\textwidth}
    \begin{figure}[H]
      \begin{center}
        \begin{tikzpicture}
          \draw[->] (-0.5, 0) -- (5, 0);
          \draw[->] (0, -0.5) -- (0, 5);

          \draw[->, thick, densely dotted] (2.5, 2.5) -- (5, 2.5);
          \draw[->, thick, densely dotted] (2.5, 2.5) -- (4.5, 4.5);
          \draw[->, thick, densely dotted] (2.5, 2.5) -- (2.5, 5);

          \draw (3, 2.5) arc(0:45:0.5cm) node[midway, right] {\footnotesize $\varphi$};
          \draw (3.41, 3.91) arc(0:45:-0.5cm) node[midway, left] {\footnotesize $\varphi$};

          \filldraw [black] (0, 0) circle (2pt);
          \node[right] at (0, -0.3) {$O$};

          \draw (2.5, 2.5) circle(2);
          \filldraw [black] (2.5, 2.5) circle (2pt);
          \node at (2.5, 2.2) {$P_0$};

          \filldraw [black] (3.91, 3.91) circle (2pt);
          \node[right] at (3.91, 3.91) {$P$};

          \draw[thick] (3.91, 3.91) -- (3.91, 2.5);
          \node at (2.8, 3) {\footnotesize $r$};
          \node at (3.5, 3) {\footnotesize $\sin \varphi$};
          \draw[thick] (3.91, 3.91) -- (2.5, 3.91);
          \node at (3.1, 4.1) {\footnotesize $\cos \varphi$};
        \end{tikzpicture}
      \end{center}
      \caption{Окръжност}\label{fig:circle}
    \end{figure}
  \end{minipage}
\end{definition}

\begin{definition}
  \hfill\allowbreak
  \bigskip

  \begin{minipage}{0.5\textwidth}
    \uline{Елипса} с \uline{фокуси} $\Point F_0$ и $\Point F_1$ и \uline{голяма полуос} $a > 0$ наричаме множеството $k$ от точки $P$, за които е изпълнено $\Norm{F_0 P} + \Norm{F_1 P} = 2a$.

    Разглеждаме частния случай, когато $\Point F_0$ и $\Point F_1$ имат координати $F_0(-c, 0)$ и $F_1(c, 0)$ спрямо $K$ за някоя константа $c \geq 0$ с $c < a$, наречена \uline{линеен ексцентрицитет} на $k$. За всяка елипса съществува единствена координатна система, в която уравненията на фокусите имат този прост вид. Въвеждаме следните допълнителни понятия
  \end{minipage}
  \begin{minipage}{0.5\textwidth}
    \begin{figure}[H]
      \begin{center}
        \begin{tikzpicture}
          \draw[domain=0:360, smooth, variable=\x, thick] plot ({2*cos(\x)}, {3/2*sin(\x)});

          \filldraw [black] (-1.32, 0) circle (2pt);
          \node[right] at (-1.32, 0) {$F_0$};

          \filldraw [black] (1.32, 0) circle (2pt);
          \node[right] at (1.32, 0) {$F_1$};

          \filldraw [black] (-0.83, 1.36) circle (2pt);
          \draw[-{Latex[length=3mm]}] (-1.32, 0) -- (-0.83, 1.36);
          \draw[-{Latex[length=3mm]}] (-0.83, 1.36) -- (1.32, 0);
          \node at (-0.83, 1.6) {$M$};
        \end{tikzpicture}
      \end{center}
      \caption{Елипса}\label{fig:ellipse}
    \end{figure}
  \end{minipage}

  \begin{itemize}
    \item Величината $b \coloneqq \sqrt{a^2 - c^2}$ наричаме \uline{малка полуос} на $k$. За елипси имаме $b = \sqrt{a^2 - c^2} \leq \sqrt{a^2} = a$.

    \item Величината $e \coloneqq \frac c a$ наричаме \uline{(числен) ексцентрицитет} на $k$. За елипси имаме  $e = \frac c a = \frac {\sqrt{a^2 - b^2}} {\sqrt{a^2}} < 1$.

    \item \uline{Директриси} на елипсата $k$ наричаме правите с уравнения $d_1: x = - \frac a e$ и $d_2: x = \frac a e$.
  \end{itemize}

  Разписваме уравнението за $k$ покоординатно:
  \begin{align*}
    \Norm{F_0 P} + \Norm{F_1 P} &= 2a \\
    \Norm{F_1 P} &= 2a - \Norm{F_0 P} \mid {(\cdot)}^2 \\
    \Norm{F_1 P}^2 &= 4a^2 - 4a \Norm{F_0 P} + \Norm{F_0 P}^2 \\
    4a \Norm{F_0 P} &= 4a^2 + \Norm{F_0 P}^2 - \Norm{F_1 P}^2 \mid \cdot / 4a \\
    \Norm{F_0 P} &= a + \frac {\Norm{F_0 P}^2 - \Norm{F_1 P}^2} {4a} \\
    \sqrt{{(x+c)}^2 + y^2} &= a + \frac {{(x + c)}^2 + y^2 - {(x - c)}^2 - y^2} {4a} \\
    \sqrt{{(x+c)}^2 + y^2} &= a + \frac c a x \mid {(\cdot)}^2 \\
    x^2 + 2xc + c^2 + y^2 &= a^2 + 2 xc + \frac {c^2} {a^2} x^2 \\
    \left(1 - \frac {c^2} {a^2} \right) x^2 + y^2 &= a^2 - c^2 \mid / {(a^2 - c^2)} \\
    \frac {x^2} {a^2} + \frac {y^2} {a^2 - c^2} &= 1.
  \end{align*}

  Така получаваме \uline{метрично канонично уравнение на елипсата $k$} спрямо $K$:
  \begin{displaymath}
    k: \frac {x^2} {a^2} + \frac {y^2} {b^2} = 1.
  \end{displaymath}

  Веднага виждаме, че елипсите са алгебрични криви от втора степен и че централното уравнение на окръжност е частен случай с $a = b = r \iff c = 0$.

  \uline{Скаларни параметрични уравнения на елипса} можем да изведем аналогично на тези за окръжност. В нашия опростен случай те имат вида
  \begin{displaymath}
    \begin{cases}
      x = a \cos \varphi \\
      y = b \sin \varphi
    \end{cases},
    \varphi \in [0, 2\pi).
  \end{displaymath}

  \begin{theorem}[Фокално свойство на елипса]
    Всеки лъч с начало $\Point F_0$ след отражение в произволна точка $M$ от елипсата минава през $\Point F_1$ (фиг.~\ref{fig:ellipse}).
  \end{theorem}
\end{definition}

\begin{definition}
  \hfill\allowbreak
  \bigskip

  \begin{minipage}{0.5\textwidth}
    \uline{Хипербола} с \uline{фокуси} $\Point F_0$ и $\Point F_1$ и \uline{реална полуос} $a > 0$ наричаме двусвързаното множество $k$ от точки $P$, за които е изпълнено $\Abs{\Norm{F_0 P} - \Norm{F_1 P}} = 2a$.

    Разглеждаме частния случай, когато $\Point F_0$ и $\Point F_1$ имат координати $F_0(-c, 0)$ и $F_1(c, 0)$ спрямо $K$ за някоя константа $c \geq 0$ с $c > a$, наречена \uline{линеен ексцентрицитет} на $k$. За всяка хипербола съществува единствена координатна система, в която уравненията на фокусите имат този прост вид. Въвеждаме следните допълнителни понятия
  \end{minipage}
  \begin{minipage}{0.5\textwidth}
    \begin{figure}[H]
      \begin{center}
        \begin{tikzpicture}
          \draw[domain=-2.2:2.2, smooth, variable=\x, thick] plot ({cosh(\x)/2}, {sinh(\x)/2});
          \draw[domain=-2.2:2.2, smooth, variable=\x, thick] plot ({-cosh(\x)/2}, {sinh(\x)/2});

          \filldraw [black] (-2.82, 0) circle (2pt);
          \node[right] at (-2.82, 0) {$F_0$};

          \filldraw [black] (2.82, 0) circle (2pt);
          \node[right] at (2.82, 0) {$F_1$};

          \filldraw [black] (-1.88, 1.81) circle (2pt);
          \draw[-{Latex[length=3mm]}] (-2.82, 0) -- (-1.88, 1.81);
          \draw[-{Latex[length=3mm]}] (-1.88, 1.81) -- (-2.82, 2.17);
          \draw[thick, dotted] (2.82, 0) -- (-1.88, 1.81);
          \node at (-2.3, 1.6) {$M$};
        \end{tikzpicture}
      \end{center}
      \caption{Хипербола}\label{fig:hyperbola}
    \end{figure}
  \end{minipage}

  \begin{itemize}
    \item Величината $b \coloneqq \sqrt{c^2 - a^2}$ наричаме \uline{имагинерна полуос} на $k$.

    \item Аналогично на елипсите, величината $e \coloneqq \frac c a$ наричаме \uline{(числен) ексцентрицитет} на $k$. За хиперболи имаме  $e = \frac c a = \frac {\sqrt{a^2 + b^2}} {\sqrt{a^2}} > 1$.

    \item Аналогично на елипсите, \uline{директриси} на хиперболата $k$ наричаме правите с уравнения $d_1: x = - \frac a e$ и $d_2: x = \frac a e$.
  \end{itemize}

  Напълно аналогично на случая с елипса, разписвайки уравнението за $k$ покоординатно стигаме до уравнението
  \begin{displaymath}
    \frac {x^2} {a^2} + \frac {y^2} {a^2 - c^2} = 1.
  \end{displaymath}

  Единствената разлика е, че тук $b = -(a^2 - c^2)$. Така получаваме \uline{метрично канонично уравнение} на хиперболата $k$ спрямо $K$:
  \begin{displaymath}
    k: \frac {x^2} {a^2} - \frac {y^2} {b^2} = 1.
  \end{displaymath}

  Веднага виждаме, че хиперболите са алгебрични криви от втора степен.

  \uline{Скаларните параметрични уравнения на хипербола} са различни за левия и десния клон на хиперболата. В нашия опростен случай те имат вида
  \begin{displaymath}
    \begin{cases}
      x = \pm a \cosh \varphi \\
      y = b \sinh \varphi
    \end{cases},
    \varphi \in [0, 2\pi).
  \end{displaymath}

  \begin{theorem}[Фокално свойство на хипербола]
    Всеки лъч с начало $\Point F_0$ след отражение в произволна точка $M$ от хиперболата лежи върху правата, минаваща през $\Point M$ и $\Point F_1$ (фиг.~\ref{fig:hyperbola}).
  \end{theorem}
\end{definition}

\begin{definition}
  \hfill\allowbreak
  \bigskip

  \begin{minipage}{0.5\textwidth}
    \uline{Парабола} с \uline{фокус} $\Point F$ и неминаваща през $F$ права $d$, наречена \uline{директриса}, наричаме множеството $k$ от точки $P$, за които е изпълнено $\Norm{F P} = \Dist(d, P)$. Дефинираме \uline{линейния и числения ексцентрицитет} на параболата $K$ да бъдат $c = e = 1$.

    Разглеждаме частния случай, когато $d$ има уравнение $d: x = - \frac p 2$ и $\Point F$ има координати $F \left(\frac p 2, 0 \right)$ спрямо $K$ за някоя константа $p > 0$, наречена \uline{параметър}. За всяка парабола съществува единствена координатна система, в която уравненията на фокуса и директрисата имат този прост вид.
  \end{minipage}
  \begin{minipage}{0.5\textwidth}
    \begin{figure}[H]
      \begin{center}
        \begin{tikzpicture}
          \draw[domain=-2.5:2.5, smooth, variable=\y, thick] plot ({\y*\y}, {\y});

          \filldraw [black] (0.5, 0) circle (2pt);
          \node[right] at (0.5, 0) {$F$};

          \draw (-0.5, -3) -- (-0.5, 3) node[midway, left] {$d$};

          \filldraw [black] (2.25, 1.5) circle (2pt);
          \draw[-{Latex[length=3mm]}] (0.5, 0) -- (2.25, 1.5);
          \draw[-{Latex[length=3mm]}] (2.25, 1.5) -- (6.5, 1.5);
          \draw[thick, dotted] (-0.5, 1.5) -- (2.25, 1.5);
          \node at (2.25, 1.8) {$M$};
        \end{tikzpicture}
      \end{center}
      \caption{Парабола}\label{fig:parabola}
    \end{figure}
  \end{minipage}

  Нека $\Point P$ има координати $P(x, y)$ спрямо $K$. Тъй като уравнението $d: x = - \frac p 2$ е нормално, теорема~\ref{thm:plane-dist} ни дава $\Dist(d, P) = x + \frac p 2$. Разписваме уравнението за $k$ покоординатно:
  \begin{align*}
    \Norm{F P} &= \Dist(d, P) \\
    {\Norm{F P}}^2 &= {\Dist(d, P)}^2 \\
    {\left(x - \frac p 2 \right)}^2 + y^2 &= {\left(x + \frac p 2 \right)}^2 \\
    y^2 &= {\left(x + \frac p 2 \right)}^2 - {\left(x - \frac p 2 \right)}^2 \\
    y^2 &= 2px.
  \end{align*}

  Така получаваме \uline{метрично канонично уравнение на параболата $k$} спрямо $K$:
  \begin{displaymath}
    k: y^2 = 2px.
  \end{displaymath}

  Веднага виждаме, че параболите са алгебрични криви от втора степен.

  Ако вземем $y$ за параметър, получаваме следните \uline{скаларни параметрични уравнения} на параболата $k$ спрямо $K$:
  \begin{displaymath}
    k: \begin{cases}
      x = \frac{\varphi^2} {2p} \\
      y = \varphi,
    \end{cases}
    \varphi \in \R.
  \end{displaymath}

  \begin{theorem}[Фокално свойство на парабола]
    Всеки лъч с начало $\Point F$ след отражение в произволна точка $M$ от параболата, става перпендикулярен на директрисата $d$ (фиг.~\ref{fig:parabola}).
  \end{theorem}
\end{definition}

\section{Примерни задачи}

Включил съм няколко по-обширни задачи от~\cite{Notes}.

\subsection{Прави в равнината}

\begin{exercise}
  Точките $A$ и $B$ и правите $a$ и $b$ имат спрямо ортонормирана координатна система $K = Oxy$ координати $A(1, -2)$, $B(0, -1)$ и общи уравнения
  \begin{align*}
    &a: 3x + 4y + 2 = 0, \\
    &b: 5x - 12y + 1 = 0.
  \end{align*}

  Да се намерят:
  \begin{enumerate}[label=\alph*)]
    \item Общо уравнение на правата $l$ през $\Point A$, успоредна на $a$
    \item Общо уравнение на правата $m$ през $\Point B$, перпендикулярна на $b$
    \item Общо уравнение на правата $AB$
    \item Координатите на $\Point B'$, която е ортогонално симетрична на $\Point B$ относно правата $a$
    \item Разстоянието от $\Point A$ до $a$
    \item Общо уравнение на ъглополовящата на правите $a$ и $b$
    \item Ъгълът между правите $a$ и $b$
    \item Общо уравнение на правата $t$ през $\Point B$ и $\Point T = a \cap b$
    \item Да се определи положението на $\Point A$ и $\Point B$ спрямо правата $a$
  \end{enumerate}
\end{exercise}

\begin{solution}
  \begin{enumerate}[label=\alph*)]
    \item Правата $l$ е успоредна на $a$, следователно тя има общо уравнение от вида
    \begin{displaymath}
      l: 3x + 4y + C = 0,
    \end{displaymath}

    където $C$ е подбрано спрямо условието $l$ да минава през $\Point A$, т.е.
    \begin{displaymath}
      C = - 3 \cdot 1 - 4 \cdot (-2) = 5.
    \end{displaymath}

    И така, $l$ има общо уравнение $l: 3x + 4y + 5 = 0$.

    \item Правата $m$ е перпендикулярна на $b$, следователно нормалните вектори на $b$ (например $n_b(5, -12)$) са колинеарни с $m$. Нека $\Point P$ има координати $P(x, y)$ спрямо $K$. От условието $P \in m \iff \V{BP} \parallel n_b$ намираме общото уравнение
    \begin{displaymath}
      m: \det
      \begin{pmatrix}
        x & 5 \\
        y + 1 & -12
      \end{pmatrix}
      = 0
      \text{ или } m: 12x + 5y + 5 = 0.
    \end{displaymath}

    \item Нека $\Point P$ има координати $P(x, y)$ спрямо $K$. От условието $P \in AB \iff \V{BP} \parallel \V{BA}$ намираме общото уравнение
    \begin{displaymath}
      AB: \det
      \begin{pmatrix}
        x & 1 \\
        y + 1 & -1
      \end{pmatrix}
      = 0
      \text{ или } AB: x + y + 1 = 0.
    \end{displaymath}

    \item Нека $B'$ има спрямо $K$ координати $B'(x', y')$. Първо намираме правата $BB'$ от условието $BB' \perp a$, което е еквивалентно на $BB' \parallel n_a(3, 4)$. Нека $\Point P$ има координати $(x, y)$ спрямо $K$. От условието $P \in BB' \iff \V{BP} \parallel n_a$ намираме общото уравнение
    \begin{displaymath}
      BB': \det
      \begin{pmatrix}
        x & 3 \\
        y + 1 & 4
      \end{pmatrix}
      = 0
      \text{ или } BB': 4x - 3y - 3 = 0.
    \end{displaymath}

    Координатите на пресечната точка $B_a$ на $a$ и $BB'$ (ортогоналната проекция на $B$ върху $a$) намираме от системата
    \begin{displaymath}
      \begin{cases}
        3x + 4y + 2 = 0 \mid (\times 3) \\
        4x - 3y - 3 = 0 \mid (\times 4)
      \end{cases}
      \sim
      \begin{cases}
        9x + 12y + 6 = 0 \\
        16x - 12y - 12 = 0
      \end{cases}
      \sim
      \begin{cases}
        25x = 6 \\
        12y = 16x - 12
      \end{cases},
    \end{displaymath}

    откъдето получаваме $B_a(6/25, -17/25)$.

    Остава да намерим координатите на $B'$. Имаме $\V{BB_a} = \V{B_a B'}$, откъдето
    \begin{displaymath}
      \begin{cases}
        6/25 = x' - 6/25 \\
        -17/25 + 1 = y' + 17/25
      \end{cases}
      \sim
      \begin{cases}
        x' = 12/25 \\
        y' = -34/25 + 1 = -9/25
      \end{cases}.
    \end{displaymath}

    Получихме $B'(12/25, -9/25)$.

    \item Разстоянието $\Dist(A, a)$ намираме използвайки теорема~\ref{thm:plane-dist}:
    \begin{displaymath}
      \Dist(A, a) = \Abs{\ODist(A, a)} = \frac {\Abs{F_a(A)}} {\Norm {n_a}},
    \end{displaymath}

    където $F_a(x, y) = 3x + 4y + 2$ е лявата част на зададеното общо уравнение на $a$, а $n_a(3, 4)$ е съответният нормален вектор. Директно пресмятаме разстоянието и получаваме
    \begin{displaymath}
      \Dist(A, a) = \frac {\Abs{3 \cdot 1 + 4 \cdot (-2) + 2}} 5 = \frac 3 5.
    \end{displaymath}

    \item Ъглополовящите $u_{1,2}$ на правите $a$ и $b$ се състоят от всички точки $P(x, y)$, за които $\Dist(P, a) = \Dist(P, b)$. Последното условие е еквивалентно на условието $\ODist(P, a) \pm \ODist(P, b) = 0$, откъдето намираме уравненията на ъглополовящите:
    \begin{align*}
      u_{1,2}&: \frac {3x + 4y + 2} 5 \pm \frac {5x - 12y + 1} {13} = 0, \\
      u_{1,2}&: (39x + 52y + 26) \pm (25x - 60y + 5) = 0, \\
      u_1&: 64x - 8y + 31 = 0, \\
      u_2&: 14x + 112y + 21 = 0.
    \end{align*}

    \item Ъгълът между $a$ и $b$ намираме чрез нормалните им вектори $n_a(3, 5)$ и $n_b(5, -12)$:
    \begin{displaymath}
      \angle(a, b) = \arccos \frac {\Abs{\Prod{n_a} {n_b}}} {\Norm{n_a} \Norm{n_b}} = \arccos \Abs {- \frac {33} {65}} = \arccos \frac {33} {65}
    \end{displaymath}

    \item Координатите на пресечната точка $T$ на $a$ и $b$ намираме от системата
    \begin{displaymath}
      \begin{cases}
        3x + 4y + 2 = 0 \mid (\times 3) \\
        5x - 12y + 1 = 0
      \end{cases}
      \sim
      \begin{cases}
        9x + 12y + 6 = 0 \\
        5x - 12y + 1 = 0
      \end{cases}
      \sim
      \begin{cases}
        14x = -7 \\
        12y = 5x + 1
      \end{cases}
    \end{displaymath}

    откъдето получаваме $T(-1/2, -1/8)$.

    Нека $\Point P$ има координати $(x, y)$ спрямо $K$. От условието $P \in t \iff \V{TP} \parallel \V{TB}$ намираме общото уравнение
    \begin{displaymath}
      t: \det
      \begin{pmatrix}
        x + 1/2 & 1/2 \\
        y + 1/8 & -7/8
      \end{pmatrix}
      =
      \frac 1 {16}
      \begin{pmatrix}
        2x + 1 & 1 \\
        8y + 1 & -7
      \end{pmatrix}
      = -14x - 8y - 8 = 0
    \end{displaymath}
    или $t: 7x + 4y + 4 = 0$.

    \item Означаваме $F_a(x, y) = 3x + 4y + 2$. Имаме, че $F_a(A) = F_a(1, -2) = -3 < 0$ и $F_a(B) = F_a(0, -1) = -2 < 0$. Двете точки не лежат върху правата $a$ и освен това $F_a(A)$ и $F_a(B)$ имат еднакви знаци, следователно $\Point A$ и $\Point B$ лежат в една и съща отворена полуравнина относно $a$.
  \end{enumerate}
\end{solution}

\bigskip
\begin{minipage}{0.5\textwidth}
    \begin{exercise}
    В равнината е зададена ортонормирана координатна система $K = Oxy$. Дадени са правите
    \begin{align*}
      &a: x + y = 0 \\
      &b: x - 3y - 2 = 0
    \end{align*}

    Светлинен лъч $l \Ray$ минава през точка $M(2, 2)$ и се отразява от правата $a$. Отразеният лъч $l' \Ray$ е колинеарен с правата $b$.

    Да се намерят уравнения спрямо $K$ на правите $l$ и $l'$, съдържащи съответно падащия лъч $l \Ray$ и отразения лъч $l' \Ray$.
  \end{exercise}
\end{minipage}
\begin{minipage}{0.5\textwidth}
  \begin{figure}[H]
    \begin{center}
      \begin{tikzpicture}[rotate=45]
        \filldraw [black] (2, 2) circle (2pt);
        \node[right] at (2, 2) {$M(2, 2)$};

        \filldraw [black] (0, 0) circle (2pt);
        \node at (1/3, 1/3) {$M_a$};

        \filldraw [black] (-2, -2) circle (2pt);
        \node[right] at (-2, -2) {$M'$};

        \filldraw [black] (1, -1) circle (2pt);
        \node at (1.5, -1.1) {$N$};

        \node[right] at (5/3, 1) {$l$};
        \draw[-{Latex[length=3mm]}] (7/3, 3) -- (1, -1);
        \draw[thick, dotted] (1, -1) -- (-1/3, -5);

        \node[right] at (3, -1/3) {$l'$};
        \draw[-{Latex[length=3mm]}] (1, -1) -- (5, 1/3);
        \draw[thick, dotted] (-3, -7/3) -- (1, -1);

        \node[right] at (-7/10, 1) {$a$};
        \draw[-] (-6/5, 6/5) -- (3, -3);

        \node[right] at (3, 1/3) {$b$};
        \draw[dotted] (-3, -5/3) -- (13/3, 7/9);
      \end{tikzpicture}
    \end{center}
    \caption{Отразен светлинен лъч}\label{fig:plane-light-ray}
  \end{figure}
\end{minipage}
\bigskip

\begin{solution}
  \mbox{}
  \begin{enumerate}
    \item Намираме координатите на ортогоналната проекция $M_a$ на точката $M$ върху $a$, използвайки нормален за $a$ вектор $n_a(1, 1)$.

    От условието $n_a \parallel \V{M M_a}$ намираме ограничението
    \begin{displaymath}
      M_a(x, y): \det
      \begin{pmatrix}
        1 & x - 2 \\
        1 & y - 2
      \end{pmatrix}
      = y - x
      = 0,
    \end{displaymath}

    а от ограничението $M_a \in a$ получаваме, че $\Point M_a$ има координати $(0, 0)$ спрямо $K$.

    \item Намираме координатите на ортогонално симетричната точка $M'(x', y')$ на $M$ относно $a$. Имаме $\V{MM_a} = \V{M_a M'}$, откъдето
    \begin{displaymath}
      \begin{cases}
        2 - 0 = 0 - x' \\
        2 - 0 = 0 - y'
      \end{cases}
      \implies
      x' = y' = -2,
    \end{displaymath}
    т.е. $M'(-2, -2)$.

    \item Намираме уравнението на правата $l'$. Тъй като $l' \parallel b$, правата $l'$ има общо уравнение
    \begin{displaymath}
      l': x - 3y + C = 0,
    \end{displaymath}

    където $C$ е подбрано спрямо условието $l'$ да минава през $\Point M'$, т.е.
    \begin{displaymath}
      C = - 1 \cdot (-2) + 3 \cdot (-2) = -4.
    \end{displaymath}

    И така, $l'$ има общо уравнение $l': x - 3y - 4 = 0$.

    \item Намираме координатите на пресечната точка $N$ на $a$ и $l'$ (а също и $l'$):
    \begin{displaymath}
      N(x, y): \begin{cases}
        x + y = 0 \\
        x - 3y - 4 = 0
      \end{cases}
      \sim
      \begin{cases}
        x = -y \\
        -4y = 4
      \end{cases}
      \sim
      \begin{cases}
        x = 1 \\
        y = -1
      \end{cases},
    \end{displaymath}

    т.е. $N(1, -1)$.

    \item Намираме общо уравнение на $l$. Използваме това, че $l$ минава през $\Point M$ и $\Point N$, т.е. $\Point P(x, y) \in l \iff \V{MP} \parallel \V{MN}$:
    \begin{displaymath}
      l: \det
      \begin{pmatrix}
        x - 2 & 1 - 2\\
        y - 2 & -1 - 2
      \end{pmatrix}
      =
      -3x + y + 4
      =
      0.
    \end{displaymath}
  \end{enumerate}
\end{solution}

\subsection{Равнини в пространството}

\begin{exercise}
  В пространството е зададена ортонормирана координатна система $K = Oxyz$. Спрямо нея точките $A$, $B$, $C$ и $D$ имат координати
  \begin{align*}
    &A(3, 1, 4) &C(1, 2, -1) \\
    &B(2, 1, 3) &D(0, -3, 2),
  \end{align*}

  а равнината $\beta$ има общо уравнение
  \begin{displaymath}
    \beta: x + y - z + 1 = 0.
  \end{displaymath}

  Да се намерят:
  \begin{enumerate}[label=\alph*)]
    \item Общо уравнение на равнината $\alpha$, минаваща през точките $A$, $B$ и $C$
    \item Параметрични уравнения на права $h$, минаваща през $\Point D$ и ортогонална на равнината $\alpha$
    \item Координатите на ортогонално симетричната на $\Point D$ спрямо $\alpha$ точка $\Point D'$
    \item Общо уравнение на равнината $\gamma$, минаваща през $\Point D$ и успоредна на $\beta$
    \item Координатите на някой вектор $v$, компланарен с $\alpha$ и $\beta$
    \item Параметрични уравнения на пресечницата $m$ на $\alpha$ и $\beta$
    \item Параметрични уравнения на права $l$, така че светлинния лъч $l\Ray$ през $\Point D$ след отразяването си от $\alpha$ (озн. отразения лъч с $l'\Ray$) пресича $\beta$ под прав ъгъл
    \item Общо уравнение на равнината $\pi_1$, минаваща през $A$ и ортогонална на правата $m$
    \item Общо уравнение на равнината $\pi_2$, съдържаща правата $l'$ и успоредна на правата $m$
    \item Общо уравнение на равнината $\pi_3$, съдържаща правата $l'$ и ортогонална на равнината $\alpha$
  \end{enumerate}
\end{exercise}

\begin{solution}
  \begin{enumerate}[label=\alph*)]
    \item Нека $\Point P(x, y, z) \in \alpha$. Тогава $\V{AP} \parallel \V{AB}(-1, 0, -1) \parallel \V{AC}(-2, 1, -5)$, което условие ни позволява да намерим общо уравнение на $\alpha$:
    \begin{multline*}
      \alpha: \det
      \begin{pmatrix}
        x - 3 & -1 & -2 \\
        y - 1 & 0 & 1 \\
        z - 4 & -1 & -5
      \end{pmatrix}
      =
      -1(z-4) + (-2)(y-1)(-1) - (x-3)(-1) - (-1)(y-1)(-5)
      = \\ =
      -z + 4 + 2y - 2 + x - 3 - 5y + 5
      =
      \boxed{x - 3y - z + 4 = 0}.
    \end{multline*}

    \item Правата $h$ е успоредна на нормалния за $\alpha$ вектор на $n_\alpha(1, -3, -1)$, следователно тя има векторно параметрично уравнение
    \begin{displaymath}
      h: \V{OD} + \lambda n_\alpha, \lambda \in \R
    \end{displaymath}
    и, съответно, скаларно параметрично уравнение
    \begin{displaymath}
      h: \begin{cases}
        x = \lambda \\
        y = -3 - 3\lambda \\
        z = 2 - \lambda
      \end{cases},
      \lambda \in \R.
    \end{displaymath}

    \item За да намерим ортогонално симетричната на $\Point D$ спрямо $\alpha$ точка $\Point D'(x', y', z')$, първо ще намерим пресечната точка $\Point D_\alpha$ на правата $h$ и равнината $\alpha$.

    Намираме стойността на параметъра $\lambda$ за уравнението на $h$, за която $h$ се пресича $\alpha$:
    \begin{align*}
      \lambda - 3(-3 - 3\lambda) - (2 - \lambda) + 4 &= 0 \\
      \lambda + 9 + 9\lambda - 2 + \lambda + 4 &= 0 \\
      11 \lambda &= -11 \\
      \lambda &= -1,
    \end{align*}

    откъдето намираме координатите $(-1, 0, 3)$ на $\Point D_\alpha$.

    След това, развиваме очевидното равенство $\V{D_\alpha D'} = \V{DD_\alpha}$ покоординатно:
    \begin{displaymath}
      \begin{cases}
        x' + 1 = -1 \\
        y' - 0 = 3 \\
        z' - 3 = 1
      \end{cases}
      \sim
      \begin{cases}
        x' = -2 \\
        y' = 3 \\
        z' = 4,
      \end{cases}
    \end{displaymath}
    т.е. $D'(-2, 3, 4)$.

    \item Тъй като равнината $\gamma$ е успоредна на $\beta$, тя има общо уравнение
    \begin{displaymath}
      \gamma: x + y - z + C = 0,
    \end{displaymath}

    където $C$ е подбрано спрямо условието $\gamma$ да минава през $\Point D$, т.е.
    \begin{displaymath}
      C = -(0 + (- 3) - 2) = 5.
    \end{displaymath}

    И така, $\gamma: x + y - z + 5 = 0$.

    \item Търсим едновременни решения на известните общи уравнения на равнините $\alpha$ и $\beta$:
    \begin{displaymath}
      \begin{cases}
        \alpha: x - 3y - z + 4 = 0 \\
        \beta: x + y - z + 1 = 0
      \end{cases}
      \sim
      \begin{cases}
        y = 3 / 4 \\
        z = x + 7 / 4.
      \end{cases}
    \end{displaymath}

    Очевидно точките $V_0\left(0, \frac 3 4, \frac 7 4 \right)$ и $V_1\left( 1, \frac 3 4, 1 + \frac 7 4 \right)$ удовлетворяват горната система, следователно векторът $v = \V{V_0 V_1}$ с координати $(1, 0, 1)$ е колинеарен и с двете равнини.

    \item Пресечницата $m$ на $\alpha$ и $\beta$ има векторно параметрично уравнение
    \begin{displaymath}
      m: \V{OV_0} + \mu v, \mu \in \R
    \end{displaymath}
    и, съответно, скаларно параметрично уравнение
    \begin{displaymath}
      m: \begin{cases}
        x = \mu \\
        y = 3/4 \\
        z = 7/4 + \mu
      \end{cases},
      \mu \in \R.
    \end{displaymath}

    \item Тъй като вече разполагаме с координатите на ортогонално симетричната на $D$ относно $\alpha$ точка $D'$ и нормалният за $\alpha$ вектор $n_\beta(1, 1, -1)$, можем да намерим векторно параметрично уравнения на правата $l'$:
    \begin{displaymath}
      l': \V{OD'} + \nu n_\beta, \nu \in \R
    \end{displaymath}
    и, съответно, скаларното параметрично уравнение
    \begin{displaymath}
      l': \begin{cases}
        x = -2 + \nu \\
        y = 3 + \nu \\
        z = 4 - \nu
      \end{cases},
      \nu \in \R.
    \end{displaymath}

    Сега намираме стойността на параметъра $\nu$, за която $l'$ пресича равнината $\alpha$:
    \begin{align*}
      (-2 + \nu) - 3(3 + \nu) - (4 - \nu) + 4 &= 0 \\
      -2 + \nu - 9 - \nu - 4 + \nu + 4 &= 0 \\
      \nu &= 11
    \end{align*}
    т.е. пресечната точка $N$ на $l'$ и $\alpha$ има координати $N(9, 14, -7)$.

    Сега намираме векторно параметрично уравнения на правата $l$, минаваща през $\Point D$ и $\Point N$:
    \begin{displaymath}
      l: \V{OD} + \xi \V{DN}, \xi \in \R
    \end{displaymath}
    и, съответно, скаларното параметрично уравнение
    \begin{displaymath}
      l: \begin{cases}
        x = 9\xi \\
        y = -3 + 17\xi \\
        z = 2 - 9\xi
      \end{cases},
      \xi \in \R.
    \end{displaymath}

    \item Първо намираме два нормални за правата $m$ вектора. Нека $u(x, y, z)$ е произволен вектор. Тъй като $v$ е направляващ за $m$, $u$ е нормален за $m$ само ако
    \begin{displaymath}
      \Prod u v = x + z = 0.
    \end{displaymath}

    От горното условие виждаме, че два нормални за $m$ вектора са $u_1(0, 1, 0)$ и $u_2(-1, 0, 1)$.

    Нека $\Point P$ има спрямо $K$ координати $(x, y, z)$. Тогава $\Point P \in \pi_1$ точно тогава, когато векторът $\V{AP}$ е колинеарен с $u_1$ и $u_2$, т.е.
    \begin{displaymath}
      \pi_1: \det
      \begin{pmatrix}
        x - 3 & 0 & -1 \\
        y - 1 & 1 & 0 \\
        z - 4 & 0 & 1
      \end{pmatrix}
      = (x - 3) - (-1)(z - 4) = \boxed{x + z - 7 = 0}.
    \end{displaymath}

    \item Имаме, че $\Point D'(-2, 3, 4) \in l'$ и векторът $n_\beta(1, 1, -1)'$ е направляващ за $l'$, а $v(1, 0, 1)$ е направляващ за $m$. Нека $\Point P$ има спрямо $K$ координати $(x, y, z)$. Тогава $\Point P \in \pi_2$ точно тогава, когато са колинеарни векторите $\V{D'P}$, $n_\beta$ и $v$, т.е.
    \begin{displaymath}
      \pi_2: \det
      \begin{pmatrix}
        x + 2 & 1  & 1 \\
        y - 3 & 1  & 0 \\
        z - 4 & -1 & 1
      \end{pmatrix}
      = (x + 2) + (y - 3)(-1) - (z - 4) - (y - 3) = \boxed{x - 2y - z + 12 = 0}.
    \end{displaymath}

    \item Аналогично на $\pi_2$, имаме, че $\Point P \in \pi_3$ точно тогава, когато са колинеарни векторите $\V{D'P}$, $n_\beta$ и $n_\alpha(1, -3, -1)$
    \begin{multline*}
      \pi_3: \det
      \begin{pmatrix}
        x + 2 & 1  & 1 \\
        y - 3 & 1  & -3 \\
        z - 4 & -1 & -1
      \end{pmatrix}
      = \\ =
      (x + 2)(-1) + (-3)(z - 4) + (y - 3)(-1) - (z - 4) - (y - 3)(-1) - (x + 2)(-3)(-1)
      = \\ =
      (-4)(x + 2) + (-4)(z - 4) = 0
      \text{ или }
      \boxed{\pi_3: x + z - 2 = 0}.
    \end{multline*}
  \end{enumerate}
\end{solution}

\subsection{Прави в пространството}

\begin{exercise}
  В пространството е зададена ортонормирана координатна система $K = Oxyz$. Спрямо нея кръстосаните прави $g$ и $h$ имат скаларни параметрични уравнения
  \begin{align*}
    g: \begin{cases}
      x = 1 + 2\lambda \\
      y = 4 + 4\lambda \\
      z = 4 + \lambda
    \end{cases},
    \lambda \in \R
    &&
    h: \begin{cases}
      x = -1 \\
      y = -1 - 5\mu \\
      z = 1 + 3\mu
    \end{cases},
    \mu \in \R.
  \end{align*}

  Да се намери (скаларно) параметрично уравнение на трансверзалата $t$ на $g$ и $h$, за която:
  \begin{enumerate}[label=\alph*)]
    \item Точката $A(1, 1, 1)$ лежи върху $t$
    \item $t$ лежи в равнината $\alpha: 2x + y - 3z + 6 = 0$
    \item $t$ е успоредна на правата
    \begin{displaymath}
      a: \begin{cases}
        x + 5y + 4z - 3 = 0 \\
        2x - 5y - 4z + 1 = 0
      \end{cases}
    \end{displaymath}
  \end{enumerate}
\end{exercise}

\begin{solution}
  Тъй като правата $t$ е трансверзала на $g$ и $h$, тя има по една обща точка с двете прави. Да означим тези точки с $G$ и $H$, т.е.
  \begin{align*}
    t \cap g = \{ G(x_g, y_g, z_g) \}
    &&
    t \cap h = \{ H(x_h, y_h, z_h) \}.
  \end{align*}

  Тъй като $G \in g$ и $H \in h$, то техните координати удовлетворяват уравненията на $g$ и $h$ за някои стойности на параметрите $\lambda$ и $\mu$, т.е.
  \begin{align*}
    &G(1 + 2\lambda_G, 4 + 4\lambda_G, 4 + 4\lambda_G) \\
    &H(-1, -1 -5\mu_H, 1 + 3\mu_H).
  \end{align*}

  Тогава векторът $\V{HG}$ има координати $\V{HG}(2 + 2\lambda_G, 5 + 4\lambda_G + 5\mu_H, 3 + \lambda_G - 3\mu_H)$.

  \begin{enumerate}[label=\alph*)]
    \item За да принадлежи точката $A$ на трансверзалата $t$, искаме векторите $\V{AG}(2\lambda_G, 3 + 4\lambda_G, 3 + \lambda_G)$ и $\V{AH}(-2, -2 - 5\mu_H, 3\mu_H)$ да бъдат колинеарни, т.е. $\exists k \in \R \setminus \{ 0 \}: \V{AG} = k\V{AH}$. Разписваме това уравнение покоординатно:
    \begin{displaymath}
      \begin{cases}
        2\lambda_G = -2k \mid (\times 1 / 2) \\
        3 + 4\lambda_G = -2k - 5k\mu_H \mid \text{ изваждаме третото уравнение} \\
        3 + \lambda_G = 3k\mu_H
      \end{cases}
      \sim
      \begin{cases}
        \lambda_G = -k \\
        3\lambda_G = -2k - 8k\mu_H \\
        3 + \lambda_G = 3k\mu_H
      \end{cases}.
    \end{displaymath}

    От второто уравнение получаваме $-k = -8k\mu_H \implies \mu_H = 1 / 8 \implies H(-1, -13/8, 11/8)$.

    Правата $t$ има векторно параметрично уравнение
    \begin{displaymath}
      t: \V{OA} + 8 \nu \V{AH}, \nu \in \R
    \end{displaymath}
    и, съответно, скаларно параметрично уравнение
    \begin{displaymath}
      t: \begin{cases}
        x = 1 - 16\nu \\
        y = 1 - 21 \nu \\
        z = 1 + 3 \nu
      \end{cases},
      \nu \in \R.
    \end{displaymath}

    \item Намираме $\lambda_G$ и $\mu_H$ като директно заместваме координатите на $G$ и $H$ в уравнението на $\alpha$:
    \begin{align*}
      2(1 + 2\lambda_G) + (4 + 4\lambda_G) - 3(4 + \lambda_G) + 6 &= 0 \\
      2 + 4\lambda_G + 4 + 4\lambda_G - 12 + 3\lambda_G + 6 &= 0 \\
      5\lambda_G &= 0,
    \end{align*}
    следователно $\lambda_G = 0$ и $\Point G$ има координати $G(1, 4, 4)$.
    \begin{align*}
      2(-1) + (-1 - 5\mu_H) - 3(1 + 3\mu_H) + 6 &= 0 \\
      -2 - 1 - 5\mu_H - 3 - 9\mu_H + 6 &= 0 \\
      -14\mu_H &= 0,
    \end{align*}
    следователно $\mu_H = 0$ и $\Point H$ има координати $H(-1, -1, 1)$.

    Правата $t$ има векторно параметрично уравнение
    \begin{displaymath}
      t: \V{OG} + \nu \V{GH}, \nu \in \R
    \end{displaymath}
    и, съответно, скаларно параметрично уравнение
    \begin{displaymath}
      t: \begin{cases}
        x = 1 - 2\nu \\
        y = 4 - 5\nu \\
        z = 4 - 3\nu
      \end{cases},
      \nu \in \R.
    \end{displaymath}

    \item Първо намираме скаларно параметрично уравнение на правата $a$. Събирайки двете уравнения, получаваме
    \begin{displaymath}
      \begin{cases}
        x + 5y + 4z - 3 = 0 \\
        2x - 5y - 4z + 1 = 0 \mid \text{ прибавяме първото уравнение}
      \end{cases}
      \sim
      \begin{cases}
        4z = 3 - x - 5y = 7 / 3 - 5y \\
        x = 2 / 3
      \end{cases}.
    \end{displaymath}

    Параметризираме горната система чрез $\xi \in \R$, полагайки $y = 4\xi$, и получаваме
    \begin{displaymath}
      a: \begin{cases}
        x = 2 / 3 \\
        y = 4\xi \\
        z = 7/{12} - 5\xi
      \end{cases},
      \xi \in \R.
    \end{displaymath}

    Означаваме направляващия вектор от горното уравнение чрез $v_a(0, 4, -5)$. За да бъдат колинеарни правите $t$ и $a$ е достатъчно двойка техни направляващи вектори да бъдат колинеарни, т.е. $\V{GH} \parallel v_a \iff \exists k \in \R \setminus \{ 0 \} : \V{GH} = k v_a$. Разписвайки това уравнение само за първата координата, получаваме
    \begin{displaymath}
      2 + 2\lambda_G = 0 \cdot k \implies \lambda_G = -1 \implies G(-1, 0, 3).
    \end{displaymath}

    Разполагайки с точка $G \in t$ и направляващ вектор $v_a \parallel t$, за трансверзалата $t$ получаваме векторно параметрично уравнение
    \begin{displaymath}
      t: \V{OG} + \nu v_a, \nu \in \R
    \end{displaymath}
    и, съответно, скаларно параметрично уравнение
    \begin{displaymath}
      t: \begin{cases}
        x = -1 \\
        y = 4\nu \\
        z = 3 - 5\nu
      \end{cases},
      \nu \in \R.
    \end{displaymath}
  \end{enumerate}
\end{solution}

\subsection{Канонизиране на криви от втора степен}

\begin{exercise}
  Спрямо ортонормирана координатна система $K = Oxy$ е зададена кривата
  \begin{displaymath}
    c: 4x^2 - 4xy + y^2 + 2x - 16y - 8 = 0.
  \end{displaymath}

  Да се намери метрично канонично уравнение на $c$ и да намерят координатите на фокусите на $c$ спрямо $K$.
\end{exercise}

\begin{solution}
  \begin{enumerate}
    \item Търсим ортонормирана координатна система $K'$, в която уравнението на кривата $c$ да няма смесен квадратен член $xy$. Означаваме матрицата на квадратичната форма $4x^2 - 4xy + y^2$ с
    \begin{displaymath}
      A = \begin{pmatrix}
        4 & -2 \\
        -2 & 1
      \end{pmatrix}.
    \end{displaymath}

    Търсим собствените вектори на $A$:
    \begin{displaymath}
      \begin{pmatrix}
        4 & -2 \\
        -2 & 1
      \end{pmatrix}
      \begin{pmatrix}
        x \\ y
      \end{pmatrix}
      =
      \lambda
      \begin{pmatrix}
        x \\ y
      \end{pmatrix},
    \end{displaymath}
    което уравнение е еквивалентно на системата
    \begin{equation}
      \label{ex:canon:eigen}
      \begin{cases}
        4x - 2y = \lambda x \\
        -2x + y = \lambda y
      \end{cases}
      \sim
      \begin{cases}
        -2y = (\lambda - 4) x \\
        (1 - \lambda) y = 2x
      \end{cases}
      \sim
      \begin{cases}
        y = \frac {4 - \lambda} 2 x \\
        y = \frac 2 {1 - \lambda} x.
      \end{cases}
    \end{equation}

    За да получим нормирани собствени вектори, искаме
    \begin{align*}
      x^2 + y^2 &= 1 \\
      \left(1 +  \frac 4 {{(1 - \lambda)}^2} \right) x^2 &= 1 \\
      \frac {\lambda^2 - 2\lambda + 5} {{(1 - \lambda)}^2} x^2 &= 1.
    \end{align*}
    За определеност взимаме само една двойка корени
    \begin{equation}
      \label{ex:canon:normed_eigen}
      x = \frac {1 - \lambda} {\sqrt{\lambda^2 - 2\lambda + 5}}
      \text{ и }
      y = \frac 2 {\sqrt{\lambda^2 - 2\lambda + 5}}
    \end{equation}

    Изваждайки двете уравнения в~\ref{ex:canon:eigen}, стигаме до характеристичния полином на $A$:
    \begin{displaymath}
      \frac {4 - \lambda} 2 - \frac 2 {1 - \lambda} = 0 \iff 4 - \lambda - 4\lambda + \lambda^2 - 4 = \lambda^2 - 5\lambda = 0,
    \end{displaymath}
    чиито корени са $\lambda_1 = 0$ и $\lambda_1 = 5$, а замествайки в~\ref{ex:canon:normed_eigen}, директно получаваме собствените вектори
    \begin{align*}
      v_1 \left(\frac 1 {\sqrt 5}, \frac 2 {\sqrt 5} \right)
      &&
      v_2 \left(\frac {-2} {\sqrt 5}, \frac 1 {\sqrt 5} \right).
    \end{align*}

    Желаната ротация има вида
    \begin{displaymath}
      R: \begin{cases}
        x = \frac 1 {\sqrt 5} (x' - 2y') \\
        y = \frac 1 {\sqrt 5} (2x' + y').
      \end{cases}
    \end{displaymath}

    Спрямо новата координатна система $K'$ кривата $c$ има вида
    \begin{displaymath}
      c: 5y'^2 + \frac 2 {\sqrt 5} (x' - 2y') - \frac {16} {\sqrt 5} (2x' + y') - 8 = 5y'^2 - \frac {30} {\sqrt 5} x' - \frac {20} {\sqrt 5} y' - 8 = 0,
    \end{displaymath}
    което можем да опростим до
    \begin{displaymath}
      c: 5 \sqrt 5 y'^2 - 30 x' - 20 y' - 8 \sqrt 5 = 0,
    \end{displaymath}

    \item Имаме уравнения на парабола, но то не е канонично. Търсим афинна координатна система $K''$, в която параболата да е центрирана, т.е. търсим транслация
    \begin{displaymath}
      T: \begin{cases}
        x' = x'' + a \\
        y' = y'' + b,
      \end{cases}
    \end{displaymath}
    така че в новата координатна система уравнението да има вида $c: y'' = 2px''$ за някое число $p$.

    Правим транслацията и след това намираме подходящи стойности за параметрите $a$ и $b$:
    \begin{align*}
      5 \sqrt 5 y'^2 - 30 x' - 20 y' - 8 \sqrt 5 &= 0 \\
      5 \sqrt 5 y''^2 + 10 \sqrt 5 y'' b + 5 \sqrt 5 b^2 - 30 x'' - 30a - 20 y'' - 20b - 8 \sqrt 5 &= 0 \\
      5 \sqrt 5 y''^2 + (10 \sqrt 5 b - 20) y'' = 30 x'' + (- 5 \sqrt 5 b^2 + 30a + 20b + 8 \sqrt 5).
    \end{align*}

    Приравняваме на нула коефициента пред $y''$ и свободния коефициент:
    \begin{displaymath}
      10 \sqrt 5 b - 20 = 0 \implies b = \frac 2 {\sqrt 5}
    \end{displaymath}
    \begin{displaymath}
      a
      =
      \frac {5 \sqrt 5 b^2 - 20 b - 8 \sqrt 5} {30}
      =
      \frac {4 \sqrt 5 - \frac {40} {\sqrt 5} - 8 \sqrt 5} {30}
      =
      \frac {20 - 40 - 40} {30 \sqrt 5}
      =
      \frac {-60} {30 \sqrt 5}
      =
      \frac {-2} {\sqrt 5}.
    \end{displaymath}

    Тогава за композираната смяна на координатната система от $K''$ към $K$ получаваме
    \begin{displaymath}
      R \circ T: \begin{cases}
        x = \frac 1 {\sqrt 5} \left( x'' + a - 2y'' - 2b \right) \\
        y = \frac 1 {\sqrt 5} \left( 2x'' + 2a + y'' + b \right)
      \end{cases}
      \sim
      \begin{cases}
        x = \frac 1 {\sqrt 5} \left( x'' - 2y'' - \frac 6 {\sqrt 5} \right) \\
        y = \frac 1 {\sqrt 5} \left( 2x'' + y'' - \frac 2 {\sqrt 5} \right)
      \end{cases}
    \end{displaymath}

    и спрямо $K''$ параболата има уравнение
    \begin{displaymath}
      c: 5 \sqrt 5 y''^2 = 30x''
      \iff
      c: y''^2 = 2 \cdot \frac 3 {\sqrt 5}x''
    \end{displaymath}

    \item Фокусът на параболата $c$ има спрямо $K''$ координати $F \left(\frac 3 {2 \sqrt 5}, 0 \right)$. Спрямо $K$ фокусът има координати
    \begin{displaymath}
      R \circ T: \begin{cases}
        x = \frac 1 {\sqrt 5} \left( \frac 3 {2 \sqrt 5} - \frac 6 {\sqrt 5} \right) = - \frac 9 {10} \\
        y = \frac 1 {\sqrt 5} \left( \frac 3 {\sqrt 5} - \frac 2 {\sqrt 5} \right) = \frac 1 5
      \end{cases},
    \end{displaymath}

    т.е. $F \left(-\frac 9 {10}, \frac 1 5 \right)$.
  \end{enumerate}
\end{solution}

\printbibliography

\end{document}
