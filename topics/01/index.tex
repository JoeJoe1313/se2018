% arara: pdflatex: { shell: true }
% arara: biber
% arara: pdflatex: { shell: true }

\documentclass[numbers=endperiod]{scrartcl}

% Base packages
\usepackage[T2A]{fontenc}
\usepackage[utf8]{inputenc}
\usepackage[bulgarian]{babel}
\usepackage[pdfencoding=unicode]{hyperref}
\usepackage{biblatex}
\usepackage[style=german]{csquotes}

% Base math packages
\usepackage{amsmath}
\usepackage{amssymb}
\usepackage{amsthm}
\usepackage{mathtools}

% Misc packages
\usepackage{enumitem}

% Custom packages
\usepackage{../../common/macros}
\usepackage{../../common/theorems}

% Bibliography
\addbibresource{./references.bib}

% Document
\title{Тема 1}
\subtitle{Уравнение на права и равнина. Формули за разстояния и ъгли. Криви от втора степен.}
\author{Янис Василев, \Email{ianis@ivasilev.net}}
\date{24 март 2019}

\begin{document}

\maketitle

\section{Анотация (от \cite{Syllabus})}

\subsection{Теория}

\begin{enumerate}
    \item Прави в равнината и пространството
    \begin{enumerate}
        \item Векторни и параметрични (скаларни) уравнения на права и равнина
        \item Общо уравнение на права в равнината
        \item Декартово уравнение
        \item Взаимно положение на две прави
        \item Нормално уравнение на права
        \item Разстояние от точка до права
        \item Ъгъл между прави
    \end{enumerate}

    \item Равнини в пространството
    \begin{enumerate}
        \item Общо уравнение на равнина
        \item Взаимно положение на две равнини
        \item Нормално уравнение на равнина
        \item Разстояние от точка до равнина
    \end{enumerate}

    \item Криви от втора степен
    \begin{enumerate}
        \item Уравнение на окръжност
        \item Канонични уравнения на елипса, хипербола и парабола
        \item Фокални свойства на елипса, хипербола и парабола
    \end{enumerate}
\end{enumerate}

\subsection{Задачи}

Не е даден списък с възможни задачи, затова съм включил всички задачи, свързани с материала, които са давани на упражнения.

\section{Теория (от \cite{Notes}}

\section{Примерни задачи (от \cite{Notes})}

\subsection{Права в равнината}

\begin{exercise}
    Точките $A$ и $B$ и правите $a$ и $b$ имат спрямо ортонормирана координатна система $K$ координати $A(1, -2)$, $B(0, -1)$ и общи уравнения
    \begin{align*}
        &a: 3x + 4y + 2 = 0, \\
        &b: 5x - 12y + 1 = 0.
    \end{align*}

    Да се намерят:
    \begin{enumerate}[label=\alph*)]
        \item Общо уравнение на правата $l$ през $\Point A$, успоредна на $a$
        \item Общо уравнение на правата $m$ през $\Point B$, перпендикулярна на $b$
        \item Общо уравнение на правата $AB$
        \item Координатите на $\Point B'$, която е ортогонално симетрична на $\Point B$ относно правата $a$
        \item Разстоянието от $\Point A$ до $a$
        \item Общо уравнение на ъглополовящата на правите $a$ и $b$
        \item Ъгълът между правите $a$ и $b$
        \item Общо уравнение на правата $t$ през $\Point B$ и $\Point T = a \cap b$
        \item Да се определи положението на $\Point A$ и $\Point B$ спрямо правата $a$
    \end{enumerate}
\end{exercise}

\begin{solution}
    \begin{enumerate}[label=\alph*)]
        \item Правата $l$ е успоредна на $a$, следователно тя има общо уравнение от вида
        \begin{displaymath}
            l: 3x + 4y + C = 0,
        \end{displaymath}

        където $C$ е подбрано спрямо условието $l$ да минава през $\Point A$, т.е.
        \begin{displaymath}
            C = - 3 \cdot 1 - 4 \cdot (-2) = 5.
        \end{displaymath}

        И така, $l$ има общо уравнение $l: 3x + 4y + 5 = 0$.

        \item Правата $m$ е перпендикулярна на $b$, следователно нормалните вектори на $b$ (например $N_b(5, -12)$) са колинеарни с $m$. Нека $\Point P$ има координати $(x, y)$ спрямо $K$. От условието $P \in m \iff \V{BP} \parallel N_b$ намираме общото уравнение
        \begin{displaymath}
            m: \det
            \begin{pmatrix}
                x & 5 \\
                y + 1 & -12
            \end{pmatrix}
            = 0
            \text{ или } m: 12x + 5y + 5 = 0.
        \end{displaymath}

        \item Нека $\Point P$ има координати $(x, y)$ спрямо $K$. От условието $P \in AB \iff \V{BP} \parallel \V{BA}$ намираме общото уравнение
        \begin{displaymath}
            AB: \det
            \begin{pmatrix}
                x & 1 \\
                y + 1 & -1
            \end{pmatrix}
            = 0
            \text{ или } AB: x + y + 1 = 0.
        \end{displaymath}

        \item Нека $B'$ има спрямо $K$ координати $B'(x', y')$. Първо намираме правата $BB'$ от условието $BB' \perp a$, което е еквивалентно на $BB' \parallel N_a(3, 4)$. Нека $\Point P$ има координати $(x, y)$ спрямо $K$. От условието $P \in BB' \iff \V{BP} \parallel \V{N_a}$ намираме общото уравнение
        \begin{displaymath}
            BB': \det
            \begin{pmatrix}
                x & 3 \\
                y + 1 & 4
            \end{pmatrix}
            = 0
            \text{ или } BB': 4x - 3y - 3 = 0.
        \end{displaymath}

        Координатите на пресечната точка $B'' = a \cap BB'$ намираме от системата
        \begin{displaymath}
            \begin{cases}
                3x + 4y + 2 = 0 \;|\; (\;\cdot\; 3) \\
                4x - 3y - 3 = 0 \;|\; (\;\cdot\; 4)
            \end{cases}
            \sim
            \begin{cases}
                9x + 12y + 6 = 0 \\
                16x - 12y - 12 = 0
            \end{cases}
            \sim
            \begin{cases}
                25x = 6 \\
                12y = 16x - 12
            \end{cases}
        \end{displaymath}

        откъдето получаваме $B''(6/25, -17/25)$.

        Остава да намерим координатите на $B'$. Имаме $BB'' = B''B'$, откъдето
        \begin{displaymath}
            \begin{cases}
                6/25 = x' - 6/25 \\
                -17/25 + 1 = y' + 17/25
            \end{cases}
            \sim
            \begin{cases}
                x' = 12/25 \\
                y' = -34/25 + 1 = -9/25
            \end{cases}.
        \end{displaymath}

        Получихме $B'(12/25, -9/25)$.

        \item Разстоянието $\Dist(A, a)$ намираме от ориентираното разстояние $\ODist(A, a)$:
        \begin{displaymath}
            \Dist(A, a) = \Abs{\ODist(A, a)} = \frac {\Abs{F_a(A)}} {\Norm {N_a}},
        \end{displaymath}

        където $F_a(x, y) = 3x + 4y + 2$ е лявата част на зададеното общо уравнение на $a$, а $N_a(3, 4)$ е съответният нормален вектор. Директно пресмятаме разстоянието и получаваме
        \begin{displaymath}
            \Dist(A, a) = \frac {\Abs{3 \cdot 1 + 4 \cdot (-2) + 2}} 5 = \frac 3 5.
        \end{displaymath}

        \item Ъглополовящите $u_{1,2}$ на правите $a$ и $b$ се състоят от всички точки $P(x, y)$, за които $\Dist(P, a) = \Dist(P, b)$. Последното условие е еквивалентно на условието $\ODist(P, a) \pm \ODist(P, b) = 0$, откъдето намираме уравненията на ъглополовящите:
        \begin{align*}
            u_{1,2}&: \frac {3x + 4y + 2} 5 \pm \frac {5x - 12y + 1} {13} = 0, \\
            u_{1,2}&: (39x + 52y + 26) \pm (25x - 60y + 5) = 0, \\
            u_1&: 64x - 8y + 31 = 0, \\
            u_2&: 14x + 112y + 21 = 0.
        \end{align*}

        \item Ъгълът между $a$ и $b$ намираме чрез нормалните им вектори $N_a(3, 5)$ и $N_b(5, -12)$:
        \begin{displaymath}
            \angle(a, b) = \arccos \frac {\Abs{\Prod{N_a} {N_b}}} {\Norm{N_a} \Norm{N_b}} = \arccos \Abs {- \frac {33} {65}} = \arccos \frac {33} {65}
        \end{displaymath}

        \item Координатите на пресечната точка $T = a \cap b$ намираме от системата
        \begin{displaymath}
            \begin{cases}
                3x + 4y + 2 = 0 \;|\; (\;\cdot\; 3) \\
                5x - 12y + 1 = 0
            \end{cases}
            \sim
            \begin{cases}
                9x + 12y + 6 = 0 \\
                5x - 12y + 1 = 0
            \end{cases}
            \sim
            \begin{cases}
                14x = -7 \\
                12y = 5x + 1
            \end{cases}
        \end{displaymath}

        откъдето получаваме $T(-1/2, -1/8)$.

        Нека $\Point P$ има координати $(x, y)$ спрямо $K$. От условието $P \in t \iff \V{TP} \parallel \V{TB}$ намираме общото уравнение
        \begin{displaymath}
            t: \det
            \begin{pmatrix}
                x + 1/2 & 1/2 \\
                y + 1/8 & -7/8
            \end{pmatrix}
            =
            \frac 1 {16}
            \begin{pmatrix}
                2x + 1 & 1 \\
                8y + 1 & -7
            \end{pmatrix}
            = -14x - 8y - 8 = 0
        \end{displaymath}
        или $t: 7x + 4y + 4 = 0$.

        \item Означаваме $F_a(x, y) = 3x + 4y + 2$. Имаме, че $F_a(A) = F_a(1, -2) = -3 < 0$ и $F_a(B) = F_a(0, -1) = -2 < 0$. Двете точки не лежат върху правата $a$ и освен това $F_a(A)$ и $F_a(B)$ имат еднакви знаци, следователно $\Point A$ и $\Point B$ лежат в една и съща отворена полуравнина относно $a$.
    \end{enumerate}
\end{solution}

\subsection{Равнина в пространството}

\subsection{Права в пространството}

\subsection{Канонизиране на крива от втора степен}

\printbibliography

\end{document}
