% arara: pdflatex: { shell: true }
% arara: biber
% arara: pdflatex: { shell: true }

\documentclass[numbers=endperiod, DIV=15]{scrartcl}

% Base packages
\usepackage[T2A]{fontenc}
\usepackage[utf8]{inputenc}
\usepackage[bulgarian]{babel}
\usepackage[pdfencoding=unicode]{hyperref}
\usepackage{biblatex}
\usepackage[style=german]{csquotes}

% Base math packages
\usepackage{amsmath}
\usepackage{amssymb}
\usepackage{amsthm}
\usepackage{mathtools}

% Misc packages
\usepackage{enumitem}
\usepackage{tikz}

% Custom packages
\usepackage{../../common/macros}
\usepackage{../../common/theorems}

% Bibliography
\addbibresource{./references.bib}

% Document
\title{Тема 1}
% \subtitle{Уравнение на права и равнина. Формули за разстояния и ъгли. Криви от втора степен.}
\author{Янис Василев, \Email{ianis@ivasilev.net}}
\date{24 май 2019}

\usetikzlibrary{arrows.meta}

\begin{document}

\maketitle

\section{Анотация (от~\cite{Syllabus})}

\subsection{Теория}

\begin{enumerate}
  \item Прави в равнината и пространството
  \begin{enumerate}
    \item Векторни и параметрични (скаларни) уравнения на права и равнина
    \item Общо уравнение на права в равнината
    \item Декартово уравнение
    \item Взаимно положение на две прави
    \item Нормално уравнение на права
    \item Разстояние от точка до права
    \item Ъгъл между прави
  \end{enumerate}

  \item Равнини в пространството
  \begin{enumerate}
    \item Общо уравнение на равнина
    \item Взаимно положение на две равнини
    \item Нормално уравнение на равнина
    \item Разстояние от точка до равнина
  \end{enumerate}

  \item Криви от втора степен
  \begin{enumerate}
    \item Уравнение на окръжност
    \item Канонични уравнения на елипса, хипербола и парабола
    \item Фокални свойства на елипса, хипербола и парабола
  \end{enumerate}
\end{enumerate}

\subsection{Задачи}

Не е даден списък с възможни задачи, затова съм включил всички задачи, свързани с материала, които са давани на упражнения.

\section{Теория (от~\cite{Notes})}

\section{Примерни задачи (от~\cite{Notes})}

\subsection{Права в равнината}

\begin{exercise}
  Точките $A$ и $B$ и правите $a$ и $b$ имат спрямо ортонормирана координатна система $K = Oxy$ координати $A(1, -2)$, $B(0, -1)$ и общи уравнения
  \begin{align*}
    &a: 3x + 4y + 2 = 0, \\
    &b: 5x - 12y + 1 = 0.
  \end{align*}

  Да се намерят:
  \begin{enumerate}[label=\alph*)]
    \item Общо уравнение на правата $l$ през $\Point A$, успоредна на $a$
    \item Общо уравнение на правата $m$ през $\Point B$, перпендикулярна на $b$
    \item Общо уравнение на правата $AB$
    \item Координатите на $\Point B'$, която е ортогонално симетрична на $\Point B$ относно правата $a$
    \item Разстоянието от $\Point A$ до $a$
    \item Общо уравнение на ъглополовящата на правите $a$ и $b$
    \item Ъгълът между правите $a$ и $b$
    \item Общо уравнение на правата $t$ през $\Point B$ и $\Point T = a \cap b$
    \item Да се определи положението на $\Point A$ и $\Point B$ спрямо правата $a$
  \end{enumerate}
\end{exercise}

\begin{solution}
  \begin{enumerate}[label=\alph*)]
    \item Правата $l$ е успоредна на $a$, следователно тя има общо уравнение от вида
    \begin{displaymath}
      l: 3x + 4y + C = 0,
    \end{displaymath}

    където $C$ е подбрано спрямо условието $l$ да минава през $\Point A$, т.е.
    \begin{displaymath}
      C = - 3 \cdot 1 - 4 \cdot (-2) = 5.
    \end{displaymath}

    И така, $l$ има общо уравнение $l: 3x + 4y + 5 = 0$.

    \item Правата $m$ е перпендикулярна на $b$, следователно нормалните вектори на $b$ (например $N_b(5, -12)$) са колинеарни с $m$. Нека $\Point P$ има координати $(x, y)$ спрямо $K$. От условието $P \in m \iff \V{BP} \parallel N_b$ намираме общото уравнение
    \begin{displaymath}
      m: \det
      \begin{pmatrix}
        x & 5 \\
        y + 1 & -12
      \end{pmatrix}
      = 0
      \text{ или } m: 12x + 5y + 5 = 0.
    \end{displaymath}

    \item Нека $\Point P$ има координати $(x, y)$ спрямо $K$. От условието $P \in AB \iff \V{BP} \parallel \V{BA}$ намираме общото уравнение
    \begin{displaymath}
      AB: \det
      \begin{pmatrix}
        x & 1 \\
        y + 1 & -1
      \end{pmatrix}
      = 0
      \text{ или } AB: x + y + 1 = 0.
    \end{displaymath}

    \item Нека $B'$ има спрямо $K$ координати $B'(x', y')$. Първо намираме правата $BB'$ от условието $BB' \perp a$, което е еквивалентно на $BB' \parallel N_a(3, 4)$. Нека $\Point P$ има координати $(x, y)$ спрямо $K$. От условието $P \in BB' \iff \V{BP} \parallel N_a$ намираме общото уравнение
    \begin{displaymath}
      BB': \det
      \begin{pmatrix}
        x & 3 \\
        y + 1 & 4
      \end{pmatrix}
      = 0
      \text{ или } BB': 4x - 3y - 3 = 0.
    \end{displaymath}

    Координатите на пресечната точка $B_a = a \cap BB'$ (ортогоналната проекция на $B$ върху $a$) намираме от системата
    \begin{displaymath}
      \begin{cases}
        3x + 4y + 2 = 0 \mid (\;\cdot \; 3) \\
        4x - 3y - 3 = 0 \mid (\;\cdot \; 4)
      \end{cases}
      \sim
      \begin{cases}
        9x + 12y + 6 = 0 \\
        16x - 12y - 12 = 0
      \end{cases}
      \sim
      \begin{cases}
        25x = 6 \\
        12y = 16x - 12
      \end{cases}
    \end{displaymath}

    откъдето получаваме $B_a(6/25, -17/25)$.

    Остава да намерим координатите на $B'$. Имаме $\V{BB_a} = \V{B_a B'}$, откъдето
    \begin{displaymath}
      \begin{cases}
        6/25 = x' - 6/25 \\
        -17/25 + 1 = y' + 17/25
      \end{cases}
      \sim
      \begin{cases}
        x' = 12/25 \\
        y' = -34/25 + 1 = -9/25
      \end{cases}.
    \end{displaymath}

    Получихме $B'(12/25, -9/25)$.

    \item Разстоянието $\Dist(A, a)$ намираме от ориентираното разстояние $\ODist(A, a)$:
    \begin{displaymath}
      \Dist(A, a) = \Abs{\ODist(A, a)} = \frac {\Abs{F_a(A)}} {\Norm {N_a}},
    \end{displaymath}

    където $F_a(x, y) = 3x + 4y + 2$ е лявата част на зададеното общо уравнение на $a$, а $N_a(3, 4)$ е съответният нормален вектор. Директно пресмятаме разстоянието и получаваме
    \begin{displaymath}
      \Dist(A, a) = \frac {\Abs{3 \cdot 1 + 4 \cdot (-2) + 2}} 5 = \frac 3 5.
    \end{displaymath}

    \item Ъглополовящите $u_{1,2}$ на правите $a$ и $b$ се състоят от всички точки $P(x, y)$, за които $\Dist(P, a) = \Dist(P, b)$. Последното условие е еквивалентно на условието $\ODist(P, a) \pm \ODist(P, b) = 0$, откъдето намираме уравненията на ъглополовящите:
    \begin{align*}
      u_{1,2}&: \frac {3x + 4y + 2} 5 \pm \frac {5x - 12y + 1} {13} = 0, \\
      u_{1,2}&: (39x + 52y + 26) \pm (25x - 60y + 5) = 0, \\
      u_1&: 64x - 8y + 31 = 0, \\
      u_2&: 14x + 112y + 21 = 0.
    \end{align*}

    \item Ъгълът между $a$ и $b$ намираме чрез нормалните им вектори $N_a(3, 5)$ и $N_b(5, -12)$:
    \begin{displaymath}
      \angle(a, b) = \arccos \frac {\Abs{\Prod{N_a} {N_b}}} {\Norm{N_a} \Norm{N_b}} = \arccos \Abs {- \frac {33} {65}} = \arccos \frac {33} {65}
    \end{displaymath}

    \item Координатите на пресечната точка $T = a \cap b$ намираме от системата
    \begin{displaymath}
      \begin{cases}
        3x + 4y + 2 = 0 \mid (\;\cdot \; 3) \\
        5x - 12y + 1 = 0
      \end{cases}
      \sim
      \begin{cases}
        9x + 12y + 6 = 0 \\
        5x - 12y + 1 = 0
      \end{cases}
      \sim
      \begin{cases}
        14x = -7 \\
        12y = 5x + 1
      \end{cases}
    \end{displaymath}

    откъдето получаваме $T(-1/2, -1/8)$.

    Нека $\Point P$ има координати $(x, y)$ спрямо $K$. От условието $P \in t \iff \V{TP} \parallel \V{TB}$ намираме общото уравнение
    \begin{displaymath}
      t: \det
      \begin{pmatrix}
        x + 1/2 & 1/2 \\
        y + 1/8 & -7/8
      \end{pmatrix}
      =
      \frac 1 {16}
      \begin{pmatrix}
        2x + 1 & 1 \\
        8y + 1 & -7
      \end{pmatrix}
      = -14x - 8y - 8 = 0
    \end{displaymath}
    или $t: 7x + 4y + 4 = 0$.

    \item Означаваме $F_a(x, y) = 3x + 4y + 2$. Имаме, че $F_a(A) = F_a(1, -2) = -3 < 0$ и $F_a(B) = F_a(0, -1) = -2 < 0$. Двете точки не лежат върху правата $a$ и освен това $F_a(A)$ и $F_a(B)$ имат еднакви знаци, следователно $\Point A$ и $\Point B$ лежат в една и съща отворена полуравнина относно $a$.
  \end{enumerate}
\end{solution}

\bigskip
\begin{minipage}{0.5\textwidth}
    \begin{exercise}
    В равнината е зададена ортонормирана координатна система $K = Oxy$. Дадени са правите
    \begin{align*}
      &a: x + y = 0 \\
      &b: x - 3y - 2 = 0
    \end{align*}

    Светлинен лъч $l \Ray$ минава през точка $M(2, 2)$ и се отразява от правата $a$. Отразеният лъч $l' \Ray$ е колинеарен с правата $b$.

    Да се намерят уравнения спрямо $K$ на правите $l$ и $l'$, съдържащи съответно падащия лъч $l \Ray$ и отразения лъч $l' \Ray$.
  \end{exercise}
\end{minipage}
\begin{minipage}{0.5\textwidth}
  \begin{center}
    \begin{tikzpicture}[rotate=45]
      \filldraw [black] (2, 2) circle (2pt);
      \node[right] at (2, 2) {$M(2, 2)$};

      \filldraw [black] (0, 0) circle (2pt);
      \node at (1/3, 1/3) {$M_a$};

      \filldraw [black] (-2, -2) circle (2pt);
      \node[right] at (-2, -2) {$M'$};

      \filldraw [black] (1, -1) circle (2pt);
      \node at (1.5, -1.1) {$N$};

      \node[right] at (5/3, 1) {$l$};
      \draw[-{Latex[length=3mm]}] (7/3, 3) -- (1, -1);
      \draw[thick, dotted] (1, -1) -- (-1/3, -5);

      \node[right] at (3, -1/3) {$l'$};
      \draw[-{Latex[length=3mm]}] (1, -1) -- (5, 1/3);
      \draw[thick, dotted] (-3, -7/3) -- (1, -1);

      \node[right] at (-7/10, 1) {$a$};
      \draw[-] (-6/5, 6/5) -- (3, -3);

      \node[right] at (3, 1/3) {$b$};
      \draw[dotted] (-3, -5/3) -- (13/3, 7/9);
    \end{tikzpicture}
  \end{center}
\end{minipage}
\bigskip

\begin{solution}
  \mbox{}
  \begin{enumerate}
    \item Намираме координатите на ортогоналната проекция $M_a$ на точката $M$ върху $a$, използвайки нормален за $a$ вектор $N_a(1, 1)$.

    От условието $N_a \parallel \V{M M_a}$ намираме ограничението
    \begin{displaymath}
      M_a(x, y): \det
      \begin{pmatrix}
        1 & x - 2 \\
        1 & y - 2
      \end{pmatrix}
      = y - x
      = 0,
    \end{displaymath}

    а от ограничението $M_a \in a$ получаваме, че $\Point M_a$ има координати $(0, 0)$ спрямо $K$.

    \item Намираме координатите на ортогонално симетричната точка $M'(x', y')$ на $M$ относно $a$. Имаме $\V{MM_a} = \V{M_a M'}$, откъдето
    \begin{displaymath}
      \begin{cases}
        2 - 0 = 0 - x' \\
        2 - 0 = 0 - y'
      \end{cases}
      \implies
      x' = y' = -2,
    \end{displaymath}
    т.е. $M'(-2, -2)$.

    \item Намираме уравнението на правата $l'$. Тъй като $l' \parallel b$, правата $l'$ има общо уравнение
    \begin{displaymath}
      l': x - 3y + C = 0,
    \end{displaymath}

    където $C$ е подбрано спрямо условието $l'$ да минава през $\Point M'$, т.е.
    \begin{displaymath}
      C = - 1 \cdot (-2) + 3 \cdot (-2) = -4.
    \end{displaymath}

    И така, $l'$ има общо уравнение $l': x - 3y - 4 = 0$.

    \item Намираме координатите на пресечната точка $N$ на $a$ и $l'$ (а също и $l'$):
    \begin{displaymath}
      N(x, y): \begin{cases}
        x + y = 0 \\
        x - 3y - 4 = 0
      \end{cases}
      \sim
      \begin{cases}
        x = -y \\
        -4y = 4
      \end{cases}
      \sim
      \begin{cases}
        x = 1 \\
        y = -1
      \end{cases},
    \end{displaymath}

    т.е. $N(1, -1)$.

    \item Намираме общо уравнение на $l$. Използваме това, че $l$ минава през $\Point M$ и $\Point N$, т.е. $\Point P(x, y) \in l \iff \V{MP} \parallel \V{MN}$:
    \begin{displaymath}
      l: \det
      \begin{pmatrix}
        x - 2 & 1 - 2\\
        y - 2 & -1 - 2
      \end{pmatrix}
      =
      -3x + y + 4
      =
      0.
    \end{displaymath}
  \end{enumerate}
\end{solution}

\subsection{Равнина в пространството}

\begin{exercise}
  В пространството е зададена ортонормирана координатна система $K = Oxyz$. Спрямо нея точките $A$, $B$, $C$ и $D$ имат координати
  \begin{align*}
    &A(3, 1, 4) &C(1, 2, -1) \\
    &B(2, 1, 3) &D(0, -3, 2),
  \end{align*}

  а равнината $\beta$ има общо уравнение
  \begin{displaymath}
    \beta: x + y - z + 1 = 0.
  \end{displaymath}

  Да се намерят:
  \begin{enumerate}[label=\alph*)]
    \item Общо уравнение на равнината $\alpha$, минаваща през точките $A$, $B$ и $C$
    \item Параметрични уравнения на права $h$, минаваща през $\Point D$ и ортогонална на равнината $\alpha$
    \item Координатите на ортогонално симетричната на $\Point D$ спрямо $\alpha$ точка $\Point D'$
    \item Общо уравнение на равнината $\gamma$, минаваща през $\Point D$ и успоредна на $\beta$
    \item Координатите на някой вектор $v$, компланарен с $\alpha$ и $\beta$
    \item Параметрични уравнения на пресечницата $m$ на $\alpha$ и $\beta$
    \item Параметрични уравнения на права $l$, така че светлинния лъч $l\Ray$ през $\Point D$ след отразяването си от $\alpha$ (озн. отразения лъч с $l'\Ray$) пресича $\beta$ под прав ъгъл
    \item Общо уравнение на равнината $\pi_1$, минаваща през $A$ и ортогонална на правата $m$
    \item Общо уравнение на равнината $\pi_2$, съдържаща правата $l'$ и успоредна на правата $m$
    \item Общо уравнение на равнината $\pi_3$, съдържаща правата $l'$ и ортогонална на равнината $\alpha$
  \end{enumerate}
\end{exercise}

\begin{solution}
  \begin{enumerate}[label=\alph*)]
    \item Нека $\Point P(x, y, z) \in \alpha$. Тогава $\V{AP} \parallel \V{AB}(-1, 0, -1) \parallel \V{AC}(-2, 1, -5)$, което условие ни позволява да намерим общо уравнение на $\alpha$:
    \begin{multline*}
      \alpha: \det
      \begin{pmatrix}
        x - 3 & -1 & -2 \\
        y - 1 & 0 & 1 \\
        z - 4 & -1 & -5
      \end{pmatrix}
      =
      -1(z-4) + (-2)(y-1)(-1) - (x-3)(-1) - (-1)(y-1)(-5)
      = \\ =
      -z + 4 + 2y - 2 + x - 3 - 5y + 5
      =
      \boxed{x - 3y - z + 4 = 0}.
    \end{multline*}

    \item Правата $h$ е успоредна на нормалния за $\alpha$ вектор на $N_\alpha(1, -3, -1)$, следователно тя има векторно параметрично уравнение
    \begin{displaymath}
      h: \V{OD} + \lambda N_\alpha, \lambda \in \R
    \end{displaymath}
    и, съответно, скаларно параметрично уравнение
    \begin{displaymath}
      h: \begin{cases}
        x = \lambda \\
        y = -3 - 3\lambda \\
        z = 2 - \lambda
      \end{cases},
      \lambda \in \R.
    \end{displaymath}

    \item За да намерим ортогонално симетричната на $\Point D$ спрямо $\alpha$ точка $\Point D'(x', y', z')$, първо ще намерим пресечната точка $\Point D_\alpha$ на правата $h$ и равнината $\alpha$.

    Намираме стойността на параметъра $\lambda$ за уравнението на $h$, за която $h$ се пресича $\alpha$:
    \begin{align*}
      \lambda - 3(-3 - 3\lambda) - (2 - \lambda) + 4 &= 0 \\
      \lambda + 9 + 9\lambda - 2 + \lambda + 4 &= 0 \\
      11 \lambda &= -11 \\
      \lambda &= -1,
    \end{align*}

    откъдето намираме координатите $(-1, 0, 3)$ на $\Point D_\alpha$.

    След това, развиваме очевидното равенство $\V{D_\alpha D'} = \V{DD_\alpha}$ покоординатно:
    \begin{displaymath}
      \begin{cases}
        x' + 1 = -1 \\
        y' - 0 = 3 \\
        z' - 3 = 1
      \end{cases}
      \sim
      \begin{cases}
        x' = -2 \\
        y' = 3 \\
        z' = 4,
      \end{cases}
    \end{displaymath}
    т.е. $D'(-2, 3, 4)$.

    \item Тъй като равнината $\gamma$ е успоредна на $\beta$, тя има общо уравнение
    \begin{displaymath}
      \gamma: x + y - z + C = 0,
    \end{displaymath}

    където $C$ е подбрано спрямо условието $\gamma$ да минава през $\Point D$, т.е.
    \begin{displaymath}
      C = -(0 + (- 3) - 2) = 5.
    \end{displaymath}

    И така, $\gamma: x + y - z + 5 = 0$.

    \item Търсим едновременни решения на известните общи уравнения на равнините $\alpha$ и $\beta$:
    \begin{displaymath}
      \begin{cases}
        \alpha: x - 3y - z + 4 = 0 \\
        \beta: x + y - z + 1 = 0
      \end{cases}
      \sim
      \begin{cases}
        y = 3 / 4 \\
        z = x + 7 / 4.
      \end{cases}
    \end{displaymath}

    Очевидно точките $V_0\left(0, \frac 3 4, \frac 7 4 \right)$ и $V_1\left( 1, \frac 3 4, 1 + \frac 7 4 \right)$ удовлетворяват горната система, следователно векторът $v = \V{V_0 V_1}$ с координати $(1, 0, 1)$ е колинеарен и с двете равнини.

    \item Пресечницата $m$ на $\alpha$ и $\beta$ има векторно параметрично уравнение
    \begin{displaymath}
      m: \V{OV_0} + \mu v, \mu \in \R
    \end{displaymath}
    и, съответно, скаларно параметрично уравнение
    \begin{displaymath}
      m: \begin{cases}
        x = \mu \\
        y = 3/4 \\
        z = 7/4 + \mu
      \end{cases},
      \mu \in \R.
    \end{displaymath}

    \item Тъй като вече разполагаме с координатите на ортогонално симетричната на $D$ относно $\alpha$ точка $D'$ и нормалният за $\alpha$ вектор $N_\beta(1, 1, -1)$, можем да намерим векторно параметрично уравнения на правата $l'$:
    \begin{displaymath}
      l': \V{OD'} + \nu N_\beta, \nu \in \R
    \end{displaymath}
    и, съответно, скаларното параметрично уравнение
    \begin{displaymath}
      l': \begin{cases}
        x = -2 + \nu \\
        y = 3 + \nu \\
        z = 4 - \nu
      \end{cases},
      \nu \in \R.
    \end{displaymath}

    Сега намираме стойността на параметъра $\nu$, за която $l'$ пресича равнината $\alpha$:
    \begin{align*}
      (-2 + \nu) - 3(3 + \nu) - (4 - \nu) + 4 &= 0 \\
      -2 + \nu - 9 - \nu - 4 + \nu + 4 &= 0 \\
      \nu &= 11
    \end{align*}
    т.е. пресечната точка $N$ на $l'$ и $\alpha$ има координати $N(9, 14, -7)$.

    Сега намираме векторно параметрично уравнения на правата $l$, минаваща през $\Point D$ и $\Point N$:
    \begin{displaymath}
      l: \V{OD} + \xi \V{DN}, \xi \in \R
    \end{displaymath}
    и, съответно, скаларното параметрично уравнение
    \begin{displaymath}
      l: \begin{cases}
        x = 9\xi \\
        y = -3 + 17\xi \\
        z = 2 - 9\xi
      \end{cases},
      \xi \in \R.
    \end{displaymath}

    \item Първо намираме два нормални за правата $m$ вектора. Нека $u(x, y, z)$ е произволен вектор. Тъй като $v$ е направляващ за $m$, $u$ е нормален за $m$ само ако
    \begin{displaymath}
      \Prod u v = x + z = 0.
    \end{displaymath}

    От горното условие виждаме, че два нормални за $m$ вектора са $u_1(0, 1, 0)$ и $u_2(-1, 0, 1)$.

    Нека $\Point P$ има спрямо $K$ координати $(x, y, z)$. Тогава $\Point P \in \pi_1$ точно тогава, когато векторът $\V{AP}$ е колинеарен с $u_1$ и $u_2$, т.е.
    \begin{displaymath}
      \pi_1: \det
      \begin{pmatrix}
        x - 3 & 0 & -1 \\
        y - 1 & 1 & 0 \\
        z - 4 & 0 & 1
      \end{pmatrix}
      = (x - 3) - (-1)(z - 4) = \boxed{x + z - 7 = 0}.
    \end{displaymath}

    \item Имаме, че $\Point D'(-2, 3, 4) \in l'$ и векторът $N_\beta(1, 1, -1)'$ е направляващ за $l'$, а $v(1, 0, 1)$ е направляващ за $m$. Нека $\Point P$ има спрямо $K$ координати $(x, y, z)$. Тогава $\Point P \in \pi_2$ точно тогава, когато са колинеарни векторите $\V{D'P}$, $N_\beta$ и $v$, т.е.
    \begin{displaymath}
      \pi_2: \det
      \begin{pmatrix}
        x + 2 & 1  & 1 \\
        y - 3 & 1  & 0 \\
        z - 4 & -1 & 1
      \end{pmatrix}
      = (x + 2) + (y - 3)(-1) - (z - 4) - (y - 3) = \boxed{x - 2y - z + 12 = 0}.
    \end{displaymath}

    \item Аналогично на $\pi_2$, имаме, че $\Point P \in \pi_3$ точно тогава, когато са колинеарни векторите $\V{D'P}$, $N_\beta$ и $N_\alpha(1, -3, -1)$
    \begin{multline*}
      \pi_3: \det
      \begin{pmatrix}
        x + 2 & 1  & 1 \\
        y - 3 & 1  & -3 \\
        z - 4 & -1 & -1
      \end{pmatrix}
      = \\ =
      (x + 2)(-1) + (-3)(z - 4) + (y - 3)(-1) - (z - 4) - (y - 3)(-1) - (x + 2)(-3)(-1)
      = \\ =
      (-4)(x + 2) + (-4)(z - 4) = 0
      \text{ или }
      \boxed{\pi_3: x + z - 2 = 0}.
    \end{multline*}
  \end{enumerate}
\end{solution}

\subsection{Права в пространството}

\subsection{Канонизиране на крива от втора степен}

\printbibliography

\end{document}
