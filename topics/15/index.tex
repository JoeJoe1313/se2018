% arara: pdflatex: { shell: true, interaction: nonstopmode }
% arara: biber
% arara: pdflatex: { shell: true }

\documentclass[numbers=endperiod, DIV=15, bibliography=totocnumbered]{scrartcl}

% Base packages
\usepackage[T2A]{fontenc}
\usepackage[utf8]{inputenc}
\usepackage[bulgarian]{babel}
\usepackage[pdfencoding=unicode]{hyperref}
\usepackage{biblatex}
\usepackage[style=german]{csquotes}

% Base math packages
\usepackage{amsmath}
\usepackage{amssymb}
\usepackage{amsthm}
\usepackage{mathtools}

% Custom packages
\usepackage{../../common/macros}
\usepackage{../../common/theorems}

% Bibliography
\addbibresource{./references.bib}

% Document
\title{Тема 15}
\subtitle{Случайни величини с дискретни разпределения - дискретно равномерно, биномно, геометрично, поасоново разпределения. Задачи, в които възникват.}
\author{Янис Василев, \Email{ianis@ivasilev.net}}
\date{15 юни 2019}

\begin{document}

\maketitle

\section{Анотация}

От~\cite{Syllabus}.

\subsection{Теория}

\begin{enumerate}
  \item Дефиниция на дискретно разпределение на случайна величина
  \item Неотрицателност и нормираност на вероятностите на дискретна случайна величина
  \item Дефиниция на моментите на дискретна случайна величина
  \item Дефиниция и свойства (без доказателства) на пораждаща/пораждаща моментите/характеристична функция (по избор)
  \item Дефиниция, коректност, мотивиращ пример, пораждаща/пораждаща моментите/характеристична функция, очакване и дисперсия за две избрани от комисията дискретни разпределения
\end{enumerate}

\subsection{Задачи}

Не е даден списък с възможни задачи, затова съм включил разни задачи, давани на държавен изпит.

\section{Теория}

Теорията е представена с минимални препратки към теорията на мярката и е базирана частично на изложението в~\cite{Borovkov} и~\cite{DimitrovYanev}. За пълнота съм включил доказателства на основните свойства на пораждащи, пораждащи моментите и характеристична функции.

\begin{definition}
  \underline{(Реална) случайна величина} над вероятностното пространство $(\Omega, \F, \Prob)$ наричаме всяка измерима функция $\xi : \Omega \to \R$.

  Условието за измеримост на $\xi$ може да се запише така: за всяко борелово множество $B \in \BorelAlgebra(\R)$ имаме
  \begin{displaymath}
    \xi^{-1} (B) = \{ \omega \in \Omega \mid \xi(\omega) \in B \} \in \F.
  \end{displaymath}

  \underline{Разпределение на $\xi$} наричаме мярката
  \begin{displaymath}
    \Prob_\xi(A) \coloneqq \Prob(\xi \in A).
  \end{displaymath}

  Две случайни величини $\xi$ и $\eta$ наричаме \underline{независими}, ако за всички $A, B \in \F$ е изпълнено
  \begin{displaymath}
    \Prob(\xi \in A, \eta \in B) = \Prob_\xi(A) \Prob_\eta(B).
  \end{displaymath}

  Случайната величина $\xi$ наричаме \underline{дискретна} и казваме, че $\xi$ има \underline{дискретно разпределение}, ако тя приема крайно или изброимо много стойности $x_1, x_2, \ldots$ с вероятности съответно
  \begin{displaymath}
    p_k = P(\xi = x_k), k = 1, 2, \ldots.
  \end{displaymath}

  Множеството от стойности на $\xi$ ще бележим с $\Image(\xi) \coloneqq \{ x_1, x_2, \ldots \}$.
\end{definition}

\begin{theorem}\label{thm:rv-iff-pf}
  Редицата $p_1, p_2, \ldots$ са е редица от вероятности на някоя дискретна случайна величина тогава и само тогава, когато са изпълнени
  \begin{enumerate}
    \item\label{thm:rv-iff-pf.bounded} (неотрицателност) $0 \leq p_k \leq 1$ за всяко $k = 1, 2, \ldots$
    \item\label{thm:rv-iff-pf.normed} (нормираност) $\sum_k p_k = 1$.
  \end{enumerate}
\end{theorem}

Теорема~\ref{thm:rv-iff-pf} ни дава обосновка да задаваме дискретни случайни величини изцяло чрез редици от вероятности без да задаваме вероятностни пространства. По тази причина понякога разпределение на дискретна случайна величина наричаме не съответната вероятностна мярка, а редицата от вероятностите.

\begin{proof}
  ($\implies$) Нека $\xi$ е дискретна случайна величина над вероятностното пространство $(\Omega, \F, \Prob)$ със стойности $x_1, x_2, \ldots$ и вероятности $p_1, p_2, \ldots$.

  Тъй като $\Prob: \F \to [0, 1]$ е вероятностна мярка, директно получаваме $p_k = \Prob(\xi = x_k) \in [0, 1]$ за всяко $k = 1, 2, \ldots$ и
  \begin{multline*}
    \sum_k p_k
    =
    \sum_k \Prob(\xi = x_k)
    =
    \sum_k \Prob(\{ \omega \in \Omega \mid \xi(\omega) = x_k \})
    =
    \Prob(\cup_k \{ \omega \in \Omega \mid \xi(\omega) = x_k \})
    = \\ =
    \Prob(\{ \omega \in \Omega \mid \xi(\omega) \in \Image(\xi) \})
    =
    \Prob(\xi \in \Image(\xi))
    =
    1.
  \end{multline*}

  ($\impliedby$) Нека редицата $p_1, p_2, \ldots$ удовлетворява условията~\ref{thm:rv-iff-pf.bounded} и~\ref{thm:rv-iff-pf.normed}. Полагаме $\Image(\xi) \coloneqq \{ p_1, p_2, \ldots \}$ и дефинираме функцията
  \begin{displaymath}
    \xi(\omega) \coloneqq \begin{cases}
      \omega, \omega \in \Image(\xi) \\
      0, \text{ иначе},
    \end{cases}
  \end{displaymath}
  която е измерима над $(\R, \BorelAlgebra(\R))$, тъй като за $B \in \BorelAlgebra(\R)$ имаме
  \begin{enumerate}
    \item Ако $0 \in \Image(\xi)$ или $0 \not\in B$, то $f^{-1} (B) \subseteq \Image(\xi)$ е крайно или изброимо безкрайно множество, следователно $f^{-1} (B) \in \BorelAlgebra(\R)$.
    \item Ако $0 \not\in \Image(\xi)$ и $0 \in B$, имаме
    \begin{displaymath}
      f^{-1} (B)
      =
      f^{-1} (\{ 0 \} \cup (B \setminus \{ 0 \}))
      =
      f^{-1} (\{ 0 \}) \cup f^{-1} (B \setminus \{ 0 \})
      =
      (\R \setminus \Image(\xi)) \cup f^{-1} (B \setminus \{ 0 \}),
    \end{displaymath}
    което е обединение на две борелови множества, следователно $f^{-1} (B) \in \BorelAlgebra(\R)$.
  \end{enumerate}

  Сега дефинираме $\Prob(A) \coloneqq \sum_{\omega \in A} \xi(\omega)$. Тази сума е добре дефинирана, тъй като $\xi$ има не повече от изброимо много ненулеви стойности $p_1, p_2, \ldots$ и редът $\sum_k p_k$ е абсолютно сходящ според~\ref{thm:rv-iff-pf.bounded} и~\ref{thm:rv-iff-pf.normed}.

  $\Prob$ е вероятностна мярка над $(\R, \BorelAlgebra(\R))$, тъй като
  \begin{enumerate}
    \item $\Prob(\varnothing) = 0$,
    \item $\Prob(A) = \sum_{\omega \in A} \xi(\omega) \geq 0$ за произволно събитие $A \in \BorelAlgebra(\R)$ според свойство~\ref{thm:rv-iff-pf.bounded},
    \item $\Prob(\cup_{i=1}^\infty A_i) = \sum_{\omega \in \cup_{i=1}^\infty A_i} \xi(\omega) = \sum_{i=1}^\infty \sum_{\omega \in A_i} \xi(\omega) = \sum_{i=1}^\infty \Prob(A_i)$,
    \item $\Prob(A) \leq 1$ за произволно събитие $A \in \BorelAlgebra(\R)$ според доказаната адитивност и свойство~\ref{thm:rv-iff-pf.normed}.
  \end{enumerate}
  Така построихме дискретна случайна величина $\xi$ над пространството $(\R, \BorelAlgebra(\R), \Prob)$ с вероятности $p_1, p_2, \ldots$.
\end{proof}

\begin{proposition}
  Дискретните случайни величини над едно вероятностно пространство $(\Omega, \F, \Prob)$ образуват линейно пространство относно операциите събиране и умножение с число.
\end{proposition}
\begin{proof}
  Ще докажем само затвореността относно операциите, тъй като останалите аксиоми за линейно пространство са изпълнени по тривиални причини.

  Сумата $\xi + \eta$ на две дискретни случайни величини приема стойности $x + y$ за $x \in \Image(\xi)$ и $y \in \Image(\eta)$, които са не повече от изброимо много, следователно $\xi + \eta$ също е дискретна случайна величина.

  Произведението $c \xi$ за $c \in \R$ приема или само една стойност при $c = 0$, или същият брой стойности като $\xi$, следователно $c \xi$ също е дискретна случайна величина.
\end{proof}

До края на темата ще считаме, че работим над вероятностното пространство $(\Omega, \F, \Prob)$.

\begin{definition}
  Нека $\xi$ е дискретна случайна величина със стойности $x_1, x_2, \ldots$. Дефинираме \underline{очакване на $\xi$} чрез
  \begin{displaymath}
    \Expect(\xi) \coloneqq \sum_k x_k \Prob(\xi = x_k) = \sum_{\omega \in \Omega} \xi(\omega) P(\{ \omega \}).
  \end{displaymath}
  Казваме, че $\xi$ има (крайно) очакване, ако горният ред е абсолютно сходящ.

  Случайни величини с очакване нула наричаме \underline{центрирани}.

  Очакване от константа $x \in \Omega$ дефинираме да бъде самата константа $x$.
\end{definition}

\begin{note}
  Формално, полагаме $\Expect(x) \coloneqq \Expect(\Ind_x)$, където $\Ind_x$ е индикатор за $x$, т.е.
  \begin{displaymath}
    \Prob(I_x = y) = \delta_{xy} = \begin{cases}
      1, x = y, \\
      0, x \neq y.
    \end{cases}
  \end{displaymath}

  Оттук директно следва $\Expect(x) = \Expect(\Ind_x) = x$.
\end{note}

\begin{proposition}\label{thm:expect-independent}
  За независими дискретни случайни величини $\xi$ и $\eta$ с крайно очакване е изпълнено
  \begin{displaymath}
    \Expect(\xi \eta) = \Expect(\xi) \Expect(\eta).
  \end{displaymath}
\end{proposition}
\begin{proof}
  Нека $x_1, x_2, \ldots$ и $y_1, y_2, \ldots$ са стойностите съответно на $\xi$ и $\eta$. Тогава
  \begin{multline*}
    \Expect(\xi) \Expect(\eta)
    =
    \left( \sum_i x_i \Prob(\xi = x_i) \right) \left( \sum_j y_j \Prob(\eta = y_j) \right)
    =
    \sum_{k=1}^\infty \sum_{m=1}^k x_m y_{k-m} \Prob(\xi = x_m) \Prob(\eta = y_{k-m})
    = \\ =
    \sum_{k=1}^\infty \sum_{m=1}^k x_m y_{k-m} \Prob(\xi = x_m, \eta = y_{k-m})
    =
    \sum_{i, j} x_i y_j \Prob(\xi = x_i, \eta = y_j)
    =
    \Expect(\xi \eta).
  \end{multline*}
\end{proof}

\begin{proposition}\label{thm:expect-linear}
  Очакването е линеен функционал над подпространството от случайни величини над $(\Omega, \F, \Prob)$, за които очакването съществува.
\end{proposition}
\begin{proof}
  Директно проверяваме дефиницията за линеен функционал:
  \begin{enumerate}
    \item За две дискретни случайни $\xi$ и $\eta$ величини с крайно очакване имаме
    \begin{multline*}
      \Expect(\xi + \eta)
      =
      \sum_{\omega \in \Omega} (\xi + \eta) (\omega) P(\{ \omega \})
      =
      \sum_{\omega \in \Omega} [\xi (\omega) + \eta (\omega)] P(\{ \omega \})
      = \\ =
      \sum_{\omega \in \Omega} \xi (\omega) P(\{ \omega \}) + \sum_{\omega \in \Omega} \eta (\omega) P(\{ \omega \})
      =
      \Expect(\xi) + \Expect(\eta).
    \end{multline*}

    \item За дискретна случайна величина $\xi$ със стойности $x_1, x_2, \ldots$ и крайно очакване и константа $c \in \R$ по твърдение~\ref{thm:lotus} имаме
    \begin{displaymath}
      \Expect(c \xi)
      =
      \sum_k (c x_k) P(\xi = x_k)
      =
      c \sum_k x_k P(\xi = x_k)
      =
      c \Expect(\xi).
    \end{displaymath}
  \end{enumerate}

  Видяхме и това, че дискретните случайни величини с крайно очакване са затворени относно събиране и умножение с число, с което доказваме, че те образуват линейно подпространство на всички дискретни случайни величини.
\end{proof}

\begin{proposition}\label{thm:lotus}
  Ако $\xi$ е (реална) дискретна случайна величина със стойности $x_1, x_2, \ldots$ и $\psi: \Complex \to \Complex$ е измерима функция, то $\psi(\xi)$ е (комплексна) случайна и е изпълнено
  \begin{displaymath}
    \Expect(\psi(\xi))
    =
    \sum_k \psi(x_k) P(\xi = x_k),
  \end{displaymath}
  при условие, че горният ред а абсолютно сходящ.
\end{proposition}

\begin{note}
  Допускаме $\phi$ да бъде комплексна функция, тъй като това ни е необходимо за дефинирането на характеристични функции.
\end{note}

\begin{proof}
  Образът на крайно или измеримо множество е най-много измеримо, затова $\psi(\xi)$ приема не повече от изброимо много различни стойности и следователно е дискретна случайна величина. Нека стойностите на $\psi(\xi)$ са $y_1, y_2, \ldots$. Праобразите $\psi^{-1}(y_i)$ на $y_i, i = 1, 2, \ldots$ са крайни или изброими, непресичащи се и $\Image(\xi) = \cup_i \psi^{-1}(y_i)$. Имаме
  \begin{multline*}
    \Expect(\psi(\xi))
    =
    \sum_i y_i P(\psi(\xi) = y_i)
    =
    \sum_i y_i \sum_{x \in \psi^{-1}(y_i)} P(\xi = x)
    = \\ =
    \sum_i \sum_{x \in \psi^{-1}(y_i)} \psi(x) P(\xi = x)
    =
    \sum_k \psi(x_k) P(\xi = x_k).
  \end{multline*}
\end{proof}

Доказаните в твърдения~\ref{thm:expect-independent},~\ref{thm:expect-linear} и~\ref{thm:lotus} свойства на очакването значително опростяват работата с него.

\begin{definition}
  \underline{Дисперсия на случайната величина $\xi$} наричаме
  \begin{displaymath}
    \Var(\xi)
    \coloneqq
    \Expect \left({(\xi - \Expect(\xi))}^2 \right)
    =
    \Expect(\xi^2 - 2 \Expect(\xi) \xi + {\Expect(\xi)}^2)
    =
    \Expect(\xi^2) - 2 {\Expect(\xi)}^2 + {\Expect(\xi)}^2
    =
    \Expect(\xi^2) - {\Expect(\xi)}^2.
  \end{displaymath}

  Числото $\Expect(\xi^n)$ наричаме \underline{$n$-ти момент на $\xi$}, а $\Expect \left( {(\xi - \Expect(\xi))}^n \right)$ наричаме \underline{$n$-ти централен момент на $\xi$}.

  Дисперсията всъщност е просто вторият централен момент.

  Случайни величини с дисперсия единица наричаме \underline{нормирани}, тъй като дисперсията играе ролята на норма в пространството на случайни величини с краен втори момент.
\end{definition}

\begin{proposition}\label{thm:lower-order-moments}
  Ако $\Expect(\xi^n)$ съществува, съществуват и моментите от по-нисък ред $\Expect(\xi^k), k = 0, \ldots, n - 1$.
\end{proposition}
\begin{proof}
  Първо да забележим, че за $y \in (0, 1)$ имаме ${\Prob(\xi = x_k)}^y < \Prob(\xi = x_k)$. Прилагаме неравенството на Йенсен за редове:
  \begin{multline*}
    \Expect(\xi^{n-1})
    \leq
    {\Expect(\xi^{n-1})}^{\frac n {n-1}}
    =
    {\left( \sum_k {x_k}^{n-1} \Prob(\xi = x_k) \right)}^{\frac n {n-1}}
    \leq
    \sum_k {\left({x_k}^{n-1} \Prob(\xi = x_k) \right)}^{\frac n {n-1}}
    < \\ <
    \sum_k {\left({x_k}^{n-1}\right)}^{\frac n {n-1}} \Prob(\xi = x_k)
    =
    \sum_k {x_k}^n \Prob(\xi = x_k)
    =
    \Expect(\xi^n).
  \end{multline*}
\end{proof}

\begin{definition}
  \underline{Пораждаща функция на $\xi$} наричаме
  \begin{displaymath}
    \PGF_\xi (z) \coloneqq \Expect(z^\xi).
  \end{displaymath}

  \underline{Пораждаща моментите функция на $\xi$} наричаме
  \begin{displaymath}
    \MGF_\xi (t) \coloneqq \Expect(e^{t\xi}).
  \end{displaymath}

  \underline{Характеристична функция на $\xi$} наричаме
  \begin{displaymath}
    \Char_\xi (t) \coloneqq \Expect(e^{it\xi}).
  \end{displaymath}

  Изпълнено е $\Char_\xi(t) = \MGF_\xi(it) = \PGF_\xi(e^{it})$.
\end{definition}

\begin{note}
  Дефинициите за моменти и функции от очакването се пренасят без изменение за случайни величини, които не са дискретни. Пораждащите функции, обаче, се полезни само за случайни величини, които приемат неотрицателни цели стойности.
\end{note}

\begin{theorem}[Свойства на пораждащите функции]
  Нека $\xi$ и $\eta$ са независими целочислени случайни величини със стойности $0, 1, \ldots$ (възможно е $P(\xi = k) > 0$ само за краен брой $k$).

  За пораждащата функция $\PGF_\xi$ са изпълнени следните свойства
  \begin{enumerate}
    \item $\PGF_\xi(z)$ е аналитична функция поне в $-1 < z < 1$.
    \item Ако пораждащите функции на $\xi$, $\eta$ и $\xi + \eta$ съществуват в точка $z \in \R$, имаме
    \begin{displaymath}
      \PGF_{\xi + \eta}(z) = \PGF_\xi(z) \PGF_\eta(z)
    \end{displaymath}

    \item $P(\xi = m) = \frac {\PGF_\xi^{(m)}} {m!} (0)$, където $\PGF_\xi^{(m)}$ е $m$-тата производна на $\PGF_\xi$.
    \item Ако пораждащите функции на две целочислени дискретни случайни величини съвпадат, самите случайни величини съвпадат.
  \end{enumerate}
\end{theorem}
\begin{proof}
  \mbox{}
  \begin{enumerate}
    \item $\PGF_\xi(z)$ се дефинира чрез степенен ред. За да докажем, че тя е аналитична в някоя област, е достатъчно да покажем сходимост на степенния ред в тази област. Използваме това, че за произволно събитие $A$ имаме $0 \leq \Prob(A) \leq 1$, и оценяваме редът отгоре
    \begin{displaymath}
      \Abs{\PGF_\xi(z)}
      =
      \Abs{\Expect(z^\xi)}
      =
      \Abs{\sum_{k=0}^\infty z^k \Prob(\xi = k)}
      \leq
      \sum_{k=0}^\infty \Abs{z^k \Prob(\xi = k)}
      =
      \sum_{k=0}^\infty \Abs{z}^k \Prob(\xi = k)
      \leq
      \sum_{k=0}^\infty \Abs{z}^k.
    \end{displaymath}
    Последният ред е сходящ при $\Abs z < 1$.

    \item Тъй като $\xi$ и $\eta$ са независими, за произволно $z \in \R$ величините $z^\xi$ и $z^\eta$ са независими и
    \begin{displaymath}
      \PGF_{\xi + \eta}(z)
      =
      \Expect(z^{\xi + \eta})
      =
      \Expect(z^\xi z^\eta)
      =
      \Expect(z^\xi) \Expect(z^\eta)
      =
      \PGF_\xi(z) \PGF_\eta(z).
    \end{displaymath}

    \item Разглеждаме степенния ред на $\PGF_\xi$:
    \begin{displaymath}
      \PGF_\xi(z)
      =
      \sum_{k=0}^\infty z^k \Prob(\xi = k)
      =
      \sum_{k=0}^{m-1} z^k \Prob(\xi = k) + z^m \Prob(\xi = m) + \sum_{k=m+1}^\infty z^k \Prob(\xi = k).
    \end{displaymath}

    Аналитичността на $\PGF_\xi$ ни позволява да диференцираме очакването почленно. След $m$-кратно диференциране получаваме
    \begin{displaymath}
      \PGF_\xi^{(m)}(z)
      =
      \sum_{k=0}^{m-1} k! \cdot 0 \Prob(\xi = k) + m! \Prob(\xi = m) + \sum_{k=m+1}^\infty (k-m) \cdots (k-1) k z^{k-m} \Prob(\xi = k).
    \end{displaymath}

    Следователно $\PGF_\xi^{(m)}(0) = m! \Prob(\xi = m)$.

    \item Директно следва от изразяването на стойностите на $\xi$ и $\eta$ чрез производните на пораждащите ги функции.
  \end{enumerate}
\end{proof}

\begin{theorem}[Свойства на пораждащите моментите функции]
  Нека $\xi$ и $\eta$ са независими дискретни случайни величини.

  За пораждащите моментите функции $\MGF_\xi$ и $\MGF_\eta$ са изпълнени следните свойства
  \begin{enumerate}
    \item В общия случай пораждащата моментите функция съществува само в $0$. Ако тя съществува в околност на $0$, то тя е гладка в тази околност, съществуват всички моменти и е изпълнено $\Expect(\xi^m) = \MGF_\xi^{(m)} (0)$ за $m = 1, 2, \ldots$.

    \item Ако пораждащите моментите функции на $\xi$, $\eta$ и $\xi + \eta$ съществуват в точка $t \in \R$, имаме
    \begin{displaymath}
      \MGF_{\xi + \eta}(t) = \MGF_\xi(t) \MGF_\eta(t).
    \end{displaymath}

    \item Ако $\MGF_\xi$ и $\MGF_\eta$ имат обща дефиниционна област, в която те съвпадат, то $\xi$ и $\eta$ също съвпадат.
  \end{enumerate}
\end{theorem}
\begin{proof}
  Нека стойностите на $\xi$ са $x_1, x_2, \ldots$.

  \mbox{}
  \begin{enumerate}
    \item Ако пораждащите моментите функции съществуват в $t \in \R$, тъй като $\xi$ и $\eta$ са независими, случайните величини $e^{t\xi}$ и $e^{t\eta}$ също са независими и
    \begin{displaymath}
      \MGF_{\xi + \eta}(t)
      =
      \Expect(e^{t(\xi + \eta)})
      =
      \Expect(e^{t\xi} e^{t\eta})
      =
      \Expect(e^{t\xi}) \Expect(e^{t\eta})
      =
      \MGF_\xi(t) \MGF_\eta(t).
    \end{displaymath}

    \item Пораждащата моментите функция винаги съществува в $0$, тъй като $\MGF_\xi(0) = \Expect(e^{0\xi}) = \Expect(1) = 1$.

    Нека $\MGF_\xi$ съществува в околност $U$ на $0$. Без ограничение на общността ще считаме, че $U$ е ограничена. Полагаме $\tau \coloneqq \min(-\inf U, \sup_U)$. Тогава сумата $\MGF_\xi(-\tau) + \MGF_\xi(\tau)$ е крайна. Развиваме тази сума в ред на Тейлър:
    \begin{displaymath}
      \MGF_\xi(-\tau) + \MGF_\xi(\tau)
      =
      \Expect(e^{-\tau\xi} + e^{\tau\xi})
      =
      \Expect\left( 2 \sum_{k=0}^\infty \frac {\xi^{2k}} {(2k)!} \tau^{2k} \right)
      =
      2 \sum_{k=0}^\infty \frac {\Expect(\xi^{2k})} {(2k)!} \tau^{2k}.
    \end{displaymath}

    Внасянето на очакването е възможно, защото всички членове на реда са неотрицателни. От $\Expect(\xi^{2m}) = \Expect(\Abs{\xi^{2m}})$ се вижда, че всички четни моменти съществуват. Според твърдение~\ref{thm:lower-order-moments} съществуват и всички нечетни моменти.

    Остава да докажем, че са налице условията за $m$-кратно почленно диференциране на реда $\Char_\xi(t) = \Expect (e^{t\xi})$ в областта $\frac 1 {2^m} U = \{ \frac t {2^m} U \mid t \in U \}$.

    Разглеждаме очакването на $m$-тата производна $\xi^m e^{t \xi}$ на $e^{t \xi}$ по $t$. Налице са условията за почленно диференциране на ред:
    \begin{enumerate}
      \item За $m > 0$ прилагаме неравенството на Коши-Буняковски-Шварц, за да докажем, че $\Expect \left( \xi^m e^{t \xi} \right)$ съществува за $t \in \frac 1 {2^m} U$:
      \begin{multline*}
        {\left(\Expect \left( \xi^m e^{t \xi} \right) \right)}^2
        =
        {\left(\sum_k x_k^m e^{t x_k} \Prob(\xi = x_k) \right)}^2
        = \\ =
        {\left(\sum_k (x_k^m \sqrt{\Prob(\xi = x_k)}) (e^{t x_k} \sqrt{\Prob(\xi = x_k)}) \right)}^2
        \leq \\ \leq
        \sum_k x_k^{2m} \Prob(\xi = x_k)
        \sum_k e^{2t x_k} \Prob(\xi = x_k)
        = \\ =
        \Expect(\xi^{2m})
        \Expect(e^{2t \xi})
        =
        \Expect(\xi^{2m})
        \Char_\xi(2t).
      \end{multline*}

      \item Функцията $x \mapsto x^m e^{t x}$ е диференцируема по $t$ за всяко $x \in \Image(\xi)$.
      \item Производната на $x^m e^{t x}$ по $t$ се мажорира по абсолютна стойност от
      \begin{displaymath}
        \Abs{x^{m+1} e^{t x}}
        \leq
        {\Abs{x}}^{m+1} \Abs{e^{t x}}
        \leq
        {\Abs{x}}^{m+1} e^{\Abs{t x}}
        \leq
        {\Abs{x}}^{m+1} e^{\tau \Abs{x}},
      \end{displaymath}
      където мажорантата не зависи от $t$.
    \end{enumerate}

    По индукция за $m = 1, 2, \ldots$ получаваме $\MGF^{(m)}_\xi(t) = i^m \Expect(\xi^m e^{t \xi})$ в $\frac 1 {2^m} U$. В частност, $\MGF^{(m)}_\xi(0) = \Expect(\xi^m e^{0 \xi}) = \Expect(\xi^m)$.

    \item Нека $z_1, z_2, \ldots$ е обединение на стойностите на $\xi$ и $\eta$. Ако функциите $\MGF_\xi$ и $\MGF_\eta$ съвпадат в областта си на дефиниция $U$, за $t \in U$ имаме
    \begin{align*}
      \MGF_\xi(t) - \MGF_\eta(t) &= 0
      \\
      \sum_k e^{t z_k} \Prob(\xi = z_k) - \sum_k e^{t z_k} \Prob(\eta = z_k) &= 0
      \\
      \sum_k e^{t z_k} (\Prob(\xi = z_k) - \Prob(\eta = z_k)) &= 0.
    \end{align*}
    Последното равенство е изпълнено за всяко $t \in U$ точно когато $\Prob(\xi = z) = \Prob(\xi = z)$ за всяко $z \in \Image(\xi) \cup \Image(\eta)$. Следователно $\xi$ и $\eta$ приемат едни и същи стойности $\Image(\xi) = \Image(\eta)$ с една и съща вероятност и тъй като и двете са дискретни, те съвпадат.
  \end{enumerate}
\end{proof}

\begin{theorem}[Свойства на характеристичните функции]
  Нека $\xi$ и $\eta$ са независими дискретни случайни величини.

  За характеристичните функции $\Char_\xi$ и $\Char_\eta$ са изпълнени следните свойства
  \begin{enumerate}
    \item $\Char_\xi$ съществува и е равномерно непрекъсната навсякъде върху реалната права.

    \item Ако $\xi^n$ има краен $n$-ти момент, е изпълнено $\Expect(\xi^m) = i^{-m} \Char_\xi^{(m)} (0)$ за $m = 1, \ldots, n$.

    \item За всяко $t \in \R$ имаме
    \begin{displaymath}
      \Char_{\xi + \eta}(t) = \Char_\xi(t) \Char_\eta(t).
    \end{displaymath}

    \item Ако $\Char_\xi$ и $\Char_\eta$ съвпадат, то $\xi$ и $\eta$ също съвпадат.
  \end{enumerate}
\end{theorem}
\begin{proof}
  Нека стойностите на $\xi$ са $x_1, x_2, \ldots$.

  \begin{enumerate}
    \item За да докажем, че $\Char_\xi$ е дефинирана навсякъде в $\R$, оценяваме отгоре абсолютната стойност на $\Char_\xi$ за $t \in \R$:
    \begin{displaymath}
      \Abs{\Char_\xi(t)}
      =
      \Abs{\Expect(e^{it\xi})}
      =
      \Abs{\sum_k e^{it x_k} \Prob(\xi = x_k)}
      \leq
      \sum_k \Abs{e^{it x_k}} \Prob(\xi = x_k)
      =
      \sum_k \Prob(\xi = x_k)
      =
      1.
    \end{displaymath}

    За да докажем и равномерната непрекъснатост в $\R$, първо оценяваме отгоре израза
    \begin{multline*}
      \Abs{\Char_\xi(t + h) - \Char_\xi(t)}
      =
      \Abs{\Expect(e^{i(t + h)\xi}) - \Expect(e^{it\xi})}
      = \\ =
      \Abs{\Expect(e^{it\xi} (e^{ih\xi} - 1))}
      =
      \Abs{\sum_k e^{it x_k} (e^{ih x_k} - 1) \Prob(\xi = x_k)}
      \leq \\ \leq
      \sum_k \Abs{e^{it x_k}} \Abs{e^{ih x_k} - 1} \Prob(\xi = x_k)
      =
      \sum_k \Abs{e^{ih x_k} - 1} \Prob(\xi = x_k).
    \end{multline*}

    Фиксираме $\varepsilon > 0$. Избираме константа $c_\varepsilon \in \R$ такава, че $\Prob(\Abs{\xi} > c_\varepsilon) < \frac \varepsilon 3$.
    Въвеждаме две множества от индекси: $A \coloneqq \{ k = 1, 2, \ldots \mid \Abs{x_k} \leq c_\varepsilon \}$ и $B = \ZPos \setminus A$.

    Ще използваме, че за всяко $z \in \Complex$ неравенството на Йенсен ни дава
    \begin{displaymath}
      \Abs{e^{iz} - 1}
      =
      \Abs{i \int_0^z e^{it} dt}
      \leq
      \int_0^{\Abs{z}} \Abs{e^{it}} dt
      =
      \int_0^{\Abs{z}} dt
      =
      \Abs{z}.
    \end{displaymath}

    За $k \in A$ имаме
    \begin{displaymath}
      \sum_{k \in A} \Abs{e^{ih x_k} - 1} \Prob(\xi = x_k)
      \leq
      \sum_{k \in A} \Abs{h x_k} \Prob(\xi = x_k)
      \leq
      c_\varepsilon \Abs h \sum_{k \in A} \Prob(\xi = x_k)
      \leq
      c_\varepsilon \Abs h.
    \end{displaymath}

    За $k \in B$ имаме

    \begin{displaymath}
      \sum_{k \in B} \Abs{e^{ih x_k} - 1} \Prob(\xi = x_k)
      \leq
      \sum_{k \in B} \left( \Abs{e^{ih x_k}} + 1 \right) \Prob(\xi = x_k)
      =
      2 \sum_{k \in B} \Prob(\xi = x_k)
      <
      \frac {2\varepsilon} 3.
    \end{displaymath}

    За целият ред тогава получаваме
    \begin{displaymath}
      \sum_k \Abs{e^{ih x_k} - 1} \Prob(\xi = x_k)
      <
      c_\varepsilon \Abs{h} + 2 \varepsilon.
    \end{displaymath}

    Полагаме $\delta = \frac \varepsilon {3 c_\varepsilon}$.

    Тогава за $\Abs h < \delta$ имаме
    \begin{displaymath}
      \Abs{\Char_\xi(t + h) - \Char_\xi(t)}
      <
      c_\varepsilon \Abs{h} + \frac {2\varepsilon} 3
      <
      \frac {\varepsilon} 3 + \frac {2\varepsilon} 3
      =
      \varepsilon.
    \end{displaymath}

    Числото $\delta$ зависи само от $\varepsilon$, следователно $\Char_\xi(t)$ е равномерно непрекъсната върху цялата реална права.

    \item Нека съществува моментът $\Expect \xi^n$. Тогава съществуват и моментите от по-нисък ред.

    Ще докажем, че са налице условията за $m$-кратно почленно диференциране на реда $\Char_\xi(t) = \Expect (e^{it \xi})$.

    Разглеждаме очакването на $m$-тата производна $i^m \xi^m e^{it \xi}$ на $e^{it \xi}$ по $t$. Налице условията за почленно диференциране на ред:
    \begin{enumerate}
      \item $\Expect \left( i^m \xi^m e^{it \xi} \right)$ съществува за $t \in \R$, тъй като
      \begin{multline*}
        \Abs{\Expect \left( i^m \xi^m e^{it \xi} \right)}
        =
        \Abs{\Expect \left( \xi^m e^{it \xi} \right)}
        =
        \Abs{\sum_k x_k^m e^{it x_k} \Prob(\xi = x_k)}
        \leq
        \sum_k \Abs{x_k^m e^{it x_k} \Prob(\xi = x_k)}
        = \\ =
        \sum_k \Abs{x_k^m} \Abs{e^{it x_k}} \Prob(\xi = x_k)
        =
        \sum_k \Abs{x_k^m} \Prob(\xi = x_k)
        =
        \Expect \left( {\Abs{\xi^m}} \right).
      \end{multline*}
      Последното очакване е крайно, тъй като $\Expect(\Abs{\xi^0}) = 1$, а за $m > 0$ по условие редът $\Expect(\xi^m)$ е абсолютно сходящ.
      \item Функцията $x \mapsto i^m x^m e^{it x}$ е диференцируема по $t$ за всяко $x \in \Image(\xi)$.
      \item Производната на $i^m x^m e^{it x}$ по $t$ се мажорира по абсолютна стойност от $\Abs{x^{m+1} e^{it x}} \leq {\Abs{x}}^{m+1}$, където мажорантата не зависи от $t$.
    \end{enumerate}

    По индукция за $m = 1, \ldots, n$ получаваме $\Char^{(m)}_\xi(t) = i^m \Expect(\xi^m e^{t \xi})$. В частност, $\Char^{(m)}_\xi(0) = i^m \Expect(\xi^m e^{0 \xi}) = i^m \Expect(\xi^m)$.

    \item Тъй като $\xi$ и $\eta$ са независими, за произволно $t \in \R$ величините $e^{t\xi}$ и $e^{t\eta}$ са независими и
    \begin{displaymath}
      \Char_{\xi + \eta}(t)
      =
      \Expect(e^{it(\xi + \eta)})
      =
      \Expect(e^{it\xi} e^{it\eta})
      =
      \Expect(e^{it\xi}) \Expect(e^{it\eta})
      =
      \Char_\xi(t) \Char_\eta(t).
    \end{displaymath}

    \item Нека $z_1, z_2, \ldots$ е обединение на стойностите на $\xi$ и $\eta$. Ако функциите $\Char_\xi$ и $\Char_\eta$ съвпадат, имаме
    \begin{align*}
      \Char_\xi(t) - \Char_\xi(t) &= 0
      \\
      \sum_k e^{i t z_k} \Prob(\xi = z_k) - \sum_k e^{i t z_k} \Prob(\xi = z_k) &= 0
      \\
      \sum_k e^{i t z_k} (\Prob(\xi = z_k) - \Prob(\xi = y_k)) &= 0.
    \end{align*}
    Последното равенство е изпълнено за всяко $t \in \Complex$ точно когато $\Prob(\xi = z) = \Prob(\xi = z)$ за всяко $z \in \Image(\xi) \cup \Image(\eta)$. Следователно $\xi$ и $\eta$ приемат едни и същи стойности $\Image(\xi) = \Image(\eta)$ с една и съща вероятност и тъй като и двете са дискретни, те съвпадат.
  \end{enumerate}
\end{proof}

\section{Задачи}

\printbibliography

\end{document}
